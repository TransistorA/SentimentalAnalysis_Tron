%* glpk.tex *%

%***********************************************************************
%  This code is part of GLPK (GNU Linear Programming Kit).
%
%  Copyright (C) 2000, 2001, 2002, 2003, 2004, 2005, 2006, 2007, 2008,
%  2009, 2010, 2011, 2013, 2014, 2015, 2016, 2017 Andrew Makhorin,
%  Department for Applied Informatics, Moscow Aviation Institute,
%  Moscow, Russia. All rights reserved. E-mail: <mao@gnu.org>.
%
%  GLPK is free software: you can redistribute it and/or modify it
%  under the terms of the GNU General Public License as published by
%  the Free Software Foundation, either version 3 of the License, or
%  (at your option) any later version.
%
%  GLPK is distributed in the hope that it will be useful, but WITHOUT
%  ANY WARRANTY; without even the implied warranty of MERCHANTABILITY
%  or FITNESS FOR A PARTICULAR PURPOSE. See the GNU General Public
%  License for more details.
%
%  You should have received a copy of the GNU General Public License
%  along with GLPK. If not, see <http://www.gnu.org/licenses/>.
%***********************************************************************

%  To produce glpk.pdf from glpk.tex run the following two commands:
%     latex glpk.tex
%     dvipdfm -p letter glpk.dvi
%  Note: You need TeX Live 2010 or later version.

\documentclass[11pt]{report}
\usepackage{amssymb}
\usepackage[dvipdfm,linktocpage,colorlinks,linkcolor=blue,
urlcolor=blue]{hyperref}
\usepackage{indentfirst}
\usepackage{lscape}
\usepackage{niceframe}
\usepackage[all]{xy}

% US Letter = 8.5 x 11 in
\setlength{\textwidth}{6.5in}
\setlength{\textheight}{9in}
\setlength{\oddsidemargin}{0in}
\setlength{\topmargin}{0in}
\setlength{\headheight}{0in}
\setlength{\headsep}{0in}
%\setlength{\footskip}{0.5in}
\setlength{\parindent}{16pt}
\setlength{\parskip}{5pt}
\setlength{\topsep}{0pt}
\setlength{\partopsep}{0pt}
%\setlength{\itemsep}{\parskip}
%\setlength{\parsep}{0pt}
%\setlength{\leftmargini}{\parindent}
%\renewcommand{\labelitemi}{---}

\newcommand{\Item}[1]{\parbox[t]{\parindent}{#1}}
\def\para#1{\noindent{\bf#1}}
\def\synopsis{\para{Synopsis}}
\def\description{\para{Description}}
\def\note{\para{Note}}
\def\returns{\para{Returns}}

\renewcommand\contentsname{\sf\bfseries Contents}
\renewcommand\chaptername{\sf\bfseries Chapter}
\renewcommand\appendixname{\sf\bfseries Appendix}

\newenvironment{retlist}
{  \def\arraystretch{1.5}
   \noindent
   \begin{tabular}{@{}p{1in}@{}p{5.5in}@{}}
}
{\end{tabular}}

\begin{document}

\thispagestyle{empty}

\artdecoframe{

\begin{center}

\vspace*{1.5in}

\begin{huge}
\sf\bfseries GNU Linear Programming Kit
\end{huge}

\vspace{0.5in}

\begin{LARGE}
\sf Reference Manual
\end{LARGE}

\vspace{0.5in}

\begin{LARGE}
\sf for GLPK Version 4.64
\end{LARGE}

\vspace{0.5in}
\begin{Large}
\sf (DRAFT, November 2017)
\end{Large}
\end{center}

\vspace*{3.8in}
}

\newpage

\vspace*{1in}

\vfill

\noindent
The GLPK package is part of the GNU Project released under the aegis of
GNU.

\noindent
Copyright \copyright{} 2000, 2001, 2002, 2003, 2004, 2005, 2006, 2007,
2008, 2009, 2010, 2011, 2013, 2014, 2015, 2016, 2017 Andrew Makhorin,
Department for Applied Informatics, Moscow Aviation Institute, Moscow,
Russia. All rights reserved.

\noindent
Free Software Foundation, Inc., 51 Franklin St, Fifth Floor, Boston, MA
02110-1301, USA.

\noindent
Permission is granted to make and distribute verbatim copies of this
manual provided the copyright notice and this permission notice are
preserved on all copies.

\noindent
Permission is granted to copy and distribute modified versions of this
manual under the conditions for verbatim copying, provided also that
the entire resulting derived work is distributed under the terms of
a permission notice identical to this one.

\noindent
Permission is granted to copy and distribute translations of this
manual into another language, under the above conditions for modified
versions.

%%%%%%%%%%%%%%%%%%%%%%%%%%%%%%%%%%%%%%%%%%%%%%%%%%%%%%%%%%%%%%%%%%%%%%%%

\newpage

{\setlength{\parskip}{0pt}\tableofcontents}

%%%%%%%%%%%%%%%%%%%%%%%%%%%%%%%%%%%%%%%%%%%%%%%%%%%%%%%%%%%%%%%%%%%%%%%%

%* glpk01.tex *%

\chapter{Introduction}

GLPK (\underline{G}NU \underline{L}inear \underline{P}rogramming
\underline{K}it) is a set of routines written in the ANSI C programming
language and organized in the form of a callable library. It is
intended for solving linear programming (LP), mixed integer programming
(MIP), and other related problems.

\section{LP problem}
\label{seclp}

GLPK assumes the following formulation of the {\it linear programming
(LP)} problem:

\noindent
\hspace{.5in} minimize (or maximize)
$$z = c_1x_{m+1} + c_2x_{m+2} + \dots + c_nx_{m+n} + c_0 \eqno (1.1)$$
\hspace{.5in} subject to linear constraints
$$
\begin{array}{r@{\:}c@{\:}r@{\:}c@{\:}r@{\:}c@{\:}r}
x_1&=&a_{11}x_{m+1}&+&a_{12}x_{m+2}&+ \dots +&a_{1n}x_{m+n} \\
x_2&=&a_{21}x_{m+1}&+&a_{22}x_{m+2}&+ \dots +&a_{2n}x_{m+n} \\
\multicolumn{7}{c}
{.\ \ .\ \ .\ \ .\ \ .\ \ .\ \ .\ \ .\ \ .\ \ .\ \ .\ \ .\ \ .\ \ .} \\
x_m&=&a_{m1}x_{m+1}&+&a_{m2}x_{m+2}&+ \dots +&a_{mn}x_{m+n} \\
\end{array} \eqno (1.2)
$$
\hspace{.5in} and bounds of variables
$$
\begin{array}{r@{\:}c@{\:}c@{\:}c@{\:}l}
l_1&\leq&x_1&\leq&u_1 \\
l_2&\leq&x_2&\leq&u_2 \\
\multicolumn{5}{c}{.\ \ .\ \ .\ \ .\ \ .}\\
l_{m+n}&\leq&x_{m+n}&\leq&u_{m+n} \\
\end{array} \eqno (1.3)
$$
where: $x_1, x_2, \dots, x_m$ are auxiliary variables;
$x_{m+1}, x_{m+2}, \dots, x_{m+n}$ are structural variables;
$z$ is the objective function;
$c_1, c_2, \dots, c_n$ are objective coefficients;
$c_0$ is the constant term (``shift'') of the objective function;
$a_{11}, a_{12}, \dots, a_{mn}$ are constraint coefficients;
$l_1, l_2, \dots, l_{m+n}$ are lower bounds of variables;
$u_1, u_2, \dots, u_{m+n}$ are upper bounds of variables.

Auxiliary variables are also called {\it rows}, because they correspond
to rows of the constraint matrix (i.e. a matrix built of the constraint
coefficients). Similarly, structural variables are also called
{\it columns}, because they correspond to columns of the constraint
matrix.

Bounds of variables can be finite as well as infinite. Besides, lower
and upper bounds can be equal to each other. Thus, the following types
of variables are possible:

\begin{center}
\begin{tabular}{r@{}c@{}ll}
\multicolumn{3}{c}{Bounds of variable} & Type of variable \\
\hline
$-\infty <$ &$\ x_k\ $& $< +\infty$ & Free (unbounded) variable \\
$l_k \leq$ &$\ x_k\ $& $< +\infty$  & Variable with lower bound \\
$-\infty <$ &$\ x_k\ $& $\leq u_k$  & Variable with upper bound \\
$l_k \leq$ &$\ x_k\ $& $\leq u_k$   & Double-bounded variable \\
$l_k =$ &$\ x_k\ $& $= u_k$         & Fixed variable \\
\end{tabular}
\end{center}

\noindent
Note that the types of variables shown above are applicable to
structural as well as to auxiliary variables.

To solve the LP problem (1.1)---(1.3) is to find such values of all
structural and auxiliary variables, which:

%\vspace*{-10pt}

%\begin{itemize}\setlength{\itemsep}{0pt}
\Item{---}satisfy to all the linear constraints (1.2), and

\Item{---}are within their bounds (1.3), and

\Item{---}provide smallest (in case of minimization) or largest (in
case of maximization) value of the objective function (1.1).
%\end{itemize}

\section{MIP problem}

{\it Mixed integer linear programming (MIP)} problem is an LP problem
in which some variables are additionally required to be integer.

GLPK assumes that MIP problem has the same formulation as ordinary
(pure) LP problem (1.1)---(1.3), i.e. includes auxiliary and structural
variables, which may have lower and/or upper bounds. However, in case
of MIP problem some variables may be required to be integer. This
additional constraint means that a value of each {\it integer variable}
must be only integer number. (Should note that GLPK allows only
structural variables to be of integer kind.)

\section{Using the package}

\subsection{Brief example}

In order to understand what GLPK is from the user's standpoint,
consider the following simple LP problem:

\noindent
\hspace{.5in} maximize
$$z = 10 x_1 + 6 x_2 + 4 x_3$$
\hspace{.5in} subject to
$$
\begin{array}{r@{\:}c@{\:}r@{\:}c@{\:}r@{\:}c@{\:}r}
x_1 &+&x_2 &+&x_3 &\leq 100 \\
10 x_1 &+& 4 x_2 & +&5 x_3 & \leq 600 \\
2 x_1 &+& 2 x_2 & +& 6 x_3 & \leq 300 \\
\end{array}
$$
\hspace{.5in} where all variables are non-negative
$$x_1 \geq 0, \ x_2 \geq 0, \ x_3 \geq 0$$

At first, this LP problem should be transformed to the standard form
(1.1)---(1.3). This can be easily done by introducing auxiliary
variables, by one for each original inequality constraint. Thus, the
problem can be reformulated as follows:

\noindent
\hspace{.5in} maximize
$$z = 10 x_1 + 6 x_2 + 4 x_3$$
\hspace{.5in} subject to
$$
\begin{array}{r@{\:}c@{\:}r@{\:}c@{\:}r@{\:}c@{\:}r}
p& = &x_1 &+&x_2 &+&x_3 \\
q& = &10 x_1 &+& 4 x_2 &+& 5 x_3 \\
r& = &2  x_1 &+& 2 x_2 &+& 6 x_3 \\
\end{array}
$$
\hspace{.5in} and bounds of variables
$$
\begin{array}{ccc}
\nonumber -\infty < p \leq 100 && 0 \leq x_1 < +\infty \\
\nonumber -\infty < q \leq 600 && 0 \leq x_2 < +\infty \\
\nonumber -\infty < r \leq 300 && 0 \leq x_3 < +\infty \\
\end{array}
$$
where $p, q, r$ are auxiliary variables (rows), and $x_1, x_2, x_3$ are
structural variables (columns).

The example C program shown below uses GLPK API routines in order to
solve this LP problem.\footnote{If you just need to solve LP or MIP
instance, you may write it in MPS or CPLEX LP format and then use the
GLPK stand-alone solver to obtain a solution. This is much less
time-consuming than programming in C with GLPK API routines.}

\begin{footnotesize}
\begin{verbatim}
/* sample.c */

#include <stdio.h>
#include <stdlib.h>
#include <glpk.h>

int main(void)
{     glp_prob *lp;
      int ia[1+1000], ja[1+1000];
      double ar[1+1000], z, x1, x2, x3;
s1:   lp = glp_create_prob();
s2:   glp_set_prob_name(lp, "sample");
s3:   glp_set_obj_dir(lp, GLP_MAX);
s4:   glp_add_rows(lp, 3);
s5:   glp_set_row_name(lp, 1, "p");
s6:   glp_set_row_bnds(lp, 1, GLP_UP, 0.0, 100.0);
s7:   glp_set_row_name(lp, 2, "q");
s8:   glp_set_row_bnds(lp, 2, GLP_UP, 0.0, 600.0);
s9:   glp_set_row_name(lp, 3, "r");
s10:  glp_set_row_bnds(lp, 3, GLP_UP, 0.0, 300.0);
s11:  glp_add_cols(lp, 3);
s12:  glp_set_col_name(lp, 1, "x1");
s13:  glp_set_col_bnds(lp, 1, GLP_LO, 0.0, 0.0);
s14:  glp_set_obj_coef(lp, 1, 10.0);
s15:  glp_set_col_name(lp, 2, "x2");
s16:  glp_set_col_bnds(lp, 2, GLP_LO, 0.0, 0.0);
s17:  glp_set_obj_coef(lp, 2, 6.0);
s18:  glp_set_col_name(lp, 3, "x3");
s19:  glp_set_col_bnds(lp, 3, GLP_LO, 0.0, 0.0);
s20:  glp_set_obj_coef(lp, 3, 4.0);
s21:  ia[1] = 1, ja[1] = 1, ar[1] =  1.0; /* a[1,1] =  1 */
s22:  ia[2] = 1, ja[2] = 2, ar[2] =  1.0; /* a[1,2] =  1 */
s23:  ia[3] = 1, ja[3] = 3, ar[3] =  1.0; /* a[1,3] =  1 */
s24:  ia[4] = 2, ja[4] = 1, ar[4] = 10.0; /* a[2,1] = 10 */
s25:  ia[5] = 3, ja[5] = 1, ar[5] =  2.0; /* a[3,1] =  2 */
s26:  ia[6] = 2, ja[6] = 2, ar[6] =  4.0; /* a[2,2] =  4 */
s27:  ia[7] = 3, ja[7] = 2, ar[7] =  2.0; /* a[3,2] =  2 */
s28:  ia[8] = 2, ja[8] = 3, ar[8] =  5.0; /* a[2,3] =  5 */
s29:  ia[9] = 3, ja[9] = 3, ar[9] =  6.0; /* a[3,3] =  6 */
s30:  glp_load_matrix(lp, 9, ia, ja, ar);
s31:  glp_simplex(lp, NULL);
s32:  z = glp_get_obj_val(lp);
s33:  x1 = glp_get_col_prim(lp, 1);
s34:  x2 = glp_get_col_prim(lp, 2);
s35:  x3 = glp_get_col_prim(lp, 3);
s36:  printf("\nz = %g; x1 = %g; x2 = %g; x3 = %g\n",
         z, x1, x2, x3);
s37:  glp_delete_prob(lp);
      return 0;
}

/* eof */
\end{verbatim}
\end{footnotesize}

The statement \verb|s1| creates a problem object. Being created the
object is initially empty. The statement \verb|s2| assigns a symbolic
name to the problem object.

The statement \verb|s3| calls the routine \verb|glp_set_obj_dir| in
order to set the optimization direction flag, where \verb|GLP_MAX|
means maximization.

The statement \verb|s4| adds three rows to the problem object.

The statement \verb|s5| assigns the symbolic name `\verb|p|' to the
first row, and the statement \verb|s6| sets the type and bounds of the
first row, where \verb|GLP_UP| means that the row has an upper bound.
The statements \verb|s7|, \verb|s8|, \verb|s9|, \verb|s10| are used in
the same way in order to assign the symbolic names `\verb|q|' and
`\verb|r|' to the second and third rows and set their types and bounds.

The statement \verb|s11| adds three columns to the problem object.

The statement \verb|s12| assigns the symbolic name `\verb|x1|' to the
first column, the statement \verb|s13| sets the type and bounds of the
first column, where \verb|GLP_LO| means that the column has an lower
bound, and the statement \verb|s14| sets the objective coefficient for
the first column. The statements \verb|s15|---\verb|s20| are used in
the same way in order to assign the symbolic names `\verb|x2|' and
`\verb|x3|' to the second and third columns and set their types,
bounds, and objective coefficients.

The statements \verb|s21|---\verb|s29| prepare non-zero elements of the
constraint matrix (i.e. constraint coefficients). Row indices of each
element are stored in the array \verb|ia|, column indices are stored in
the array \verb|ja|, and numerical values of corresponding elements are
stored in the array \verb|ar|. Then the statement \verb|s30| calls
the routine \verb|glp_load_matrix|, which loads information from these
three arrays into the problem object.

Now all data have been entered into the problem object, and therefore
the statement \verb|s31| calls the routine \verb|glp_simplex|, which is
a driver to the simplex method, in order to solve the LP problem. This
routine finds an optimal solution and stores all relevant information
back into the problem object.

The statement \verb|s32| obtains a computed value of the objective
function, and the statements \verb|s33|---\verb|s35| obtain computed
values of structural variables (columns), which correspond to the
optimal basic solution found by the solver.

The statement \verb|s36| writes the optimal solution to the standard
output. The printout may look like follows:

\newpage

\begin{footnotesize}
\begin{verbatim}
*     0:   objval =   0.000000000e+00   infeas =   0.000000000e+00 (0)
*     2:   objval =   7.333333333e+02   infeas =   0.000000000e+00 (0)
OPTIMAL SOLUTION FOUND

z = 733.333; x1 = 33.3333; x2 = 66.6667; x3 = 0
\end{verbatim}
\end{footnotesize}

Finally, the statement \verb|s37| calls the routine
\verb|glp_delete_prob|, which frees all the memory allocated to the
problem object.

\subsection{Compiling}

The GLPK package has the only header file \verb|glpk.h|, which should
be available on compiling a C (or C++) program using GLPK API routines.

If the header file is installed in the default location
\verb|/usr/local/include|, the following typical command may be used to
compile, say, the example C program described above with the GNU C
compiler:

\begin{verbatim}
   $ gcc -c sample.c
\end{verbatim}

If \verb|glpk.h| is not in the default location, the corresponding
directory containing it should be made known to the C compiler through
\verb|-I| option, for example:

\begin{verbatim}
   $ gcc -I/foo/bar/glpk-4.15/include -c sample.c
\end{verbatim}

In any case the compilation results in an object file \verb|sample.o|.

\subsection{Linking}

The GLPK library is a single file \verb|libglpk.a|. (On systems which
support shared libraries there may be also a shared version of the
library \verb|libglpk.so|.)

If the library is installed in the default
location \verb|/usr/local/lib|, the following typical command may be
used to link, say, the example C program described above against with
the library:

\begin{verbatim}
   $ gcc sample.o -lglpk -lm
\end{verbatim}

If the GLPK library is not in the default location, the corresponding
directory containing it should be made known to the linker through
\verb|-L| option, for example:

\begin{verbatim}
   $ gcc -L/foo/bar/glpk-4.15 sample.o -lglpk -lm
\end{verbatim}

Depending on configuration of the package linking against with the GLPK
library may require optional libraries, in which case these libraries
should be also made known to the linker, for example:

\begin{verbatim}
   $ gcc sample.o -lglpk -lgmp -lm
\end{verbatim}

For more details about configuration options of the GLPK package see
Appendix \ref{install}, page \pageref{install}.

%* eof *%


%* glpk02.tex *%

\chapter{Basic API Routines}

\section{General conventions}

\subsection{Library header}

All GLPK API data types and routines are defined in the header file
\verb|glpk.h|. It should be included in all source files which use
GLPK API, either directly or indirectly through some other header file
as follows:

\begin{verbatim}
   #include <glpk.h>
\end{verbatim}

\subsection{Error handling}

If some GLPK API routine detects erroneous or incorrect data passed by
the application program, it writes appropriate diagnostic messages to
the terminal and then abnormally terminates the application program.
In most practical cases this allows to simplify programming by avoiding
numerous checks of return codes. Thus, in order to prevent crashing the
application program should check all data, which are suspected to be
incorrect, before calling GLPK API routines.

Should note that this kind of error handling is used only in cases of
incorrect data passed by the application program. If, for example, the
application program calls some GLPK API routine to read data from an
input file and these data are incorrect, the GLPK API routine reports
about error in the usual way by means of the return code.

\subsection{Thread safety}

The standard version of GLPK API is {\it not} thread safe and therefore
should not be used in multi-threaded programs.

\subsection{Array indexing}

Normally all GLPK API routines start array indexing from 1, not from 0
(except the specially stipulated cases). This means, for example, that
if some vector $x$ of the length $n$ is passed as an array to some GLPK
API routine, the latter expects vector components to be placed in
locations \verb|x[1]|, \verb|x[2]|, \dots, \verb|x[n]|, and the
location \verb|x[0]| normally is not used.

To avoid indexing errors it is most convenient and most reliable to
declare the array \verb|x| as follows:

\begin{verbatim}
   double x[1+n];
\end{verbatim}

\noindent
or to allocate it as follows:

\begin{verbatim}
   double *x;
   . . .
   x = calloc(1+n, sizeof(double));
   . . .
\end{verbatim}

\noindent
In both cases one extra location \verb|x[0]| is reserved that allows
passing the array to GLPK routines in a usual way.

\section{Problem object}

All GLPK API routines deal with so called {\it problem object}, which
is a program object of type \verb|glp_prob| and intended to represent
a particular LP or MIP instance.

The type \verb|glp_prob| is a data structure declared in the header
file \verb|glpk.h| as follows:

\begin{verbatim}
   typedef struct glp_prob glp_prob;
\end{verbatim}

Problem objects (i.e. program objects of the \verb|glp_prob| type) are
allocated and managed internally by the GLPK API routines. The
application program should never use any members of the \verb|glp_prob|
structure directly and should deal only with pointers to these objects
(that is, \verb|glp_prob *| values).

The problem object consists of the following segments:

%\vspace*{-8pt}

%\begin{itemize}\setlength{\itemsep}{0pt}
\Item{---}problem segment,

\Item{---}basis segment,

\Item{---}interior-point segment, and

\Item{---}MIP segment.
%\end{itemize}

\subsection{Problem segment}

The {\it problem segment} contains original LP/MIP data, which
corresponds to the problem formulation (1.1)---(1.3) (see Section
\ref{seclp}, page \pageref{seclp}). This segment includes the following
components:

%\vspace*{-8pt}

%\begin{itemize}\setlength{\itemsep}{0pt}
\Item{---}rows (auxiliary variables),

\Item{---}columns (structural variables),

\Item{---}objective function, and

\Item{---}constraint matrix.
%\end{itemize}

%\vspace*{-7pt}

Rows and columns have the same set of the following attributes:

%\vspace*{-7pt}

%\begin{itemize}\setlength{\itemsep}{0pt}
\Item{---}ordinal number,

\Item{---}symbolic name (1 up to 255 arbitrary graphic characters),

\Item{---}type (free, lower bound, upper bound, double bound, fixed),

\Item{---}numerical values of lower and upper bounds,

\Item{---}scale factor.
%\end{itemize}

%\vspace*{-7pt}

{\it Ordinal numbers} are intended for referencing rows and columns.
Row ordinal numbers are integers $1, 2, \dots, m$, and column ordinal
numbers are integers $1, 2, \dots, n$, where $m$ and $n$ are,
respectively, the current number of rows and columns in the problem
object.

{\it Symbolic names} are intended for informational purposes. They also
can be used for referencing rows and columns.

{\it Types and bounds} of rows (auxiliary variables) and columns
(structural variables) are explained above (see Section \ref{seclp},
page \pageref{seclp}).

{\it Scale factors} are used internally for scaling rows and columns of
the constraint matrix.

Information about the {\it objective function} includes numerical
values of objective coefficients and a flag, which defines the
optimization direction (i.e. minimization or maximization).

The {\it constraint matrix} is a $m \times n$ rectangular matrix built
of constraint coefficients $a_{ij}$, which defines the system of linear
constraints (1.2) (see Section \ref{seclp}, page \pageref{seclp}). This
matrix is stored in the problem object in both row-wise and column-wise
sparse formats.

Once the problem object has been created, the application program can
access and modify any components of the problem segment in arbitrary
order.

\subsection{Basis segment}

The {\it basis segment} of the problem object keeps information related
to the current basic solution. It includes:

%\vspace*{-8pt}

%\begin{itemize}\setlength{\itemsep}{0pt}
\Item{---}row and column statuses,

\Item{---}basic solution statuses,

\Item{---}factorization of the current basis matrix, and

\Item{---}basic solution components.
%\end{itemize}

%\vspace*{-8pt}

The {\it row and column statuses} define which rows and columns are
basic and which are non-basic. These statuses may be assigned either by
the application program of by some API routines. Note that these
statuses are always defined independently on whether the corresponding
basis is valid or not.

The {\it basic solution statuses} include the {\it primal status} and
the {\it dual status}, which are set by the simplex-based solver once
the problem has been solved. The primal status shows whether a primal
basic solution is feasible, infeasible, or undefined. The dual status
shows the same for a dual basic solution.

The {\it factorization of the basis matrix} is some factorized form
(like {\it LU}-factorization) of the current basis matrix (defined by
the current row and column statuses). The factorization is used by
simplex-based solvers and kept when the solver terminates the search.
This feature allows efficiently reoptimizing the problem after some
modifications (for example, after changing some bounds or objective
coefficients). It also allows performing the post-optimal analysis (for
example, computing components of the simplex table, etc.).

The {\it basic solution components} include primal and dual values of
all auxiliary and structural variables for the most recently obtained
basic solution.

\subsection{Interior-point segment}

The {\it interior-point segment} contains interior-point solution
components, which include the solution status, and primal and dual
values of all auxiliary and structural variables.

\subsection{MIP segment}

The {\it MIP segment} is used only for MIP problems. This segment
includes:

%\vspace*{-8pt}

%\begin{itemize}\setlength{\itemsep}{0pt}
\Item{---}column kinds,

\Item{---}MIP solution status, and

\Item{---}MIP solution components.
%\end{itemize}

%\vspace*{-8pt}

The {\it column kinds} define which columns (i.e. structural variables)
are integer and which are continuous.

The {\it MIP solution status} is set by the MIP solver and shows whether
a MIP solution is integer optimal, integer feasible (non-optimal), or
undefined.

The {\it MIP solution components} are computed by the MIP solver and
include primal values of all auxiliary and structural variables for the
most recently obtained MIP solution.

Note that in case of MIP problem the basis segment corresponds to
the optimal solution of LP relaxation, which is also available to the
application program.

Currently the search tree is not kept in the MIP segment, so if the
search has been terminated, it cannot be continued.

%%%%%%%%%%%%%%%%%%%%%%%%%%%%%%%%%%%%%%%%%%%%%%%%%%%%%%%%%%%%%%%%%%%%%%%%

\newpage

\section{Problem creating and modifying routines}

\subsection{glp\_create\_prob --- create problem object}

\synopsis

\begin{verbatim}
   glp_prob *glp_create_prob(void);
\end{verbatim}

\description

The routine \verb|glp_create_prob| creates a new problem object, which
initially is ``empty'', i.e. has no rows and columns.

\returns

The routine returns a pointer to the created object, which should be
used in any subsequent operations on this object.

\subsection{glp\_set\_prob\_name --- assign (change) problem name}

\synopsis

\begin{verbatim}
   void glp_set_prob_name(glp_prob *P, const char *name);
\end{verbatim}

\description

The routine \verb|glp_set_prob_name| assigns a given symbolic
\verb|name| (1 up to 255 characters) to the specified problem object.

If the parameter \verb|name| is \verb|NULL| or empty string, the
routine erases an existing symbolic name of the problem object.

\subsection{glp\_set\_obj\_name --- assign (change) objective function
name}

\synopsis

\begin{verbatim}
   void glp_set_obj_name(glp_prob *P, const char *name);
\end{verbatim}

\description

The routine \verb|glp_set_obj_name| assigns a given symbolic
\verb|name| (1 up to 255 characters) to the objective function of the
specified problem object.

If the parameter \verb|name| is \verb|NULL| or empty string, the
routine erases an existing symbolic name of the objective function.

\newpage

\subsection{glp\_set\_obj\_dir --- set (change) optimization direction
flag}

\synopsis

\begin{verbatim}
   void glp_set_obj_dir(glp_prob *P, int dir);
\end{verbatim}

\description

The routine \verb|glp_set_obj_dir| sets (changes) the optimization
direction flag (i.e. ``sense'' of the objective function) as specified
by the parameter \verb|dir|:

\verb|GLP_MIN| means minimization;

\verb|GLP_MAX| means maximization.

Note that by default the problem is minimization.

\subsection{glp\_add\_rows --- add new rows to problem object}

\synopsis

\begin{verbatim}
   int glp_add_rows(glp_prob *P, int nrs);
\end{verbatim}

\description

The routine \verb|glp_add_rows| adds \verb|nrs| rows (constraints) to
the specified problem object. New rows are always added to the end of
the row list, so the ordinal numbers of existing rows are not changed.

Being added each new row is initially free (unbounded) and has empty
list of the constraint coefficients.

Each new row becomes a non-active (non-binding) constraint, i.e. the
corresponding auxiliary variable is marked as basic.

If the basis factorization exists, adding row(s) invalidates it.

\returns

The routine \verb|glp_add_rows| returns the ordinal number of the first
new row added to the problem object.

\subsection{glp\_add\_cols --- add new columns to problem object}

\synopsis

\begin{verbatim}
   int glp_add_cols(glp_prob *P, int ncs);
\end{verbatim}

\description

The routine \verb|glp_add_cols| adds \verb|ncs| columns (structural
variables) to the specified problem object. New columns are always
added to the end of the column list, so the ordinal numbers of existing
columns are not changed.

Being added each new column is initially fixed at zero and has empty
list of the constraint coefficients.

Each new column is marked as non-basic, i.e. zero value of the
corresponding structural variable becomes an active (binding) bound.

If the basis factorization exists, it remains valid.

\returns

The routine \verb|glp_add_cols| returns the ordinal number of the first
new column added to the problem object.

\subsection{glp\_set\_row\_name --- assign (change) row name}

\synopsis

\begin{verbatim}
   void glp_set_row_name(glp_prob *P, int i, const char *name);
\end{verbatim}

\description

The routine \verb|glp_set_row_name| assigns a given symbolic
\verb|name| (1 up to 255 characters) to \verb|i|-th row (auxiliary
variable) of the specified problem object.

If the parameter \verb|name| is \verb|NULL| or empty string, the
routine erases an existing name of $i$-th row.

\subsection{glp\_set\_col\_name --- assign (change) column name}

\synopsis

\begin{verbatim}
   void glp_set_col_name(glp_prob *P, int j, const char *name);
\end{verbatim}

\description

The routine \verb|glp_set_col_name| assigns a given symbolic
\verb|name| (1 up to 255 characters) to \verb|j|-th column (structural
variable) of the specified problem object.

If the parameter \verb|name| is \verb|NULL| or empty string, the
routine erases an existing name of $j$-th column.

\subsection{glp\_set\_row\_bnds --- set (change) row bounds}

\synopsis

{\tt void glp\_set\_row\_bnds(glp\_prob *P, int i, int type,
double lb, double ub);}

\description

The routine \verb|glp_set_row_bnds| sets (changes) the type and bounds
of \verb|i|-th row (auxiliary variable) of the specified problem
object.

The parameters \verb|type|, \verb|lb|, and \verb|ub| specify the type,
lower bound, and upper bound, respectively, as follows:

\begin{center}
\begin{tabular}{cr@{}c@{}ll}
Type & \multicolumn{3}{c}{Bounds} & Comment \\
\hline
\verb|GLP_FR| & $-\infty <$ &$\ x\ $& $< +\infty$
   & Free (unbounded) variable \\
\verb|GLP_LO| & $lb \leq$ &$\ x\ $& $< +\infty$
   & Variable with lower bound \\
\verb|GLP_UP| & $-\infty <$ &$\ x\ $& $\leq ub$
   & Variable with upper bound \\
\verb|GLP_DB| & $lb \leq$ &$\ x\ $& $\leq ub$
   & Double-bounded variable \\
\verb|GLP_FX| & $lb =$ &$\ x\ $& $= ub$
   & Fixed variable \\
\end{tabular}
\end{center}

\noindent
where $x$ is the auxiliary variable associated with $i$-th row.

If the row has no lower bound, the parameter \verb|lb| is ignored. If
the row has no upper bound, the parameter \verb|ub| is ignored. If the
row is an equality constraint (i.e. the corresponding auxiliary
variable is of fixed type), only the parameter \verb|lb| is used while
the parameter \verb|ub| is ignored.

Being added to the problem object each row is initially free, i.e. its
type is \verb|GLP_FR|.

\subsection{glp\_set\_col\_bnds --- set (change) column bounds}

\synopsis

{\tt void glp\_set\_col\_bnds(glp\_prob *P, int j, int type,
double lb, double ub);}

\description

The routine \verb|glp_set_col_bnds| sets (changes) the type and bounds
of \verb|j|-th column (structural variable) of the specified problem
object.

The parameters \verb|type|, \verb|lb|, and \verb|ub| specify the type,
lower bound, and upper bound, respectively, as follows:

\begin{center}
\begin{tabular}{cr@{}c@{}ll}
Type & \multicolumn{3}{c}{Bounds} & Comment \\
\hline
\verb|GLP_FR| & $-\infty <$ &$\ x\ $& $< +\infty$
   & Free (unbounded) variable \\
\verb|GLP_LO| & $lb \leq$ &$\ x\ $& $< +\infty$
   & Variable with lower bound \\
\verb|GLP_UP| & $-\infty <$ &$\ x\ $& $\leq ub$
   & Variable with upper bound \\
\verb|GLP_DB| & $lb \leq$ &$\ x\ $& $\leq ub$
   & Double-bounded variable \\
\verb|GLP_FX| & $lb =$ &$\ x\ $& $= ub$
   & Fixed variable \\
\end{tabular}
\end{center}

\noindent
where $x$ is the structural variable associated with $j$-th column.

If the column has no lower bound, the parameter \verb|lb| is ignored.
If the column has no upper bound, the parameter \verb|ub| is ignored.
If the column is of fixed type, only the parameter \verb|lb| is used
while the parameter \verb|ub| is ignored.

Being added to the problem object each column is initially fixed at
zero, i.e. its type is \verb|GLP_FX| and both bounds are 0.

%\newpage

\subsection{glp\_set\_obj\_coef --- set (change) objective coefficient
or constant term}

\synopsis

\begin{verbatim}
   void glp_set_obj_coef(glp_prob *P, int j, double coef);
\end{verbatim}

\description

The routine \verb|glp_set_obj_coef| sets (changes) the objective
coefficient at \verb|j|-th column (structural variable). A new value of
the objective coefficient is specified by the parameter \verb|coef|.

If the parameter \verb|j| is 0, the routine sets (changes) the constant
term (``shift'') of the objective function.

\newpage

\subsection{glp\_set\_mat\_row --- set (replace) row of the constraint
matrix}

\synopsis

\begin{verbatim}
   void glp_set_mat_row(glp_prob *P, int i, int len, const int ind[],
                        const double val[]);
\end{verbatim}

\description

The routine \verb|glp_set_mat_row| stores (replaces) the contents of
\verb|i|-th row of the constraint matrix of the specified problem
object.

Column indices and numerical values of new row elements should be
placed in locations\linebreak \verb|ind[1]|, \dots, \verb|ind[len]| and
\verb|val[1]|, \dots, \verb|val[len]|, respectively, where
$0 \leq$ \verb|len| $\leq n$ is the new length of $i$-th row, $n$ is
the current number of columns in the problem object. Elements with
identical column indices are not allowed. Zero elements are allowed,
but they are not stored in the constraint matrix.

If the parameter \verb|len| is 0, the parameters \verb|ind| and/or
\verb|val| can be specified as \verb|NULL|.

\note

If the basis factorization exists and changing the row changes
coefficients at basic column(s), the factorization is invalidated.

\subsection{glp\_set\_mat\_col --- set (replace) column of the
constr\-aint matrix}

\synopsis

\begin{verbatim}
   void glp_set_mat_col(glp_prob *P, int j, int len, const int ind[],
                        const double val[]);
\end{verbatim}

\description

The routine \verb|glp_set_mat_col| stores (replaces) the contents of
\verb|j|-th column of the constraint matrix of the specified problem
object.

Row indices and numerical values of new column elements should be
placed in locations\linebreak \verb|ind[1]|, \dots, \verb|ind[len]| and
\verb|val[1]|, \dots, \verb|val[len]|, respectively, where
$0 \leq$ \verb|len| $\leq m$ is the new length of $j$-th column, $m$ is
the current number of rows in the problem object. Elements with
identical row indices are not allowed. Zero elements are allowed, but
they are not stored in the constraint matrix.

If the parameter \verb|len| is 0, the parameters \verb|ind| and/or
\verb|val| can be specified as \verb|NULL|.

\note

If the basis factorization exists, changing the column corresponding
to a basic structural variable invalidates it.

\newpage

\subsection{glp\_load\_matrix --- load (replace) the whole constraint
matrix}

\synopsis

\begin{verbatim}
   void glp_load_matrix(glp_prob *P, int ne, const int ia[],
                        const int ja[], const double ar[]);
\end{verbatim}

\description

The routine \verb|glp_load_matrix| loads the constraint matrix passed
in  the arrays \verb|ia|, \verb|ja|, and \verb|ar| into the specified
problem object. Before loading the current contents of the constraint
matrix is destroyed.

Constraint coefficients (elements of the constraint matrix) should be
specified as triplets (\verb|ia[k]|, \verb|ja[k]|, \verb|ar[k]|) for
$k=1,\dots,ne$, where \verb|ia[k]| is the row index, \verb|ja[k]| is
the column index, and \verb|ar[k]| is a numeric value of corresponding
constraint coefficient. The parameter \verb|ne| specifies the total
number of (non-zero) elements in the matrix to be loaded. Coefficients
with identical indices are not allowed. Zero coefficients are allowed,
however, they are not stored in the constraint matrix.

If the parameter \verb|ne| is 0, the parameters \verb|ia|, \verb|ja|,
and/or \verb|ar| can be specified as \verb|NULL|.

\note

If the basis factorization exists, this operation invalidates it.

\subsection{glp\_check\_dup --- check for duplicate elements in sparse
matrix}

\synopsis

{\tt int glp\_check\_dup(int m, int n, int ne, const int ia[],
const int ja[]);}

\description

The routine \verb|glp_check_dup checks| for duplicate elements (that
is, elements with identical indices) in a sparse matrix specified in
the coordinate format.

The parameters $m$ and $n$ specifies, respectively, the number of rows
and columns in the matrix, $m\geq 0$, $n\geq 0$.

The parameter {\it ne} specifies the number of (structurally) non-zero
elements in the matrix,\linebreak {\it ne} $\geq 0$.

Elements of the matrix are specified as doublets $(ia[k],ja[k])$ for
$k=1,\dots,ne$, where $ia[k]$ is a row index, $ja[k]$ is a column
index.

The routine \verb|glp_check_dup| can be used prior to a call to the
routine \verb|glp_load_matrix| to check that the constraint matrix to
be loaded has no duplicate elements.

\returns

\begin{retlist}
0&    the matrix representation is correct;\\
$-k$& indices $ia[k]$ or/and $ja[k]$ are out of range;\\
$+k$& element $(ia[k],ja[k])$ is duplicate.\\
\end{retlist}

\subsection{glp\_sort\_matrix --- sort elements of the constraint
matrix}

\synopsis

\begin{verbatim}
   void glp_sort_matrix(glp_prob *P);
\end{verbatim}

\description

The routine \verb|glp_sort_matrix| sorts elements of the constraint
matrix by rebuilding its row and column linked lists.

On exit from the routine the constraint matrix is not changed, however,
elements in the row linked lists become ordered by ascending column
indices, and the elements in the column linked lists become ordered by
ascending row indices.

\subsection{glp\_del\_rows --- delete rows from problem object}

\synopsis

\begin{verbatim}
   void glp_del_rows(glp_prob *P, int nrs, const int num[]);
\end{verbatim}

\description

The routine \verb|glp_del_rows| deletes rows from the specified problem
object. Ordinal numbers of rows to be deleted should be placed in
locations \verb|num[1]|, \dots, \verb|num[nrs]|, where ${\tt nrs}>0$.

Note that deleting rows involves changing ordinal numbers of other
rows remaining in the problem object. New ordinal numbers of the
remaining rows are assigned under the assumption that the original
order of rows is not changed. Let, for example, before deletion there
be five rows $a$, $b$, $c$, $d$, $e$ with ordinal numbers 1, 2, 3, 4,
5, and let rows $b$ and $d$ have been deleted. Then after deletion the
remaining rows $a$, $c$, $e$ are assigned new oridinal numbers 1, 2, 3.

If the basis factorization exists, deleting active (binding) rows,
i.e. whose auxiliary variables are marked as non-basic, invalidates it.

%\newpage

\subsection{glp\_del\_cols --- delete columns from problem object}

\synopsis

\begin{verbatim}
   void glp_del_cols(glp_prob *P, int ncs, const int num[]);
\end{verbatim}

\description

The routine \verb|glp_del_cols| deletes columns from the specified
problem object. Ordinal numbers of columns to be deleted should be
placed in locations \verb|num[1]|, \dots, \verb|num[ncs]|, where
${\tt ncs}>0$.

Note that deleting columns involves changing ordinal numbers of other
columns remaining in\linebreak the problem object. New ordinal numbers
of the remaining columns are assigned under the assumption that the
original order of columns is not changed. Let, for example, before
deletion  there be six columns $p$, $q$, $r$, $s$, $t$, $u$ with
ordinal numbers 1, 2, 3, 4, 5, 6, and let columns $p$, $q$, $s$ have
been deleted. Then after deletion the remaining columns $r$, $t$, $u$
are assigned new ordinal numbers 1, 2, 3.

If the basis factorization exists, deleting basic columns invalidates
it.

\subsection{glp\_copy\_prob --- copy problem object content}

\synopsis

\begin{verbatim}
   void glp_copy_prob(glp_prob *dest, glp_prob *prob, int names);
\end{verbatim}

\description

The routine \verb|glp_copy_prob| copies the content of the problem
object \verb|prob| to the problem object \verb|dest|.

The parameter \verb|names| is a flag. If it is \verb|GLP_ON|,
the routine also copies all symbolic names; otherwise, if it is
\verb|GLP_OFF|, no symbolic names are copied.

\subsection{glp\_erase\_prob --- erase problem object content}

\synopsis

\begin{verbatim}
   void glp_erase_prob(glp_prob *P);
\end{verbatim}

\description

The routine \verb|glp_erase_prob| erases the content of the specified
problem object. The effect of this operation is the same as if the
problem object would be deleted with the routine \verb|glp_delete_prob|
and then created anew with the routine \verb|glp_create_prob|, with the
only exception that the pointer to the problem object remains valid.

%\newpage

\subsection{glp\_delete\_prob --- delete problem object}

\synopsis

\begin{verbatim}
   void glp_delete_prob(glp_prob *P);
\end{verbatim}

\description

The routine \verb|glp_delete_prob| deletes a problem object, which the
parameter \verb|lp| points to, freeing all the memory allocated to this
object.

%%%%%%%%%%%%%%%%%%%%%%%%%%%%%%%%%%%%%%%%%%%%%%%%%%%%%%%%%%%%%%%%%%%%%%%%

\newpage

\section{Problem retrieving routines}

\subsection{glp\_get\_prob\_name --- retrieve problem name}

\synopsis

\begin{verbatim}
   const char *glp_get_prob_name(glp_prob *P);
\end{verbatim}

\returns

The routine \verb|glp_get_prob_name| returns a pointer to an internal
buffer, which contains symbolic name of the problem. However, if the
problem has no assigned name, the routine returns \verb|NULL|.

\subsection{glp\_get\_obj\_name --- retrieve objective function name}

\synopsis

\begin{verbatim}
   const char *glp_get_obj_name(glp_prob *P);
\end{verbatim}

\returns

The routine \verb|glp_get_obj_name| returns a pointer to an internal
buffer, which contains symbolic name assigned to the objective
function. However, if the objective function has no assigned name, the
routine returns \verb|NULL|.

\subsection{glp\_get\_obj\_dir --- retrieve optimization direction
flag}

\synopsis

\begin{verbatim}
   int glp_get_obj_dir(glp_prob *P);
\end{verbatim}

\returns

The routine \verb|glp_get_obj_dir| returns the optimization direction
flag (i.e. ``sense'' of the objective function):

\verb|GLP_MIN| means minimization;

\verb|GLP_MAX| means maximization.

\subsection{glp\_get\_num\_rows --- retrieve number of rows}

\synopsis

\begin{verbatim}
   int glp_get_num_rows(glp_prob *P);
\end{verbatim}

\returns

The routine \verb|glp_get_num_rows| returns the current number of rows
in the specified problem object.

\newpage

\subsection{glp\_get\_num\_cols --- retrieve number of columns}

\synopsis

\begin{verbatim}
   int glp_get_num_cols(glp_prob *P);
\end{verbatim}

\returns

The routine \verb|glp_get_num_cols| returns the current number of
columns in the specified problem object.

\subsection{glp\_get\_row\_name --- retrieve row name}

\synopsis

\begin{verbatim}
   const char *glp_get_row_name(glp_prob *P, int i);
\end{verbatim}

\returns

The routine \verb|glp_get_row_name| returns a pointer to an internal
buffer, which contains a symbolic name assigned to \verb|i|-th row.
However, if the row has no assigned name, the routine returns
\verb|NULL|.

\subsection{glp\_get\_col\_name --- retrieve column name}

\synopsis

\begin{verbatim}
   const char *glp_get_col_name(glp_prob *P, int j);
\end{verbatim}

\returns

The routine \verb|glp_get_col_name| returns a pointer to an internal
buffer, which contains a symbolic name assigned to \verb|j|-th column.
However, if the column has no assigned name, the routine returns
\verb|NULL|.

\subsection{glp\_get\_row\_type --- retrieve row type}

\synopsis

\begin{verbatim}
   int glp_get_row_type(glp_prob *P, int i);
\end{verbatim}

\returns

The routine \verb|glp_get_row_type| returns the type of \verb|i|-th
row, i.e. the type of corresponding auxiliary variable, as follows:

\verb|GLP_FR| --- free (unbounded) variable;

\verb|GLP_LO| --- variable with lower bound;

\verb|GLP_UP| --- variable with upper bound;

\verb|GLP_DB| --- double-bounded variable;

\verb|GLP_FX| --- fixed variable.

\subsection{glp\_get\_row\_lb --- retrieve row lower bound}

\synopsis

\begin{verbatim}
   double glp_get_row_lb(glp_prob *P, int i);
\end{verbatim}

\returns

The routine \verb|glp_get_row_lb| returns the lower bound of
\verb|i|-th row, i.e. the lower bound of corresponding auxiliary
variable. However, if the row has no lower bound, the routine returns
\verb|-DBL_MAX|.

\vspace*{-4pt}

\subsection{glp\_get\_row\_ub --- retrieve row upper bound}

\synopsis

\begin{verbatim}
   double glp_get_row_ub(glp_prob *P, int i);
\end{verbatim}

\returns

The routine \verb|glp_get_row_ub| returns the upper bound of
\verb|i|-th row, i.e. the upper bound of corresponding auxiliary
variable. However, if the row has no upper bound, the routine returns
\verb|+DBL_MAX|.

\vspace*{-4pt}

\subsection{glp\_get\_col\_type --- retrieve column type}

\synopsis

\begin{verbatim}
   int glp_get_col_type(glp_prob *P, int j);
\end{verbatim}

\returns

The routine \verb|glp_get_col_type| returns the type of \verb|j|-th
column, i.e. the type of corresponding structural variable, as follows:

\verb|GLP_FR| --- free (unbounded) variable;

\verb|GLP_LO| --- variable with lower bound;

\verb|GLP_UP| --- variable with upper bound;

\verb|GLP_DB| --- double-bounded variable;

\verb|GLP_FX| --- fixed variable.

\vspace*{-4pt}

\subsection{glp\_get\_col\_lb --- retrieve column lower bound}

\synopsis

\begin{verbatim}
   double glp_get_col_lb(glp_prob *P, int j);
\end{verbatim}

\returns

The routine \verb|glp_get_col_lb| returns the lower bound of
\verb|j|-th column, i.e. the lower bound of corresponding structural
variable. However, if the column has no lower bound, the routine
returns \verb|-DBL_MAX|.

\subsection{glp\_get\_col\_ub --- retrieve column upper bound}

\synopsis

\begin{verbatim}
   double glp_get_col_ub(glp_prob *P, int j);
\end{verbatim}

\returns

The routine \verb|glp_get_col_ub| returns the upper bound of
\verb|j|-th column, i.e. the upper bound of corresponding structural
variable. However, if the column has no upper bound, the routine
returns \verb|+DBL_MAX|.

\subsection{glp\_get\_obj\_coef --- retrieve objective coefficient or
constant term}

\synopsis

\begin{verbatim}
   double glp_get_obj_coef(glp_prob *P, int j);
\end{verbatim}

\returns

The routine \verb|glp_get_obj_coef| returns the objective coefficient
at \verb|j|-th structural variable (column).

If the parameter \verb|j| is 0, the routine returns the constant term
(``shift'') of the objective function.

\subsection{glp\_get\_num\_nz --- retrieve number of constraint
coefficients}

\synopsis

\begin{verbatim}
   int glp_get_num_nz(glp_prob *P);
\end{verbatim}

\returns

The routine \verb|glp_get_num_nz| returns the number of non-zero
elements in the constraint matrix of the specified problem object.

\subsection{glp\_get\_mat\_row --- retrieve row of the constraint
matrix}

\synopsis

\begin{verbatim}
   int glp_get_mat_row(glp_prob *P, int i, int ind[], double val[]);
\end{verbatim}

\description

The routine \verb|glp_get_mat_row| scans (non-zero) elements of
\verb|i|-th row of the constraint matrix of the specified problem
object and stores their column indices and numeric values to locations
\verb|ind[1]|, \dots, \verb|ind[len]| and \verb|val[1]|, \dots,
\verb|val[len]|, respectively, where $0\leq{\tt len}\leq n$ is the
number of elements in $i$-th row, $n$ is the number of columns.

The parameter \verb|ind| and/or \verb|val| can be specified as
\verb|NULL|, in which case corresponding information is not stored.

%\newpage

\returns

The routine \verb|glp_get_mat_row| returns the length \verb|len|, i.e.
the number of (non-zero) elements in \verb|i|-th row.

\subsection{glp\_get\_mat\_col --- retrieve column of the constraint
matrix}

\synopsis

\begin{verbatim}
   int glp_get_mat_col(glp_prob *P, int j, int ind[], double val[]);
\end{verbatim}

\description

The routine \verb|glp_get_mat_col| scans (non-zero) elements of
\verb|j|-th column of the constraint matrix of the specified problem
object and stores their row indices and numeric values to locations
\linebreak \verb|ind[1]|, \dots, \verb|ind[len]| and \verb|val[1]|,
\dots, \verb|val[len]|, respectively, where $0\leq{\tt len}\leq m$ is
the number of elements in $j$-th column, $m$ is the number of rows.

The parameter \verb|ind| and/or \verb|val| can be specified as
\verb|NULL|, in which case corresponding information is not stored.

\returns

The routine \verb|glp_get_mat_col| returns the length \verb|len|, i.e.
the number of (non-zero) elements in \verb|j|-th column.

%%%%%%%%%%%%%%%%%%%%%%%%%%%%%%%%%%%%%%%%%%%%%%%%%%%%%%%%%%%%%%%%%%%%%%%%

\newpage

\section{Row and column searching routines}

Sometimes it may be necessary to find rows and/or columns by their
names (assigned with the routines \verb|glp_set_row_name| and
\verb|glp_set_col_name|). Though a particular row/column can be found
by its name using simple enumeration of all rows/columns, in case of
large instances such a {\it linear} search may take too long time.

To significantly reduce the search time the application program may
create the row/column name index, which is an auxiliary data structure
implementing a {\it binary} search. Even in worst cases the search
takes logarithmic time, i.e. the time needed to find a particular row
(or column) by its name is $O(\log_2m)$ (or $O(\log_2n)$), where $m$
and $n$ are, resp., the number of rows and columns in the problem
object.

It is important to note that:

\Item{1.}On creating the problem object with the routine
\verb|glp_create_prob| the name index is {\it not} created.

\Item{2.}The name index can be created (destroyed) at any time with the
routine \verb|glp_create_index| (\verb|glp_delete_index|). Having been
created the name index becomes part of the corresponding problem
object.

\Item{3.}The time taken to create the name index is
$O[(m+n)\log_2(m+n)]$, so it is recommended to create the index only
once, for example, just after the problem object was created.

\Item{4.}If the name index exists, it is automatically updated every
time the name of a row/column is assigned/changed. The update operation
takes logarithmic time.

\Item{5.}If the name index does not exist, the application should not
call the routines \verb|glp_find_row| and \verb|glp_find_col|.
Otherwise, an error message will be issued and abnormal program
termination will occur.

\Item{6.}On destroying the problem object with the routine
\verb|glp_delete_prob|, the name index, if exists, is automatically
destroyed.

\subsection{glp\_create\_index --- create the name index}

\synopsis

\begin{verbatim}
   void glp_create_index(glp_prob *P);
\end{verbatim}

\description

The routine \verb|glp_create_index| creates the name index for the
specified problem object. The name index is an auxiliary data
structure, which is intended to quickly (i.e. for logarithmic time)
find rows and columns by their names.

This routine can be called at any time. If the name index already
exists, the routine does nothing.

\newpage

\subsection{glp\_find\_row --- find row by its name}

\synopsis

\begin{verbatim}
   int glp_find_row(glp_prob *P, const char *name);
\end{verbatim}

\returns

The routine \verb|glp_find_row| returns the ordinal number of a row,
which is assigned the specified symbolic \verb|name|. If no such row
exists, the routine returns 0.

\subsection{glp\_find\_col --- find column by its name}

\synopsis

\begin{verbatim}
   int glp_find_col(glp_prob *P, const char *name);
\end{verbatim}

\returns

The routine \verb|glp_find_col| returns the ordinal number of a column,
which is assigned the specified symbolic \verb|name|. If no such column
exists, the routine returns 0.

\subsection{glp\_delete\_index --- delete the name index}

\synopsis

\begin{verbatim}
   void glp_delete_index(glp_prob *P);
\end{verbatim}

\description

The routine \verb|glp_delete_index| deletes the name index previously
created by the routine\linebreak \verb|glp_create_index| and frees the
memory allocated to this auxiliary data structure.

This routine can be called at any time. If the name index does not
exist, the routine does nothing.

%%%%%%%%%%%%%%%%%%%%%%%%%%%%%%%%%%%%%%%%%%%%%%%%%%%%%%%%%%%%%%%%%%%%%%%%

\newpage

\section{Problem scaling routines}

\subsection{Background}

In GLPK the {\it scaling} means a linear transformation applied to the
constraint matrix to improve its numerical properties.\footnote{In many
cases a proper scaling allows making the constraint matrix to be better
conditioned, i.e. decreasing its condition number, that makes
computations numerically more stable.}

The main equality is the following:
$$\widetilde{A}=RAS,\eqno(2.1)$$
where $A=(a_{ij})$ is the original constraint matrix, $R=(r_{ii})>0$ is
a diagonal matrix used to scale rows (constraints), $S=(s_{jj})>0$ is a
diagonal matrix used to scale columns (variables), $\widetilde{A}$ is
the scaled constraint matrix.

From (2.1) it follows that in the {\it scaled} problem instance each
original constraint coefficient $a_{ij}$ is replaced by corresponding
scaled constraint coefficient:
$$\widetilde{a}_{ij}=r_{ii}a_{ij}s_{jj}.\eqno(2.2)$$

Note that the scaling is performed internally and therefore
transparently to the user. This means that on API level the user always
deal with unscaled data.

Scale factors $r_{ii}$ and $s_{jj}$ can be set or changed at any time
either directly by the application program in a problem specific way
(with the routines \verb|glp_set_rii| and \verb|glp_set_sjj|), or by
some API routines intended for automatic scaling.

\subsection{glp\_set\_rii --- set (change) row scale factor}

\synopsis

\begin{verbatim}
   void glp_set_rii(glp_prob *P, int i, double rii);
\end{verbatim}

\description

The routine \verb|glp_set_rii| sets (changes) the scale factor $r_{ii}$
for $i$-th row of the specified problem object.

\subsection{glp\_set\_sjj --- set (change) column scale factor}

\synopsis

\begin{verbatim}
   void glp_set_sjj(glp_prob *P, int j, double sjj);
\end{verbatim}

\description

The routine \verb|glp_set_sjj| sets (changes) the scale factor $s_{jj}$
for $j$-th column of the specified problem object.

\newpage

\subsection{glp\_get\_rii --- retrieve row scale factor}

\synopsis

\begin{verbatim}
   double glp_get_rii(glp_prob *P, int i);
\end{verbatim}

\returns

The routine \verb|glp_get_rii| returns current scale factor $r_{ii}$
for $i$-th row of the specified problem object.

\vspace*{-6pt}

\subsection{glp\_get\_sjj --- retrieve column scale factor}

\vspace*{-4pt}

\synopsis

\begin{verbatim}
   double glp_get_sjj(glp_prob *P, int j);
\end{verbatim}

\returns

The routine \verb|glp_get_sjj| returns current scale factor $s_{jj}$
for $j$-th column of the specified problem object.

\vspace*{-6pt}

\subsection{glp\_scale\_prob --- scale problem data}

\vspace*{-4pt}

\synopsis

\begin{verbatim}
   void glp_scale_prob(glp_prob *P, int flags);
\end{verbatim}

\description

The routine \verb|glp_scale_prob| performs automatic scaling of problem
data for the specified problem object.

The parameter \verb|flags| specifies scaling options used by the
routine. The options can be combined with the bitwise OR operator and
may be the following:

\verb|GLP_SF_GM  | --- perform geometric mean scaling;

\verb|GLP_SF_EQ  | --- perform equilibration scaling;

\verb|GLP_SF_2N  | --- round scale factors to nearest power of two;

\verb|GLP_SF_SKIP| --- skip scaling, if the problem is well scaled.

The parameter \verb|flags| may be also specified as \verb|GLP_SF_AUTO|,
in which case the routine chooses the scaling options automatically.

\vspace*{-6pt}

\subsection{glp\_unscale\_prob --- unscale problem data}

\vspace*{-4pt}

\synopsis

\begin{verbatim}
   void glp_unscale_prob(glp_prob *P);
\end{verbatim}

The routine \verb|glp_unscale_prob| performs unscaling of problem data
for the specified problem object.

``Unscaling'' means replacing the current scaling matrices $R$ and $S$
by unity matrices that cancels the scaling effect.

%%%%%%%%%%%%%%%%%%%%%%%%%%%%%%%%%%%%%%%%%%%%%%%%%%%%%%%%%%%%%%%%%%%%%%%%

\newpage

\section{LP basis constructing routines}

\subsection{Background}

To start the search the simplex method needs a valid initial basis.
In GLPK the basis is completely defined by a set of {\it statuses}
assigned to {\it all} (auxiliary and structural) variables, where the
status may be one of the following:

\verb|GLP_BS| --- basic variable;

\verb|GLP_NL| --- non-basic variable having active lower bound;

\verb|GLP_NU| --- non-basic variable having active upper bound;

\verb|GLP_NF| --- non-basic free variable;

\verb|GLP_NS| --- non-basic fixed variable.

The basis is {\it valid}, if the basis matrix, which is a matrix built
of columns of the augmented constraint matrix $(I\:|-A)$ corresponding
to basic variables, is non-singular. This, in particular, means that
the number of basic variables must be the same as the number of rows in
the problem object. (For more details see Section \ref{lpbasis}, page
\pageref{lpbasis}.)

Any initial basis may be constructed (or restored) with the API
routines \verb|glp_set_row_stat| and \verb|glp_set_col_stat| by
assigning appropriate statuses to auxiliary and structural variables.
Another way to construct an initial basis is to use API routines like
\verb|glp_adv_basis|, which implement so called
{\it crashing}.\footnote{This term is from early linear programming
systems and means a heuristic to construct a valid initial basis.} Note
that on normal exit the simplex solver remains the basis valid, so in
case of re-optimization there is no need to construct an initial basis
from scratch.

\subsection{glp\_set\_row\_stat --- set (change) row status}

\synopsis

\begin{verbatim}
   void glp_set_row_stat(glp_prob *P, int i, int stat);
\end{verbatim}

\description

The routine \verb|glp_set_row_stat| sets (changes) the current status
of \verb|i|-th row (auxiliary variable) as specified by the parameter
\verb|stat|:

\verb|GLP_BS| --- make the row basic (make the constraint inactive);

\verb|GLP_NL| --- make the row non-basic (make the constraint active);

\verb|GLP_NU| --- make the row non-basic and set it to the upper bound;
if the row is not double-bounded, this status is equivalent to
\verb|GLP_NL| (only in case of this routine);

\verb|GLP_NF| --- the same as \verb|GLP_NL| (only in case of this
routine);

\verb|GLP_NS| --- the same as \verb|GLP_NL| (only in case of this
routine).

\newpage

\subsection{glp\_set\_col\_stat --- set (change) column status}

\synopsis

\begin{verbatim}
   void glp_set_col_stat(glp_prob *P, int j, int stat);
\end{verbatim}

\description

The routine \verb|glp_set_col_stat sets| (changes) the current status
of \verb|j|-th column (structural variable) as specified by the
parameter \verb|stat|:

\verb|GLP_BS| --- make the column basic;

\verb|GLP_NL| --- make the column non-basic;

\verb|GLP_NU| --- make the column non-basic and set it to the upper
bound; if the column is not double-bounded, this status is equivalent
to \verb|GLP_NL| (only in case of this routine);

\verb|GLP_NF| --- the same as \verb|GLP_NL| (only in case of this
routine);

\verb|GLP_NS| --- the same as \verb|GLP_NL| (only in case of this
routine).

\subsection{glp\_std\_basis --- construct standard initial LP basis}

\synopsis

\begin{verbatim}
   void glp_std_basis(glp_prob *P);
\end{verbatim}

\description

The routine \verb|glp_std_basis| constructs the ``standard'' (trivial)
initial LP basis for the specified problem object.

In the ``standard'' LP basis all auxiliary variables (rows) are basic,
and all structural variables (columns) are non-basic (so the
corresponding basis matrix is unity).

\subsection{glp\_adv\_basis --- construct advanced initial LP basis}

\synopsis

\begin{verbatim}
   void glp_adv_basis(glp_prob *P, int flags);
\end{verbatim}

\description

The routine \verb|glp_adv_basis| constructs an advanced initial LP
basis for the specified problem object.

The parameter \verb|flags| is reserved for use in the future and must
be specified as zero.

In order to construct the advanced initial LP basis the routine does
the following:

1) includes in the basis all non-fixed auxiliary variables;

2) includes in the basis as many non-fixed structural variables as
possible keeping the triangular form of the basis matrix;

3) includes in the basis appropriate (fixed) auxiliary variables to
complete the basis.

As a result the initial LP basis has as few fixed variables as possible
and the corresponding basis matrix is triangular.

\subsection{glp\_cpx\_basis --- construct Bixby's initial LP basis}

\synopsis

\begin{verbatim}
   void glp_cpx_basis(glp_prob *P);
\end{verbatim}

\description

The routine \verb|glp_cpx_basis| constructs an initial basis for the
specified problem object with the algorithm proposed by
R.~Bixby.\footnote{Robert E. Bixby, ``Implementing the Simplex Method:
The Initial Basis.'' ORSA Journal on Computing, Vol. 4, No. 3, 1992,
pp. 267-84.}

%%%%%%%%%%%%%%%%%%%%%%%%%%%%%%%%%%%%%%%%%%%%%%%%%%%%%%%%%%%%%%%%%%%%%%%%

\newpage

\section{Simplex method routines}

The {\it simplex method} is a well known efficient numerical procedure
to solve LP problems.

On each iteration the simplex method transforms the original system of
equaility constraints (1.2) resolving them through different sets of
variables to an equivalent system called {\it the simplex table} (or
sometimes {\it the simplex tableau}), which has the following form:
$$
\begin{array}{r@{\:}c@{\:}r@{\:}c@{\:}r@{\:}c@{\:}r}
z&=&d_1(x_N)_1&+&d_2(x_N)_2&+ \dots +&d_n(x_N)_n \\
(x_B)_1&=&\xi_{11}(x_N)_1& +& \xi_{12}(x_N)_2& + \dots +&
   \xi_{1n}(x_N)_n \\
(x_B)_2&=& \xi_{21}(x_N)_1& +& \xi_{22}(x_N)_2& + \dots +&
   \xi_{2n}(x_N)_n \\
\multicolumn{7}{c}
{.\ \ .\ \ .\ \ .\ \ .\ \ .\ \ .\ \ .\ \ .\ \ .\ \ .\ \ .\ \ .\ \ .} \\
(x_B)_m&=&\xi_{m1}(x_N)_1& +& \xi_{m2}(x_N)_2& + \dots +&
   \xi_{mn}(x_N)_n \\
\end{array} \eqno (2.3)
$$
where: $(x_B)_1, (x_B)_2, \dots, (x_B)_m$ are basic variables;
$(x_N)_1, (x_N)_2, \dots, (x_N)_n$ are non-basic variables;
$d_1, d_2, \dots, d_n$ are reduced costs;
$\xi_{11}, \xi_{12}, \dots, \xi_{mn}$ are coefficients of the
simplex table. (May note that the original LP problem (1.1)---(1.3)
also has the form of a simplex table, where all equalities are resolved
through auxiliary variables.)

From the linear programming theory it is known that if an optimal
solution of the LP problem (1.1)---(1.3) exists, it can always be
written in the form (2.3), where non-basic variables are set on their
bounds while values of the objective function and basic variables are
determined by the corresponding equalities of the simplex table.

A set of values of all basic and non-basic variables determined by the
simplex table is called {\it basic solution}. If all basic variables
are within their bounds, the basic solution is called {\it (primal)
feasible}, otherwise it is called {\it (primal) infeasible}. A feasible
basic solution, which provides a smallest (in case of minimization) or
a largest (in case of maximization) value of the objective function is
called {\it optimal}. Therefore, for solving LP problem the simplex
method tries to find its optimal basic solution.

Primal feasibility of some basic solution may be stated by simple
checking if all basic variables are within their bounds. Basic solution
is optimal if additionally the following optimality conditions are
satisfied for all non-basic variables:
\begin{center}
\begin{tabular}{lcc}
Status of $(x_N)_j$ & Minimization & Maximization \\
\hline
$(x_N)_j$ is free & $d_j = 0$ & $d_j = 0$ \\
$(x_N)_j$ is on its lower bound & $d_j \geq 0$ & $d_j \leq 0$ \\
$(x_N)_j$ is on its upper bound & $d_j \leq 0$ & $d_j \geq 0$ \\
\end{tabular}
\end{center}
In other words, basic solution is optimal if there is no non-basic
variable, which changing in the feasible direction (i.e. increasing if
it is free or on its lower bound, or decreasing if it is free or on its
upper bound) can improve (i.e. decrease in case of minimization or
increase in case of maximization) the objective function.

If all non-basic variables satisfy to the optimality conditions shown
above (independently on whether basic variables are within their bounds
or not), the basic solution is called {\it dual feasible}, otherwise it
is called {\it dual infeasible}.

It may happen that some LP problem has no primal feasible solution due
to incorrect\linebreak formulation --- this means that its constraints
conflict with each other. It also may happen that some LP problem has
unbounded solution again due to incorrect formulation --- this means
that some non-basic variable can improve the objective function, i.e.
the optimality conditions are violated, and at the same time this
variable can infinitely change in the feasible direction meeting
no resistance from basic variables. (May note that in the latter case
the LP problem has no dual feasible solution.)

\subsection{glp\_simplex --- solve LP problem with the primal or dual
simplex method}

\synopsis

\begin{verbatim}
   int glp_simplex(glp_prob *P, const glp_smcp *parm);
\end{verbatim}

\description

The routine \verb|glp_simplex| is a driver to the LP solver based on
the simplex method. This routine retrieves problem data from the
specified problem object, calls the solver to solve the problem
instance, and stores results of computations back into the problem
object.

The simplex solver has a set of control parameters. Values of the
control parameters can be passed in the structure \verb|glp_smcp|,
which the parameter \verb|parm| points to. For detailed description of
this structure see paragraph ``Control parameters'' below.
Before specifying some control parameters the application program
should initialize the structure \verb|glp_smcp| by default values of
all control parameters using the routine \verb|glp_init_smcp| (see the
next subsection). This is needed for backward compatibility, because in
the future there may appear new members in the structure
\verb|glp_smcp|.

The parameter \verb|parm| can be specified as \verb|NULL|, in which
case the solver uses default settings.

\returns

\begin{retlist}
0 & The LP problem instance has been successfully solved. (This code
does {\it not} necessarily mean that the solver has found optimal
solution. It only means that the solution process was successful.) \\

\verb|GLP_EBADB| & Unable to start the search, because the initial
basis specified in the problem object is invalid---the number of basic
(auxiliary and structural) variables is not the same as the number of
rows in the problem object.\\

\verb|GLP_ESING| & Unable to start the search, because the basis matrix
corresponding to the initial basis is singular within the working
precision.\\

\verb|GLP_ECOND| & Unable to start the search, because the basis matrix
corresponding to the initial basis is ill-conditioned, i.e. its
condition number is too large.\\

\verb|GLP_EBOUND| & Unable to start the search, because some
double-bounded (auxiliary or structural) variables have incorrect
bounds.\\

\verb|GLP_EFAIL| & The search was prematurely terminated due to the
solver failure.\\

\verb|GLP_EOBJLL| & The search was prematurely terminated, because the
objective function being maximized has reached its lower limit and
continues decreasing (the dual simplex only).\\

\verb|GLP_EOBJUL| & The search was prematurely terminated, because the
objective function being minimized has reached its upper limit and
continues increasing (the dual simplex only).\\

\verb|GLP_EITLIM| & The search was prematurely terminated, because the
simplex iteration limit has been exceeded.\\

\verb|GLP_ETMLIM| & The search was prematurely terminated, because the
time limit has been exceeded.\\
\end{retlist}

\begin{retlist}
\verb|GLP_ENOPFS| & The LP problem instance has no primal feasible
solution (only if the LP presolver is used).\\

\verb|GLP_ENODFS| & The LP problem instance has no dual feasible
solution (only if the LP presolver is used).\\
\end{retlist}

\para{Built-in LP presolver}

The simplex solver has {\it built-in LP presolver}. It is a subprogram
that transforms the original LP problem specified in the problem object
to an equivalent LP problem, which may be easier for solving with the
simplex method than the original one. This is attained mainly due to
reducing the problem size and improving its numeric properties (for
example, by removing some inactive constraints or by fixing some
non-basic variables). Once the transformed LP problem has been solved,
the presolver transforms its basic solution back to the corresponding
basic solution of the original problem.

Presolving is an optional feature of the routine \verb|glp_simplex|,
and by default it is disabled. In order to enable the LP presolver the
control parameter \verb|presolve| should be set to \verb|GLP_ON| (see
paragraph ``Control parameters'' below). Presolving may be used when
the problem instance is solved for the first time. However, on
performing re-optimization the presolver should be disabled.

The presolving procedure is transparent to the API user in the sense
that all necessary processing is performed internally, and a basic
solution of the original problem recovered by the presolver is the same
as if it were computed directly, i.e. without presolving.

Note that the presolver is able to recover only optimal solutions. If
a computed solution is infeasible or non-optimal, the corresponding
solution of the original problem cannot be recovered and therefore
remains undefined. If you need to know a basic solution even if it is
infeasible or non-optimal, the presolver should be disabled.

\para{Terminal output}

Solving large problem instances may take a long time, so the solver
reports some information about the current basic solution, which is
sent to the terminal. This information has the following format:

\begin{verbatim}
   nnn:  obj = xxx  infeas = yyy (num) cnt
\end{verbatim}

\noindent
where: `\verb|nnn|' is the iteration number, `\verb|xxx|' is the
current value of the objective function (it is unscaled and has correct
sign); `\verb|yyy|' is the current sum of primal or dual
infeasibilities (it is scaled and therefore may be used only for visual
estimating), `\verb|num|' is the current number of primal or dual
infeasibilities (phase I) or non-optimalities (phase II), `\verb|cnt|'
is the number of basis factorizations since the last terminal output.

The symbol preceding the iteration number indicates which phase of the
simplex method is in effect:

{\it Blank} means that the solver is searching for primal feasible
solution using the primal simplex or for dual feasible solution using
the dual simplex;

{\it Asterisk} (\verb|*|) means that the solver is searching for
optimal solution using the primal simplex;

{\it Hash} (\verb|#|) means that the solver is searching for optimal
solution using the dual simplex.

\newpage

\para{Control parameters}

This paragraph describes all control parameters currently used in the
simplex solver. Symbolic names of control parameters are names of
corresponding members in the structure \verb|glp_smcp|.

\bigskip

{\tt int msg\_lev} (default: {\tt GLP\_MSG\_ALL})

Message level for terminal output:

\verb|GLP_MSG_OFF| --- no output;

\verb|GLP_MSG_ERR| --- error and warning messages only;

\verb|GLP_MSG_ON | --- normal output;

\verb|GLP_MSG_ALL| --- full output (including informational messages).

\bigskip

{\tt int meth} (default: {\tt GLP\_PRIMAL})

Simplex method option:

\verb|GLP_PRIMAL| --- use two-phase primal simplex;

\verb|GLP_DUAL  | --- use two-phase dual simplex;

\verb|GLP_DUALP | --- use two-phase dual simplex, and if it fails,
switch to the primal simplex.

\bigskip

{\tt int pricing} (default: {\tt GLP\_PT\_PSE})

Pricing technique:

\verb|GLP_PT_STD| --- standard (``textbook'');

\verb|GLP_PT_PSE| --- projected steepest edge.

\bigskip

{\tt int r\_test} (default: {\tt GLP\_RT\_HAR})

Ratio test technique:

\verb|GLP_RT_STD| --- standard (``textbook'');

\verb|GLP_RT_HAR| --- Harris' two-pass ratio test.

\bigskip

{\tt double tol\_bnd} (default: {\tt 1e-7})

Tolerance used to check if the basic solution is primal feasible.
(Do not change this parameter without detailed understanding its
purpose.)

%\newpage
\bigskip

{\tt double tol\_dj} (default: {\tt 1e-7})

Tolerance used to check if the basic solution is dual feasible.
(Do not change this parameter without detailed understanding its
purpose.)

\bigskip

{\tt double tol\_piv} (default: {\tt 1e-9})

Tolerance used to choose eligble pivotal elements of the simplex table.
(Do not change this parameter without detailed understanding its
purpose.)

%\bigskip
\newpage

{\tt double obj\_ll} (default: {\tt -DBL\_MAX})

Lower limit of the objective function. If the objective function
reaches this limit and continues decreasing, the solver terminates the
search. (Used in the dual simplex only.)

\bigskip

{\tt double obj\_ul} (default: {\tt +DBL\_MAX})

Upper limit of the objective function. If the objective function
reaches this limit and continues increasing, the solver terminates the
search. (Used in the dual simplex only.)

\bigskip

{\tt int it\_lim} (default: {\tt INT\_MAX})

Simplex iteration limit.

\bigskip

{\tt int tm\_lim} (default: {\tt INT\_MAX})

Searching time limit, in milliseconds.

\bigskip

{\tt int out\_frq} (default: {\tt 500})

Output frequency, in iterations. This parameter specifies how
frequently the solver sends information about the solution process to
the terminal.

\bigskip

{\tt int out\_dly} (default: {\tt 0})

Output delay, in milliseconds. This parameter specifies how long the
solver should delay sending information about the solution process to
the terminal.

\bigskip

{\tt int presolve} (default: {\tt GLP\_OFF})

LP presolver option:

\verb|GLP_ON | --- enable using the LP presolver;

\verb|GLP_OFF| --- disable using the LP presolver.

\newpage

\para{Example 1}

The following example main program reads LP problem instance in fixed
MPS format from file \verb|25fv47.mps|,\footnote{This instance in fixed
MPS format can be found in the Netlib LP collection; see
{\tt ftp://ftp.netlib.org/lp/data/}.} constructs an advanced initial
basis, solves the instance with the primal simplex method (by default),
and writes the solution to file \verb|25fv47.txt|.

\begin{footnotesize}
\begin{verbatim}
/* spxsamp1.c */

#include <stdio.h>
#include <stdlib.h>
#include <glpk.h>

int main(void)
{     glp_prob *P;
      P = glp_create_prob();
      glp_read_mps(P, GLP_MPS_DECK, NULL, "25fv47.mps");
      glp_adv_basis(P, 0);
      glp_simplex(P, NULL);
      glp_print_sol(P, "25fv47.txt");
      glp_delete_prob(P);
      return 0;
}

/* eof */
\end{verbatim}
\end{footnotesize}

Below here is shown the terminal output from this example program.

\begin{footnotesize}
\begin{verbatim}
Reading problem data from '25fv47.mps'...
Problem: 25FV47
Objective: R0000
822 rows, 1571 columns, 11127 non-zeros
6919 records were read
One free row was removed
Constructing initial basis...
Size of triangular part is 812
GLPK Simplex Optimizer, v4.57
821 rows, 1571 columns, 10400 non-zeros
      0: obj =   7.131703290e+03 inf =   2.145e+05 (204)
    500: obj =   1.886711682e+04 inf =   8.273e+02 (36) 4
    741: obj =   1.846047936e+04 inf =   5.575e-14 (0) 2
*  1000: obj =   9.220063473e+03 inf =   2.423e-14 (432) 2
*  1500: obj =   6.187659664e+03 inf =   1.019e-13 (368) 4
*  2000: obj =   5.503442062e+03 inf =   0.000e+00 (33) 5
*  2052: obj =   5.501845888e+03 inf =   0.000e+00 (0)
OPTIMAL LP SOLUTION FOUND
Writing basic solution to '25fv47.txt'...
\end{verbatim}
\end{footnotesize}

\newpage

\para{Example 2}

The following example main program solves the same LP problem instance
as in Example 1 above, however, it uses the dual simplex method, which
starts from the standard initial basis.

\begin{footnotesize}
\begin{verbatim}
/* spxsamp2.c */

#include <stdio.h>
#include <stdlib.h>
#include <glpk.h>

int main(void)
{     glp_prob *P;
      glp_smcp parm;
      P = glp_create_prob();
      glp_read_mps(P, GLP_MPS_DECK, NULL, "25fv47.mps");
      glp_init_smcp(&parm);
      parm.meth = GLP_DUAL;
      glp_simplex(P, &parm);
      glp_print_sol(P, "25fv47.txt");
      glp_delete_prob(P);
      return 0;
}

/* eof */
\end{verbatim}
\end{footnotesize}

Below here is shown the terminal output from this example program.

\begin{footnotesize}
\begin{verbatim}
Reading problem data from '25fv47.mps'...
Problem: 25FV47
Objective: R0000
822 rows, 1571 columns, 11127 non-zeros
6919 records were read
One free row was removed
GLPK Simplex Optimizer, v4.57
821 rows, 1571 columns, 10400 non-zeros
      0:                         inf =   1.223e+02 (41)
    258:                         inf =   3.091e-16 (0) 2
#   500: obj =  -5.071287080e+03 inf =   2.947e-15 (292) 2
#  1000: obj =  -1.352843873e+03 inf =   8.452e-15 (302) 5
#  1500: obj =   7.985859737e+02 inf =   1.127e-14 (263) 5
#  2000: obj =   3.059023029e+03 inf =   6.290e-11 (197) 4
#  2500: obj =   5.354770966e+03 inf =   7.172e-13 (130) 5
#  2673: obj =   5.501845888e+03 inf =   3.802e-16 (0) 2
OPTIMAL LP SOLUTION FOUND
Writing basic solution to '25fv47.txt'...
\end{verbatim}
\end{footnotesize}

\newpage

\subsection{glp\_exact --- solve LP problem in exact arithmetic}

\synopsis

\begin{verbatim}
   int glp_exact(glp_prob *P, const glp_smcp *parm);
\end{verbatim}

\description

The routine \verb|glp_exact| is a tentative implementation of the
primal two-phase simplex method based on exact (rational) arithmetic.
It is similar to the routine \verb|glp_simplex|, however, for all
internal computations it uses arithmetic of rational numbers, which is
exact in mathematical sense, i.e. free of round-off errors unlike
floating-point arithmetic.

Note that the routine \verb|glp_exact| uses only three control
parameters passed in the structure \verb|glp_smcp|, namely,
\verb|msg_lev|, \verb|it_lim|, and \verb|tm_lim|.

\returns

\begin{retlist}
0 & The LP problem instance has been successfully solved. (This code
does {\it not} necessarily mean that the solver has found optimal
solution. It only means that the solution process was successful.) \\

\verb|GLP_EBADB| & Unable to start the search, because the initial basis
specified in the problem object is invalid---the number of basic
(auxiliary and structural) variables is not the same as the number of
rows in the problem object.\\

\verb|GLP_ESING| & Unable to start the search, because the basis matrix
corresponding to the initial basis is exactly singular.\\

\verb|GLP_EBOUND| & Unable to start the search, because some
double-bounded (auxiliary or structural) variables have incorrect
bounds.\\

\verb|GLP_EFAIL| & The problem instance has no rows/columns.\\

\verb|GLP_EITLIM| & The search was prematurely terminated, because the
simplex iteration limit has been exceeded.\\

\verb|GLP_ETMLIM| & The search was prematurely terminated, because the
time limit has been exceeded.\\
\end{retlist}

\para{Note}

Computations in exact arithmetic are very time-consuming, so solving
LP problem with the routine \verb|glp_exact| from the very beginning is
not a good idea. It is much better at first to find an optimal basis
with the routine \verb|glp_simplex| and only then to call
\verb|glp_exact|, in which case only a few simplex iterations need to
be performed in exact arithmetic.

\newpage

\subsection{glp\_init\_smcp --- initialize simplex solver control
parameters}

\synopsis

\begin{verbatim}
   int glp_init_smcp(glp_smcp *parm);
\end{verbatim}

\description

The routine \verb|glp_init_smcp| initializes control parameters, which
are used by the simplex solver, with default values.

Default values of the control parameters are stored in
a \verb|glp_smcp| structure, which the parameter \verb|parm| points to.

\subsection{glp\_get\_status --- determine generic status of basic
solution}

\synopsis

\begin{verbatim}
   int glp_get_status(glp_prob *P);
\end{verbatim}

\returns

The routine \verb|glp_get_status| reports the generic status of the
current basic solution for the specified problem object as follows:

\verb|GLP_OPT   | --- solution is optimal;

\verb|GLP_FEAS  | --- solution is feasible;

\verb|GLP_INFEAS| --- solution is infeasible;

\verb|GLP_NOFEAS| --- problem has no feasible solution;

\verb|GLP_UNBND | --- problem has unbounded solution;

\verb|GLP_UNDEF | --- solution is undefined.

More detailed information about the status of basic solution can be
retrieved with the routines \verb|glp_get_prim_stat| and
\verb|glp_get_dual_stat|.

\subsection{glp\_get\_prim\_stat --- retrieve status of primal basic
solution}

\synopsis

\begin{verbatim}
   int glp_get_prim_stat(glp_prob *P);
\end{verbatim}

\returns

The routine \verb|glp_get_prim_stat| reports the status of the primal
basic solution for the specified problem object as follows:

\verb|GLP_UNDEF | --- primal solution is undefined;

\verb|GLP_FEAS  | --- primal solution is feasible;

\verb|GLP_INFEAS| --- primal solution is infeasible;

\verb|GLP_NOFEAS| --- no primal feasible solution exists.

\subsection{glp\_get\_dual\_stat --- retrieve status of dual basic
solution}

\synopsis

\begin{verbatim}
   int glp_get_dual_stat(glp_prob *P);
\end{verbatim}

\returns

The routine \verb|glp_get_dual_stat| reports the status of the dual
basic solution for the specified problem object as follows:

\verb|GLP_UNDEF | --- dual solution is undefined;

\verb|GLP_FEAS  | --- dual solution is feasible;

\verb|GLP_INFEAS| --- dual solution is infeasible;

\verb|GLP_NOFEAS| --- no dual feasible solution exists.

\subsection{glp\_get\_obj\_val --- retrieve objective value}

\synopsis

\begin{verbatim}
   double glp_get_obj_val(glp_prob *P);
\end{verbatim}

\returns

The routine \verb|glp_get_obj_val| returns current value of the
objective function.

\subsection{glp\_get\_row\_stat --- retrieve row status}

\synopsis

\begin{verbatim}
   int glp_get_row_stat(glp_prob *P, int i);
\end{verbatim}

\returns

The routine \verb|glp_get_row_stat| returns current status assigned to
the auxiliary variable associated with \verb|i|-th row as follows:

\verb|GLP_BS| --- basic variable;

\verb|GLP_NL| --- non-basic variable on its lower bound;

\verb|GLP_NU| --- non-basic variable on its upper bound;

\verb|GLP_NF| --- non-basic free (unbounded) variable;

\verb|GLP_NS| --- non-basic fixed variable.

%\newpage

\subsection{glp\_get\_row\_prim --- retrieve row primal value}

\synopsis

\begin{verbatim}
   double glp_get_row_prim(glp_prob *P, int i);
\end{verbatim}

\returns

The routine \verb|glp_get_row_prim| returns primal value of the
auxiliary variable associated with \verb|i|-th row.

\subsection{glp\_get\_row\_dual --- retrieve row dual value}

\synopsis

\begin{verbatim}
   double glp_get_row_dual(glp_prob *P, int i);
\end{verbatim}

\returns

The routine \verb|glp_get_row_dual| returns dual value (i.e. reduced
cost) of the auxiliary variable associated with \verb|i|-th row.

\subsection{glp\_get\_col\_stat --- retrieve column status}

\synopsis

\begin{verbatim}
   int glp_get_col_stat(glp_prob *P, int j);
\end{verbatim}

\returns

The routine \verb|glp_get_col_stat| returns current status assigned to
the structural variable associated with \verb|j|-th column as follows:

\verb|GLP_BS| --- basic variable;

\verb|GLP_NL| --- non-basic variable on its lower bound;

\verb|GLP_NU| --- non-basic variable on its upper bound;

\verb|GLP_NF| --- non-basic free (unbounded) variable;

\verb|GLP_NS| --- non-basic fixed variable.

\subsection{glp\_get\_col\_prim --- retrieve column primal value}

\synopsis

\begin{verbatim}
   double glp_get_col_prim(glp_prob *P, int j);
\end{verbatim}

\returns

The routine \verb|glp_get_col_prim| returns primal value of the
structural variable associated with \verb|j|-th column.

%\newpage

\subsection{glp\_get\_col\_dual --- retrieve column dual value}

\synopsis

\begin{verbatim}
   double glp_get_col_dual(glp_prob *P, int j);
\end{verbatim}

\returns

The routine \verb|glp_get_col_dual| returns dual value (i.e. reduced
cost) of the structural variable associated with \verb|j|-th column.

\newpage

\subsection{glp\_get\_unbnd\_ray --- determine variable causing
unboundedness}

\synopsis

\begin{verbatim}
   int glp_get_unbnd_ray(glp_prob *P);
\end{verbatim}

\returns

The routine \verb|glp_get_unbnd_ray| returns the number $k$ of
a variable, which causes primal or dual unboundedness.
If $1\leq k\leq m$, it is $k$-th auxiliary variable, and if
$m+1\leq k\leq m+n$, it is $(k-m)$-th structural variable, where $m$ is
the number of rows, $n$ is the number of columns in the problem object.
If such variable is not defined, the routine returns 0.

\para{Note}

If it is not exactly known which version of the simplex solver
detected unboundedness, i.e. whether the unboundedness is primal or
dual, it is sufficient to check the status of the variable
with the routine \verb|glp_get_row_stat| or \verb|glp_get_col_stat|.
If the variable is non-basic, the unboundedness is primal, otherwise,
if the variable is basic, the unboundedness is dual (the latter case
means that the problem has no primal feasible dolution).

%%%%%%%%%%%%%%%%%%%%%%%%%%%%%%%%%%%%%%%%%%%%%%%%%%%%%%%%%%%%%%%%%%%%%%%%

\newpage

\section{Interior-point method routines}

{\it Interior-point methods} (also known as {\it barrier methods}) are
more modern and powerful numerical methods for large-scale linear
programming. Such methods are especially efficient for very sparse LP
problems and allow solving such problems much faster than the simplex
method.

In brief, the GLPK interior-point solver works as follows.

At first, the solver transforms the original LP to a {\it working} LP
in the standard format:

\medskip

\noindent
\hspace{.5in} minimize
$$z = c_1x_{m+1} + c_2x_{m+2} + \dots + c_nx_{m+n} + c_0 \eqno (2.4)$$
\hspace{.5in} subject to linear constraints
$$
\begin{array}{r@{\:}c@{\:}r@{\:}c@{\:}r@{\:}c@{\:}l}
a_{11}x_{m+1}&+&a_{12}x_{m+2}&+ \dots +&a_{1n}x_{m+n}&=&b_1 \\
a_{21}x_{m+1}&+&a_{22}x_{m+2}&+ \dots +&a_{2n}x_{m+n}&=&b_2 \\
\multicolumn{7}{c}
{.\ \ .\ \ .\ \ .\ \ .\ \ .\ \ .\ \ .\ \ .\ \ .\ \ .\ \ .\ \ .\ \ .} \\
a_{m1}x_{m+1}&+&a_{m2}x_{m+2}&+ \dots +&a_{mn}x_{m+n}&=&b_m \\
\end{array} \eqno (2.5)
$$
\hspace{.5in} and non-negative variables
$$x_1\geq 0,\ \ x_2\geq 0,\ \ \dots,\ \ x_n\geq 0 \eqno(2.6)$$
where: $z$ is the objective function; $x_1$, \dots, $x_n$ are variables;
$c_1$, \dots, $c_n$ are objective coefficients; $c_0$ is a constant term
of the objective function; $a_{11}$, \dots, $a_{mn}$ are constraint
coefficients; $b_1$, \dots, $b_m$ are right-hand sides.

Using vector and matrix notations the working LP (2.4)---(2.6) can be
written as follows:
$$z=c^Tx+c_0\ \rightarrow\ \min,\eqno(2.7)$$
$$Ax=b,\eqno(2.8)$$
$$x\geq 0,\eqno(2.9)$$
where: $x=(x_j)$ is $n$-vector of variables, $c=(c_j)$ is $n$-vector of
objective coefficients, $A=(a_{ij})$ is $m\times n$-matrix of
constraint coefficients, and $b=(b_i)$ is $m$-vector of right-hand
sides.

Karush--Kuhn--Tucker optimality conditions for LP (2.7)---(2.9) are the
following:
$$Ax=b,\eqno(2.10)$$
$$A^T\pi+\lambda=c,\eqno(2.11)$$
$$\lambda^Tx=0,\eqno(2.12)$$
$$x\geq 0,\ \ \lambda\geq 0,\eqno(2.13)$$
where:
$\pi$ is $m$-vector of Lagrange multipliers (dual variables) for
equality constraints (2.8),\linebreak $\lambda$ is $n$-vector of
Lagrange multipliers (dual variables) for non-negativity constraints
(2.9),\linebreak (2.10) is the primal feasibility condition, (2.11) is
the dual feasibility condition, (2.12) is the primal-dual
complementarity condition, and (2.13) is the non-negativity conditions.

The main idea of the primal-dual interior-point method is based on
finding a point in the primal-dual space (i.e. in the space of all
primal and dual variables $x$, $\pi$, and $\lambda$), which satisfies
to all optimality conditions (2.10)---(2.13). Obviously, $x$-component
of such point then provides an optimal solution to the working LP
(2.7)---(2.9).

To find the optimal point $(x^*,\pi^*,\lambda^*)$ the interior-point
method attempts to solve the system of equations (2.10)---(2.12), which
is closed in the sense that the number of variables $x_j$, $\pi_i$, and
$\lambda_j$ and the number equations are the same and equal to $m+2n$.
Due to condition (2.12) this system of equations is non-linear, so it
can be solved with a version of {\it Newton's method} provided with
additional rules to keep the current point within the positive orthant
as required by the non-negativity conditions (2.13).

Finally, once the optimal point $(x^*,\pi^*,\lambda^*)$ has been found,
the solver performs inverse transformations to recover corresponding
solution to the original LP passed to the solver from the application
program.

\subsection{glp\_interior --- solve LP problem with the interior-point
method}

\synopsis

\begin{verbatim}
   int glp_interior(glp_prob *P, const glp_iptcp *parm);
\end{verbatim}

\description

The routine \verb|glp_interior| is a driver to the LP solver based on
the primal-dual interior-point method. This routine retrieves problem
data from the specified problem object, calls the solver to solve the
problem instance, and stores results of computations back into the
problem object.

The interior-point solver has a set of control parameters. Values of
the control parameters can be passed in the structure \verb|glp_iptcp|,
which the parameter \verb|parm| points to. For detailed description of
this structure see paragraph ``Control parameters'' below. Before
specifying some control parameters the application program should
initialize the structure \verb|glp_iptcp| by default values of all
control parameters using the routine \verb|glp_init_iptcp| (see the
next subsection). This is needed for backward compatibility, because in
the future there may appear new members in the structure
\verb|glp_iptcp|.

The parameter \verb|parm| can be specified as \verb|NULL|, in which
case the solver uses default settings.

\returns

\begin{retlist}
0 & The LP problem instance has been successfully solved. (This code
does {\it not} necessarily mean that the solver has found optimal
solution. It only means that the solution process was successful.) \\

\verb|GLP_EFAIL| & The problem has no rows/columns.\\

\verb|GLP_ENOCVG| & Very slow convergence or divergence.\\

\verb|GLP_EITLIM| & Iteration limit exceeded.\\

\verb|GLP_EINSTAB| & Numerical instability on solving Newtonian
system.\\
\end{retlist}

\newpage

\para{Comments}

The routine \verb|glp_interior| implements an easy version of
the primal-dual interior-point method based on Mehrotra's
technique.\footnote{S. Mehrotra. On the implementation of a primal-dual
interior point method. SIAM J. on Optim., 2(4), pp. 575-601, 1992.}

Note that currently the GLPK interior-point solver does not include
many important features, in particular:

%\vspace*{-8pt}

%\begin{itemize}
\Item{---}it is not able to process dense columns. Thus, if the
constraint matrix of the LP problem has dense columns, the solving
process may be inefficient;

\Item{---}it has no features against numerical instability. For some LP
problems premature termination may happen if the matrix $ADA^T$ becomes
singular or ill-conditioned;

\Item{---}it is not able to identify the optimal basis, which
corresponds to the interior-point solution found.
%\end{itemize}

%\vspace*{-8pt}

\para{Terminal output}

Solving large LP problems may take a long time, so the solver reports
some information about every interior-point iteration,\footnote{Unlike
the simplex method the interior point method usually needs 30---50
iterations (independently on the problem size) in order to find an
optimal solution.} which is sent to the terminal. This information has
the following format:

\begin{verbatim}
nnn: obj = fff; rpi = ppp; rdi = ddd; gap = ggg
\end{verbatim}

\noindent where: \verb|nnn| is iteration number, \verb|fff| is the
current value of the objective function (in the case of maximization it
has wrong sign), \verb|ppp| is the current relative primal
infeasibility (cf. (2.10)):
$$\frac{\|Ax^{(k)}-b\|}{1+\|b\|},\eqno(2.14)$$
\verb|ddd| is the current relative dual infeasibility (cf. (2.11)):
$$\frac{\|A^T\pi^{(k)}+\lambda^{(k)}-c\|}{1+\|c\|},\eqno(2.15)$$
\verb|ggg| is the current primal-dual gap (cf. (2.12)):
$$\frac{|c^Tx^{(k)}-b^T\pi^{(k)}|}{1+|c^Tx^{(k)}|},\eqno(2.16)$$
and $[x^{(k)},\pi^{(k)},\lambda^{(k)}]$ is the current point on $k$-th
iteration, $k=0,1,2,\dots$\ . Note that all solution components are
internally scaled, so information sent to the terminal is suitable only
for visual inspection.

\newpage

\para{Control parameters}

This paragraph describes all control parameters currently used in the
interior-point solver. Symbolic names of control parameters are names of
corresponding members in the structure \verb|glp_iptcp|.

\bigskip

{\tt int msg\_lev} (default: {\tt GLP\_MSG\_ALL})

Message level for terminal output:

\verb|GLP_MSG_OFF|---no output;

\verb|GLP_MSG_ERR|---error and warning messages only;

\verb|GLP_MSG_ON |---normal output;

\verb|GLP_MSG_ALL|---full output (including informational messages).

\bigskip

{\tt int ord\_alg} (default: {\tt GLP\_ORD\_AMD})

Ordering algorithm used prior to Cholesky factorization:

\verb|GLP_ORD_NONE  |---use natural (original) ordering;

\verb|GLP_ORD_QMD   |---quotient minimum degree (QMD);

\verb|GLP_ORD_AMD   |---approximate minimum degree (AMD);

\verb|GLP_ORD_SYMAMD|---approximate minimum degree (SYMAMD).

\bigskip

\para{Example}

The following main program reads LP problem instance in fixed MPS
format from file\linebreak \verb|25fv47.mps|,\footnote{This instance in
fixed MPS format can be found in the Netlib LP collection; see
{\tt ftp://ftp.netlib.org/lp/data/}.} solves it with the interior-point
solver, and writes the solution to file \verb|25fv47.txt|.

\begin{footnotesize}
\begin{verbatim}
/* iptsamp.c */

#include <stdio.h>
#include <stdlib.h>
#include <glpk.h>

int main(void)
{     glp_prob *P;
      P = glp_create_prob();
      glp_read_mps(P, GLP_MPS_DECK, NULL, "25fv47.mps");
      glp_interior(P, NULL);
      glp_print_ipt(P, "25fv47.txt");
      glp_delete_prob(P);
      return 0;
}

/* eof */
\end{verbatim}
\end{footnotesize}

\newpage

Below here is shown the terminal output from this example program.

\begin{footnotesize}
\begin{verbatim}
Reading problem data from `25fv47.mps'...
Problem: 25FV47
Objective: R0000
822 rows, 1571 columns, 11127 non-zeros
6919 records were read
Original LP has 822 row(s), 1571 column(s), and 11127 non-zero(s)
Working LP has 821 row(s), 1876 column(s), and 10705 non-zero(s)
Matrix A has 10705 non-zeros
Matrix S = A*A' has 11895 non-zeros (upper triangle)
Minimal degree ordering...
Computing Cholesky factorization S = L'*L...
Matrix L has 35411 non-zeros
Guessing initial point...
Optimization begins...
  0: obj =   1.823377629e+05; rpi =  1.3e+01; rdi =  1.4e+01; gap =  9.3e-01
  1: obj =   9.260045192e+04; rpi =  5.3e+00; rdi =  5.6e+00; gap =  6.8e+00
  2: obj =   3.596999742e+04; rpi =  1.5e+00; rdi =  1.2e+00; gap =  1.8e+01
  3: obj =   1.989627568e+04; rpi =  4.7e-01; rdi =  3.0e-01; gap =  1.9e+01
  4: obj =   1.430215557e+04; rpi =  1.1e-01; rdi =  8.6e-02; gap =  1.4e+01
  5: obj =   1.155716505e+04; rpi =  2.3e-02; rdi =  2.4e-02; gap =  6.8e+00
  6: obj =   9.660273208e+03; rpi =  6.7e-03; rdi =  4.6e-03; gap =  3.9e+00
  7: obj =   8.694348283e+03; rpi =  3.7e-03; rdi =  1.7e-03; gap =  2.0e+00
  8: obj =   8.019543639e+03; rpi =  2.4e-03; rdi =  3.9e-04; gap =  1.0e+00
  9: obj =   7.122676293e+03; rpi =  1.2e-03; rdi =  1.5e-04; gap =  6.6e-01
 10: obj =   6.514534518e+03; rpi =  6.1e-04; rdi =  4.3e-05; gap =  4.1e-01
 11: obj =   6.361572203e+03; rpi =  4.8e-04; rdi =  2.2e-05; gap =  3.0e-01
 12: obj =   6.203355508e+03; rpi =  3.2e-04; rdi =  1.7e-05; gap =  2.6e-01
 13: obj =   6.032943411e+03; rpi =  2.0e-04; rdi =  9.3e-06; gap =  2.1e-01
 14: obj =   5.796553021e+03; rpi =  9.8e-05; rdi =  3.2e-06; gap =  1.0e-01
 15: obj =   5.667032431e+03; rpi =  4.4e-05; rdi =  1.1e-06; gap =  5.6e-02
 16: obj =   5.613911867e+03; rpi =  2.5e-05; rdi =  4.1e-07; gap =  3.5e-02
 17: obj =   5.560572626e+03; rpi =  9.9e-06; rdi =  2.3e-07; gap =  2.1e-02
 18: obj =   5.537276001e+03; rpi =  5.5e-06; rdi =  8.4e-08; gap =  1.1e-02
 19: obj =   5.522746942e+03; rpi =  2.2e-06; rdi =  4.0e-08; gap =  6.7e-03
 20: obj =   5.509956679e+03; rpi =  7.5e-07; rdi =  1.8e-08; gap =  2.9e-03
 21: obj =   5.504571733e+03; rpi =  1.6e-07; rdi =  5.8e-09; gap =  1.1e-03
 22: obj =   5.502576367e+03; rpi =  3.4e-08; rdi =  1.0e-09; gap =  2.5e-04
 23: obj =   5.502057119e+03; rpi =  8.1e-09; rdi =  3.0e-10; gap =  7.7e-05
 24: obj =   5.501885996e+03; rpi =  9.4e-10; rdi =  1.2e-10; gap =  2.4e-05
 25: obj =   5.501852464e+03; rpi =  1.4e-10; rdi =  1.2e-11; gap =  3.0e-06
 26: obj =   5.501846549e+03; rpi =  1.4e-11; rdi =  1.2e-12; gap =  3.0e-07
 27: obj =   5.501845954e+03; rpi =  1.4e-12; rdi =  1.2e-13; gap =  3.0e-08
 28: obj =   5.501845895e+03; rpi =  1.5e-13; rdi =  1.2e-14; gap =  3.0e-09
OPTIMAL SOLUTION FOUND
Writing interior-point solution to `25fv47.txt'...
\end{verbatim}
\end{footnotesize}

\newpage

\subsection{glp\_init\_iptcp --- initialize interior-point solver
control parameters}

\synopsis

\begin{verbatim}
   int glp_init_iptcp(glp_iptcp *parm);
\end{verbatim}

\description

The routine \verb|glp_init_iptcp| initializes control parameters, which
are used by the interior-point solver, with default values.

Default values of the control parameters are stored in the structure
\verb|glp_iptcp|, which the parameter \verb|parm| points to.

\subsection{glp\_ipt\_status --- determine solution status}

\synopsis

\begin{verbatim}
   int glp_ipt_status(glp_prob *P);
\end{verbatim}

\returns

The routine \verb|glp_ipt_status| reports the status of a solution
found by the interior-point solver as follows:

\verb|GLP_UNDEF | --- interior-point solution is undefined;

\verb|GLP_OPT   | --- interior-point solution is optimal;

\verb|GLP_INFEAS| --- interior-point solution is infeasible;

\verb|GLP_NOFEAS| --- no feasible primal-dual solution exists.

\subsection{glp\_ipt\_obj\_val --- retrieve objective value}

\synopsis

\begin{verbatim}
   double glp_ipt_obj_val(glp_prob *P);
\end{verbatim}

\returns

The routine \verb|glp_ipt_obj_val| returns value of the objective
function for interior-point solution.

\subsection{glp\_ipt\_row\_prim --- retrieve row primal value}

\synopsis

\begin{verbatim}
   double glp_ipt_row_prim(glp_prob *P, int i);
\end{verbatim}

\returns

The routine \verb|glp_ipt_row_prim| returns primal value of the
auxiliary variable associated with \verb|i|-th row.

\newpage

\subsection{glp\_ipt\_row\_dual --- retrieve row dual value}

\synopsis

\begin{verbatim}
   double glp_ipt_row_dual(glp_prob *P, int i);
\end{verbatim}

\returns

The routine \verb|glp_ipt_row_dual| returns dual value (i.e. reduced
cost) of the auxiliary variable associated with \verb|i|-th row.

\subsection{glp\_ipt\_col\_prim --- retrieve column primal value}

\synopsis

\begin{verbatim}
   double glp_ipt_col_prim(glp_prob *P, int j);
\end{verbatim}

\returns

The routine \verb|glp_ipt_col_prim| returns primal value of the
structural variable associated with \verb|j|-th column.

\subsection{glp\_ipt\_col\_dual --- retrieve column dual value}

\synopsis

\begin{verbatim}
   double glp_ipt_col_dual(glp_prob *P, int j);
\end{verbatim}

\returns

The routine \verb|glp_ipt_col_dual| returns dual value (i.e. reduced
cost) of the structural variable associated with \verb|j|-th column.

%%%%%%%%%%%%%%%%%%%%%%%%%%%%%%%%%%%%%%%%%%%%%%%%%%%%%%%%%%%%%%%%%%%%%%%%

\newpage

\section{Mixed integer programming routines}

\subsection{glp\_set\_col\_kind --- set (change) column kind}

\synopsis

\begin{verbatim}
   void glp_set_col_kind(glp_prob *P, int j, int kind);
\end{verbatim}

\description

The routine \verb|glp_set_col_kind| sets (changes) the kind of
\verb|j|-th column (structural variable) as specified by the parameter
\verb|kind|:

\verb|GLP_CV| --- continuous variable;

\verb|GLP_IV| --- integer variable;

\verb|GLP_BV| --- binary variable.

Setting a column to \verb|GLP_BV| has the same effect as if it were
set to \verb|GLP_IV|, its lower bound were set 0, and its upper bound
were set to 1.

\subsection{glp\_get\_col\_kind --- retrieve column kind}

\synopsis

\begin{verbatim}
   int glp_get_col_kind(glp_prob *P, int j);
\end{verbatim}

\returns

The routine \verb|glp_get_col_kind| returns the kind of \verb|j|-th
column (structural variable) as follows:

\verb|GLP_CV| --- continuous variable;

\verb|GLP_IV| --- integer variable;

\verb|GLP_BV| --- binary variable.

\subsection{glp\_get\_num\_int --- retrieve number of integer columns}

\synopsis

\begin{verbatim}
   int glp_get_num_int(glp_prob *P);
\end{verbatim}

\returns

The routine \verb|glp_get_num_int| returns the number of columns
(structural variables), which are marked as integer. Note that this
number {\it does} include binary columns.

\newpage

\subsection{glp\_get\_num\_bin --- retrieve number of binary columns}

\synopsis

\begin{verbatim}
   int glp_get_num_bin(glp_prob *P);
\end{verbatim}

\returns

The routine \verb|glp_get_num_bin| returns the number of columns
(structural variables), which are marked as integer and whose lower
bound is zero and upper bound is one.

\subsection{glp\_intopt --- solve MIP problem with the branch-and-cut
method}

\synopsis

\begin{verbatim}
   int glp_intopt(glp_prob *P, const glp_iocp *parm);
\end{verbatim}

\description

The routine \verb|glp_intopt| is a driver to the MIP solver based on
the branch-and-cut method, which is a hybrid of branch-and-bound and
cutting plane methods.

If the presolver is disabled (see paragraph ``Control parameters''
below), on entry to the routine \verb|glp_intopt| the problem object,
which the parameter \verb|mip| points to, should contain optimal
solution to LP relaxation (it can be obtained, for example, with the
routine \verb|glp_simplex|). Otherwise, if the presolver is enabled, it
is not necessary.

The MIP solver has a set of control parameters. Values of the control
parameters can be passed in the structure \verb|glp_iocp|, which the
parameter \verb|parm| points to. For detailed description of this
structure see paragraph ``Control parameters'' below. Before specifying
some control parameters the application program should initialize the
structure \verb|glp_iocp| by default values of all control parameters
using the routine \verb|glp_init_iocp| (see the next subsection). This
is needed for backward compatibility, because in the future there may
appear new members in the structure \verb|glp_iocp|.

The parameter \verb|parm| can be specified as \verb|NULL|, in which case
the solver uses default settings.

Note that the GLPK branch-and-cut solver is not perfect, so it is
unable to solve hard or very large scale MIP instances for a reasonable
time.

\returns

\begin{retlist}
0 & The MIP problem instance has been successfully solved. (This code
does {\it not} necessarily mean that the solver has found optimal
solution. It only means that the solution process was successful.) \\

\verb|GLP_EBOUND| & Unable to start the search, because some
double-bounded variables have incorrect bounds or some integer
variables have non-integer (fractional) bounds.\\

\verb|GLP_EROOT| & Unable to start the search, because optimal basis
for initial LP relaxation is not provided. (This code may appear only
if the presolver is disabled.)\\

\verb|GLP_ENOPFS| & Unable to start the search, because LP relaxation
of the MIP problem instance has no primal feasible solution. (This code
may appear only if the presolver is enabled.)\\
\end{retlist}

\newpage

\begin{retlist}
\verb|GLP_ENODFS| & Unable to start the search, because LP relaxation
of the MIP problem instance has no dual feasible solution. In other
word, this code means that if the LP relaxation has at least one primal
feasible solution, its optimal solution is unbounded, so if the MIP
problem has at least one integer feasible solution, its (integer)
optimal solution is also unbounded. (This code may appear only if the
presolver is enabled.)\\

\verb|GLP_EFAIL| & The search was prematurely terminated due to the
solver failure.\\

\verb|GLP_EMIPGAP| & The search was prematurely terminated, because the
relative mip gap tolerance has been reached.\\

\verb|GLP_ETMLIM| & The search was prematurely terminated, because the
time limit has been exceeded.\\

\verb|GLP_ESTOP| & The search was prematurely terminated by application.
(This code may appear only if the advanced solver interface is used.)\\
\end{retlist}

\para{Built-in MIP presolver}

The branch-and-cut solver has {\it built-in MIP presolver}. It is
a subprogram that transforms the original MIP problem specified in the
problem object to an equivalent MIP problem, which may be easier for
solving with the branch-and-cut method than the original one. For
example, the presolver can remove redundant constraints and variables,
whose optimal values are known, perform bound and coefficient reduction,
etc. Once the transformed MIP problem has been solved, the presolver
transforms its solution back to corresponding solution of the original
problem.

Presolving is an optional feature of the routine \verb|glp_intopt|, and
by default it is disabled. In order to enable the MIP presolver, the
control parameter \verb|presolve| should be set to \verb|GLP_ON| (see
paragraph ``Control parameters'' below).

\para{Advanced solver interface}

The routine \verb|glp_intopt| allows the user to control the
branch-and-cut search by passing to the solver a user-defined callback
routine. For more details see Chapter ``Branch-and-Cut API Routines''.

\para{Terminal output}

Solving a MIP problem may take a long time, so the solver reports some
information about best known solutions, which is sent to the terminal.
This information has the following format:

\begin{verbatim}
+nnn: mip = xxx <rho> yyy gap (ppp; qqq)
\end{verbatim}

\noindent
where: `\verb|nnn|' is the simplex iteration number; `\verb|xxx|' is a
value of the objective function for the best known integer feasible
solution (if no integer feasible solution has been found yet,
`\verb|xxx|' is the text `\verb|not found yet|'); `\verb|rho|' is the
string `\verb|>=|' (in case of minimization) or `\verb|<=|' (in case of
maximization); `\verb|yyy|' is a global bound for exact integer optimum
(i.e. the exact integer optimum is always in the range from `\verb|xxx|'
to `\verb|yyy|'); `\verb|gap|' is the relative mip gap, in percents,
computed as $gap=|xxx-yyy|/(|xxx|+{\tt DBL\_EPSILON})\cdot 100\%$ (if
$gap$ is greater than $999.9\%$, it is not printed); `\verb|ppp|' is the
number of subproblems in the active list, `\verb|qqq|' is the number of
subproblems which have been already fathomed and therefore removed from
the branch-and-bound search tree.

\newpage

\subsubsection{Control parameters}

This paragraph describes all control parameters currently used in the
MIP solver. Symbolic names of control parameters are names of
corresponding members in the structure \verb|glp_iocp|.

\bigskip\vspace*{-2pt}

{\tt int msg\_lev} (default: {\tt GLP\_MSG\_ALL})

Message level for terminal output:

\verb|GLP_MSG_OFF| --- no output;

\verb|GLP_MSG_ERR| --- error and warning messages only;

\verb|GLP_MSG_ON | --- normal output;

\verb|GLP_MSG_ALL| --- full output (including informational messages).

\bigskip\vspace*{-2pt}

{\tt int br\_tech} (default: {\tt GLP\_BR\_DTH})

Branching technique option:

\verb|GLP_BR_FFV| --- first fractional variable;

\verb|GLP_BR_LFV| --- last fractional variable;

\verb|GLP_BR_MFV| --- most fractional variable;

\verb|GLP_BR_DTH| --- heuristic by Driebeck and Tomlin;

\verb|GLP_BR_PCH| --- hybrid pseudo-cost heuristic.

\bigskip\vspace*{-2pt}

{\tt int bt\_tech} (default: {\tt GLP\_BT\_BLB})

Backtracking technique option:

\verb|GLP_BT_DFS| --- depth first search;

\verb|GLP_BT_BFS| --- breadth first search;

\verb|GLP_BT_BLB| --- best local bound;

\verb|GLP_BT_BPH| --- best projection heuristic.

\bigskip\vspace*{-2pt}

{\tt int pp\_tech} (default: {\tt GLP\_PP\_ALL})

Preprocessing technique option:

\verb|GLP_PP_NONE| --- disable preprocessing;

\verb|GLP_PP_ROOT| --- perform preprocessing only on the root level;

\verb|GLP_PP_ALL | --- perform preprocessing on all levels.

\bigskip\vspace*{-2pt}

{\tt int sr\_heur} (default: {\tt GLP\_ON})

Simple rounding heuristic option:

\verb|GLP_ON | --- enable applying the simple rounding heuristic;

\verb|GLP_OFF| --- disable applying the simple rounding heuristic.

\newpage

{\tt int fp\_heur} (default: {\tt GLP\_OFF})

Feasibility pump heuristic option:

\verb|GLP_ON | --- enable applying the feasibility pump heuristic;

\verb|GLP_OFF| --- disable applying the feasibility pump heuristic.

\bigskip

{\tt int ps\_heur} (default: {\tt GLP\_OFF})

Proximity search heuristic\footnote{The Fischetti--Monaci Proximity
Search (a.k.a. Proxy) heuristic. This algorithm is often capable of
rapidly improving a feasible solution of a MIP problem with binary
variables. It allows to quickly obtain suboptimal solutions in some
problems which take too long time to be solved to optimality.} option:

\verb|GLP_ON | --- enable applying the proximity search heuristic;

\verb|GLP_OFF| --- disable applying the proximity search pump heuristic.

\bigskip

{\tt int ps\_tm\_lim} (default: {\tt 60000})

Time limit, in milliseconds, for the proximity search heuristic (see
above).

\bigskip

{\tt int gmi\_cuts} (default: {\tt GLP\_OFF})

Gomory's mixed integer cut option:

\verb|GLP_ON | --- enable generating Gomory's cuts;

\verb|GLP_OFF| --- disable generating Gomory's cuts.

\bigskip

{\tt int mir\_cuts} (default: {\tt GLP\_OFF})

Mixed integer rounding (MIR) cut option:

\verb|GLP_ON | --- enable generating MIR cuts;

\verb|GLP_OFF| --- disable generating MIR cuts.

\bigskip

{\tt int cov\_cuts} (default: {\tt GLP\_OFF})

Mixed cover cut option:

\verb|GLP_ON | --- enable generating mixed cover cuts;

\verb|GLP_OFF| --- disable generating mixed cover cuts.

\bigskip

{\tt int clq\_cuts} (default: {\tt GLP\_OFF})

Clique cut option:

\verb|GLP_ON | --- enable generating clique cuts;

\verb|GLP_OFF| --- disable generating clique cuts.

\newpage

{\tt double tol\_int} (default: {\tt 1e-5})

Absolute tolerance used to check if optimal solution to the current LP
relaxation is integer feasible. (Do not change this parameter without
detailed understanding its purpose.)

\bigskip

{\tt double tol\_obj} (default: {\tt 1e-7})

Relative tolerance used to check if the objective value in optimal
solution to the current LP relaxation is not better than in the best
known integer feasible solution. (Do not change this parameter without
detailed understanding its purpose.)

\bigskip

{\tt double mip\_gap} (default: {\tt 0.0})

The relative mip gap tolerance. If the relative mip gap for currently
known best integer feasible solution falls below this tolerance, the
solver terminates the search. This allows obtainig suboptimal integer
feasible solutions if solving the problem to optimality takes too long
time.

\bigskip

{\tt int tm\_lim} (default: {\tt INT\_MAX})

Searching time limit, in milliseconds.

\bigskip

{\tt int out\_frq} (default: {\tt 5000})

Output frequency, in milliseconds. This parameter specifies how
frequently the solver sends information about the solution process to
the terminal.

\bigskip

{\tt int out\_dly} (default: {\tt 10000})

Output delay, in milliseconds. This parameter specifies how long the
solver should delay sending information about solution of the current
LP relaxation with the simplex method to the terminal.

\bigskip

{\tt void (*cb\_func)(glp\_tree *tree, void *info)}
(default: {\tt NULL})

Entry point to the user-defined callback routine. \verb|NULL| means
the advanced solver interface is not used. For more details see Chapter
``Branch-and-Cut API Routines''.

\bigskip

{\tt void *cb\_info} (default: {\tt NULL})

Transit pointer passed to the routine \verb|cb_func| (see above).

\bigskip

{\tt int cb\_size} (default: {\tt 0})

The number of extra (up to 256) bytes allocated for each node of the
branch-and-bound tree to store application-specific data. On creating
a node these bytes are initialized by binary zeros.

\bigskip

{\tt int presolve} (default: {\tt GLP\_OFF})

MIP presolver option:

\verb|GLP_ON | --- enable using the MIP presolver;

\verb|GLP_OFF| --- disable using the MIP presolver.

\newpage

{\tt int binarize} (default: {\tt GLP\_OFF})

Binarization option (used only if the presolver is enabled):

\verb|GLP_ON | --- replace general integer variables by binary ones;

\verb|GLP_OFF| --- do not use binarization.

\subsection{glp\_init\_iocp --- initialize integer optimizer control
parameters}

\synopsis

\begin{verbatim}
   void glp_init_iocp(glp_iocp *parm);
\end{verbatim}

\description

The routine \verb|glp_init_iocp| initializes control parameters, which
are used by the branch-and-cut solver, with default values.

Default values of the control parameters are stored in
a \verb|glp_iocp| structure, which the parameter \verb|parm| points to.

\subsection{glp\_mip\_status --- determine status of MIP solution}

\synopsis

\begin{verbatim}
   int glp_mip_status(glp_prob *P);
\end{verbatim}

\returns

The routine \verb|glp_mip_status| reports the status of a MIP solution
found by the MIP solver as follows:

\verb|GLP_UNDEF | --- MIP solution is undefined;

\verb|GLP_OPT   | --- MIP solution is integer optimal;

\verb|GLP_FEAS  | --- MIP solution is integer feasible, however, its
optimality (or non-optimality) has not been proven, perhaps due to
premature termination of the search;

\verb|GLP_NOFEAS| --- problem has no integer feasible solution (proven
by the solver).

\subsection{glp\_mip\_obj\_val --- retrieve objective value}

\synopsis

\begin{verbatim}
   double glp_mip_obj_val(glp_prob *P);
\end{verbatim}

\returns

The routine \verb|glp_mip_obj_val| returns value of the objective
function for MIP solution.

\newpage

\subsection{glp\_mip\_row\_val --- retrieve row value}

\synopsis

\begin{verbatim}
   double glp_mip_row_val(glp_prob *P, int i);
\end{verbatim}

\returns

The routine \verb|glp_mip_row_val| returns value of the auxiliary
variable associated with \verb|i|-th row for MIP solution.

\subsection{glp\_mip\_col\_val --- retrieve column value}

\synopsis

\begin{verbatim}
   double glp_mip_col_val(glp_prob *P, int j);
\end{verbatim}

\returns

The routine \verb|glp_mip_col_val| returns value of the structural
variable associated with \verb|j|-th column for MIP solution.

%%%%%%%%%%%%%%%%%%%%%%%%%%%%%%%%%%%%%%%%%%%%%%%%%%%%%%%%%%%%%%%%%%%%%%%%

\newpage

\section{Additional routines}

\subsection{glp\_check\_kkt --- check feasibility/optimality
conditions}

\synopsis

{\parskip=0pt
\tt void glp\_check\_kkt(glp\_prob *P, int sol, int cond,
double *ae\_max, int *ae\_ind,

\hspace{105pt}double *re\_max, int *re\_ind);}

\description

The routine \verb|glp_check_kkt| allows to check
feasibility/optimality conditions for the current solution stored in
the specified problem object. (For basic and interior-point solutions
these conditions are known as {\it Karush--Kuhn--Tucker optimality
conditions}.)

The parameter \verb|sol| specifies which solution should be checked:

\verb|GLP_SOL| --- basic solution;

\verb|GLP_IPT| --- interior-point solution;

\verb|GLP_MIP| --- mixed integer solution.

The parameter \verb|cond| specifies which condition should be checked:

\verb|GLP_KKT_PE| --- check primal equality constraints (KKT.PE);

\verb|GLP_KKT_PB| --- check primal bound constraints (KKT.PB);

\verb|GLP_KKT_DE| --- check dual equality constraints (KKT.DE). This
conditions can be checked only for basic or interior-point solution;

\verb|GLP_KKT_DB| --- check dual bound constraints (KKT.DB). This
conditions can be checked only for basic or interior-point solution.

Detailed explanations of these conditions are given below in paragraph
``Background''.

On exit the routine stores the following information to locations
specified by parameters \verb|ae_max|, \verb|ae_ind|, \verb|re_max|,
and \verb|re_ind| (if some parameter is a null pointer, corresponding
information is not stored):

\verb|ae_max| --- largest absolute error;

\verb|ae_ind| --- number of row (KKT.PE), column (KKT.DE), or variable
(KKT.PB, KKT.DB) with the largest absolute error;

\verb|re_max| --- largest relative error;

\verb|re_ind| --- number of row (KKT.PE), column (KKT.DE), or variable
(KKT.PB, KKT.DB) with the largest relative error.

Row (auxiliary variable) numbers are in the range 1 to $m$, where $m$
is the number of rows in the problem object. Column (structural
variable) numbers are in the range 1 to $n$, where $n$ is the number
of columns in the problem object. Variable numbers are in the range
1 to $m+n$, where variables with numbers 1 to $m$ correspond to rows,
and variables with numbers $m+1$ to $m+n$ correspond to columns. If
the error reported is exact zero, corresponding row, column or variable
number is set to zero.

\newpage

\para{Background}

\def\arraystretch{1.5}

The first condition checked by the routine is the following:
$$x_R - A x_S = 0, \eqno{\rm (KKT.PE)}$$
where $x_R$ is the subvector of auxiliary variables (rows), $x_S$ is
the subvector of structural variables (columns), $A$ is the constraint
matrix. This condition expresses the requirement that all primal
variables should satisfy to the system of equality constraints of the
original LP problem. In case of exact arithmetic this condition would
be satisfied for any basic solution; however, in case of inexact
(floating-point) arithmetic, this condition shows how accurate the
primal solution is, that depends on accuracy of a representation of the
basis matrix used by the simplex method, or on accuracy provided by the
interior-point method.

To check the condition (KKT.PE) the routine computes the vector of
residuals:
$$g = x_R - A x_S,$$
and determines component of this vector that correspond to largest
absolute and relative errors:
$${\tt ae\_max}=\max_{1\leq i\leq m}|g_i|,$$
$${\tt re\_max}=\max_{1\leq i\leq m}\frac{|g_i|}{1+|(x_R)_i|}.$$

The second condition checked by the routine is the following:
$$l_k \leq x_k \leq u_k {\rm \ \ \ for\ all}\ k=1,\dots,m+n,
\eqno{\rm (KKT.PB)}$$
where $x_k$ is auxiliary ($1\leq k\leq m$) or structural
($m+1\leq k\leq m+n$) variable, $l_k$ and $u_k$ are, respectively,
lower and upper bounds of the variable $x_k$ (including cases of
infinite bounds). This condition expresses the requirement that all
primal variables shoudl satisfy to bound constraints of the original
LP problem. In case of basic solution all non-basic variables are
placed on their active bounds, so actually the condition (KKT.PB) needs
to be checked for basic variables only. If the primal solution has
sufficient accuracy, this condition shows its primal feasibility.

To check the condition (KKT.PB) the routine computes a vector of
residuals:
$$
h_k = \left\{
\begin{array}{ll}
0,         & {\rm if}\ l_k \leq x_k \leq u_k \\
x_k - l_k, & {\rm if}\ x_k < l_k \\
x_k - u_k, & {\rm if}\ x_k > u_k \\
\end{array}
\right.
$$
for all $k=1,\dots,m+n$, and determines components of this vector that
correspond to largest absolute and relative errors:
$${\tt ae\_max}=\max_{1\leq k \leq m+n}|h_k|,$$
$${\tt re\_max}=\max_{1\leq k \leq m+n}\frac{|h_k|}{1+|x_k|}.$$

\newpage

The third condition checked by the routine is:
$${\rm grad}\;Z = c = (\tilde{A})^T \pi + d,$$
where $Z$ is the objective function, $c$ is the vector of objective
coefficients, $(\tilde{A})^T$ is a matrix transposed to the expanded
constraint matrix $\tilde{A} = (I|-A)$, $\pi$ is a vector of Lagrange
multipliers that correspond to equality constraints of the original LP
problem, $d$ is a vector of Lagrange multipliers that correspond to
bound constraints for all (auxiliary and structural) variables of the
original LP problem. Geometrically the third condition expresses the
requirement that the gradient of the objective function should belong
to the orthogonal complement of a linear subspace defined by the
equality and active bound constraints, i.e. that the gradient is
a linear combination of normals to the constraint hyperplanes, where
Lagrange multipliers $\pi$ and $d$ are coefficients of that linear
combination.

To eliminate the vector $\pi$ rewrite the third condition as:
$$
\left(\begin{array}{@{}c@{}}I \\ -A^T\end{array}\right) \pi =
\left(\begin{array}{@{}c@{}}d_R \\ d_S\end{array}\right) +
\left(\begin{array}{@{}c@{}}c_R \\ c_S\end{array}\right),
$$
or, equivalently,
$$
\left\{
\begin{array}{r@{}c@{}c}
\pi + d_R&\ =\ &c_R, \\
-A^T\pi + d_S&\ =\ &c_S. \\
\end{array}
\right.
$$

Then substituting the vector $\pi$ from the first equation into the
second we finally have:
$$A^T (d_R - c_R) + (d_S - c_S) = 0, \eqno{\rm(KKT.DE)}$$
where $d_R$ is the subvector of reduced costs of auxiliary variables
(rows), $d_S$ is the subvector of reduced costs of structural variables
(columns), $c_R$ and $c_S$ are subvectors of objective coefficients at,
respectively, auxiliary and structural variables, $A^T$ is a matrix
transposed to the constraint matrix of the original LP problem. In case
of exact arithmetic this condition would be satisfied for any basic
solution; however, in case of inexact (floating-point) arithmetic, this
condition shows how accurate the dual solution is, that depends on
accuracy of a representation of the basis matrix used by the simplex
method, or on accuracy provided by the interior-point method.

To check the condition (KKT.DE) the routine computes a vector of
residuals:
$$u = A^T (d_R - c_R) + (d_S - c_S),$$
and determines components of this vector that correspond to largest
absolute and relative errors:
$${\tt ae\_max}=\max_{1\leq j\leq n}|u_j|,$$
$${\tt re\_max}=\max_{1\leq j\leq n}\frac{|u_j|}{1+|(d_S)_j-(c_S)_j|}.$$

\newpage

The fourth condition checked by the routine is the following:
$$
\left\{
\begin{array}{l@{\ }r@{\ }c@{\ }c@{\ }c@{\ }l@{\ }c@{\ }c@{\ }c@{\ }l}
{\rm if} & -\infty & < & x_k & < & +\infty,
& {\rm then} & d_k & = & 0 \\
{\rm if} & l_k     & \leq & x_k & < & +\infty,
& {\rm then} & d_k & \geq & 0\ {\rm(minimization)} \\
&&&&&&       & d_k & \leq & 0\ {\rm(maximization)} \\
{\rm if} & -\infty & <    & x_k & \leq & u_k,
& {\rm then} & d_k & \leq & 0\ {\rm(minimization)} \\
&&&&&&       & d_k & \geq & 0\ {\rm(maximization)} \\
{\rm if} & l_k     & \leq & x_k & \leq & u_k,
& {\rm then} & d_k & {\rm is} & {\rm of\ any\ sign} \\
\end{array}\right.\eqno{\rm(KKT.DB)}
$$
for all $k=1,\dots,m+n$, where $d_k$ is a reduced cost (Lagrange
multiplier) of auxiliary ($1\leq k\leq m$) or structural
($m+1\leq k\leq m+n$) variable $x_k$. Geometrically this condition
expresses the requirement that constraints of the original problem must
``hold'' the point preventing its movement along the anti-gradient (in
case of minimization) or the gradient (in case of maximization) of the
objective function. In case of basic solution reduced costs of all
basic variables are placed on their active (zero) bounds, so actually
the condition (KKT.DB) needs to be checked for non-basic variables
only. If the dual solution has sufficient accuracy, this condition
shows the dual feasibility of the solution.

To check the condition (KKT.DB) the routine computes a vector of
residuals:
$$
v_k = \left\{
\begin{array}{ll}
0,         & {\rm if}\ d_k\ {\rm has\ correct\ sign} \\
|d_k|,     & {\rm if}\ d_k\ {\rm has\ wrong\ sign} \\
\end{array}
\right.
$$
for all $k=1,\dots,m+n$, and determines components of this vector that
correspond to largest absolute and relative errors:
$${\tt ae\_max}=\max_{1\leq k\leq m+n}|v_k|,$$
$${\tt re\_max}=\max_{1\leq k\leq m+n}\frac{|v_k|}{1+|d_k - c_k|}.$$

Note that the complete set of Karush-Kuhn-Tucker optimality conditions
also includes the fifth, so called {\it complementary slackness
condition}, which expresses the requirement that at least either
a primal variable $x_k$ or its dual counterpart $d_k$ should be on its
bound for all $k=1,\dots,m+n$. Currently checking this condition is
not implemented yet.

\def\arraystretch{1}

%* eof *%


%* glpk03.tex *%

\chapter{Utility API routines}

\section{Problem data reading/writing routines}

\subsection{glp\_read\_mps --- read problem data in MPS format}

\synopsis

\begin{verbatim}
   int glp_read_mps(glp_prob *P, int fmt, const glp_mpscp *parm,
                    const char *fname);
\end{verbatim}

\description

The routine \verb|glp_read_mps| reads problem data in MPS format from a
text file. (The MPS format is described in Appendix \ref{champs}, page
\pageref{champs}.)

The parameter \verb|fmt| specifies the MPS format version as follows:

\verb|GLP_MPS_DECK| --- fixed (ancient) MPS format;

\verb|GLP_MPS_FILE| --- free (modern) MPS format.

The parameter \verb|parm| is reserved for use in the future and should
be specified as \verb|NULL|.

The character string \verb|fname| specifies a name of the text file to
be read in. (If the file name ends with suffix `\verb|.gz|', the file
is assumed to be compressed, in which case the routine
\verb|glp_read_mps| decompresses it ``on the fly''.)

Note that before reading data the current content of the problem object
is completely erased with the routine \verb|glp_erase_prob|.

\returns

If the operation was successful, the routine \verb|glp_read_mps|
returns zero. Otherwise, it prints an error message and returns
non-zero.

\newpage

\subsection{glp\_write\_mps --- write problem data in MPS format}

\synopsis

\begin{verbatim}
   int glp_write_mps(glp_prob *P, int fmt, const glp_mpscp *parm,
                     const char *fname);
\end{verbatim}

\description

The routine \verb|glp_write_mps| writes problem data in MPS format to
a text file. (The MPS format is described in Appendix \ref{champs},
page \pageref{champs}.)

The parameter \verb|fmt| specifies the MPS format version as follows:

\verb|GLP_MPS_DECK| --- fixed (ancient) MPS format;

\verb|GLP_MPS_FILE| --- free (modern) MPS format.

The parameter \verb|parm| is reserved for use in the future and should
be specified as \verb|NULL|.

The character string \verb|fname| specifies a name of the text file to
be written out. (If the file name ends with suffix `\verb|.gz|', the
file is assumed to be compressed, in which case the routine
\verb|glp_write_mps| performs automatic compression on writing it.)

\returns

If the operation was successful, the routine \verb|glp_write_mps|
returns zero. Otherwise, it prints an error message and returns
non-zero.

\subsection{glp\_read\_lp --- read problem data in CPLEX LP format}

\synopsis

{\tt int glp\_read\_lp(glp\_prob *P, const glp\_cpxcp *parm,
const char *fname);}

\description

The routine \verb|glp_read_lp| reads problem data in CPLEX LP format
from a text file. (The CPLEX LP format is described in Appendix
\ref{chacplex}, page \pageref{chacplex}.)

The parameter \verb|parm| is reserved for use in the future and should
be specified as \verb|NULL|.

The character string \verb|fname| specifies a name of the text file to
be read in. (If the file name ends with suffix `\verb|.gz|', the file
is assumed to be compressed, in which case the routine
\verb|glp_read_lp| decompresses it ``on the fly''.)

Note that before reading data the current content of the problem object
is completely erased with the routine \verb|glp_erase_prob|.

\returns

If the operation was successful, the routine \verb|glp_read_lp| returns
zero. Otherwise, it prints an error message and returns non-zero.

\newpage

\subsection{glp\_write\_lp --- write problem data in CPLEX LP format}

\synopsis

{\tt int glp\_write\_lp(glp\_prob *P, const glp\_cpxcp *parm,
const char *fname);}

\description

The routine \verb|glp_write_lp| writes problem data in CPLEX LP format
to a text file. (The CPLEX LP format is described in Appendix
\ref{chacplex}, page \pageref{chacplex}.)

The parameter \verb|parm| is reserved for use in the future and should
be specified as \verb|NULL|.

The character string \verb|fname| specifies a name of the text file to
be written out. (If the file name ends with suffix `\verb|.gz|', the
file is assumed to be compressed, in which case the routine
\verb|glp_write_lp| performs automatic compression on writing it.)

\returns

If the operation was successful, the routine \verb|glp_write_lp|
returns zero. Otherwise, it prints an error message and returns
non-zero.

\subsection{glp\_read\_prob --- read problem data in GLPK format}

\synopsis

\begin{verbatim}
   int glp_read_prob(glp_prob *P, int flags, const char *fname);
\end{verbatim}

\description

The routine \verb|glp_read_prob| reads problem data in the GLPK LP/MIP
format from a text file. (For description of the GLPK LP/MIP format see
below.)

The parameter \verb|flags| is reserved for use in the future and should
be specified as zero.

The character string \verb|fname| specifies a name of the text file to
be read in. (If the file name ends with suffix `\verb|.gz|', the file
is assumed to be compressed, in which case the routine
\verb|glp_read_prob| decompresses it ``on the fly''.)

Note that before reading data the current content of the problem object
is completely erased with the routine \verb|glp_erase_prob|.

\returns

If the operation was successful, the routine \verb|glp_read_prob|
returns zero. Otherwise, it prints an error message and returns
non-zero.

%\newpage

\para{GLPK LP/MIP format}

The GLPK LP/MIP format is a DIMACS-like format.\footnote{The DIMACS
formats were developed by the Center for Discrete Mathematics and
Theoretical Computer Science (DIMACS) to facilitate exchange of problem
data. For details see: {\tt <http://dimacs.rutgers.edu/Challenges/>}. }
The file in this format is a plain ASCII text file containing lines of
several types described below. A line is terminated with the
end-of-line character. Fields in each line are separated by at least
one blank space. Each line begins with a one-character designator to
identify the line type.

\newpage

The first line of the data file must be the problem line (except
optional comment lines, which may precede the problem line). The last
line of the data file must be the end line. Other lines may follow in
arbitrary order, however, duplicate lines are not allowed.

\para{Comment lines.} Comment lines give human-readable
information about the data file and are ignored by GLPK routines.
Comment lines can appear anywhere in the data file. Each comment line
begins with the lower-case character \verb|c|.

\begin{verbatim}
   c This is an example of comment line
\end{verbatim}

\para{Problem line.} There must be exactly one problem line in the
data file. This line must appear before any other lines except comment
lines and has the following format:

\begin{verbatim}
   p CLASS DIR ROWS COLS NONZ
\end{verbatim}

The lower-case letter \verb|p| specifies that this is the problem line.

The \verb|CLASS| field defines the problem class and can contain either
the keyword \verb|lp| (that means linear programming problem) or
\verb|mip| (that means mixed integer programming problem).

The \verb|DIR| field defines the optimization direction (that is, the
objective function sense) and can contain either the keyword \verb|min|
(that means minimization) or \verb|max| (that means maximization).

The \verb|ROWS|, \verb|COLS|, and \verb|NONZ| fields contain
non-negative integer values specifying, respectively, the number of
rows (constraints), columns (variables), and non-zero constraint
coefficients in the problem instance. Note that \verb|NONZ| value does
not account objective coefficients.

\para{Row descriptors.} There must be at most one row descriptor line
in the data file for each row (constraint). This line has one of the
following formats:

\begin{verbatim}
   i ROW f
   i ROW l RHS
   i ROW u RHS
   i ROW d RHS1 RHS2
   i ROW s RHS
\end{verbatim}

The lower-case letter \verb|i| specifies that this is the row
descriptor line.

The \verb|ROW| field specifies the row ordinal number, an integer
between 1 and $m$, where $m$ is the number of rows in the problem
instance.

The next lower-case letter specifies the row type as follows:

\verb|f| --- free (unbounded) row: $-\infty<\sum a_jx_j<+\infty$;

\verb|l| --- inequality constraint of `$\geq$' type:
$\sum a_jx_j\geq b$;

\verb|u| --- inequality constraint of `$\leq$' type:
$\sum a_jx_j\leq b$;

\verb|d| --- double-sided inequality constraint:
$b_1\leq\sum a_jx_j\leq b_2$;

\verb|s| --- equality constraint: $\sum a_jx_j=b$.

The \verb|RHS| field contains a floaing-point value specifying the
row right-hand side. The \verb|RHS1| and \verb|RHS2| fields contain
floating-point values specifying, respectively, the lower and upper
right-hand sides for the double-sided row.

If for some row its descriptor line does not appear in the data file,
by default that row is assumed to be an equality constraint with zero
right-hand side.

\para{Column descriptors.} There must be at most one column descriptor
line in the data file for each column (variable). This line has one of
the following formats depending on the problem class specified in the
problem line:

\begin{tabular}{@{}l@{\hspace*{40pt}}l}
LP class & MIP class \\
\hline
\verb|j COL f|           & \verb|j COL KIND f|           \\
\verb|j COL l BND|       & \verb|j COL KIND l BND|       \\
\verb|j COL u BND|       & \verb|j COL KIND u BND|       \\
\verb|j COL d BND1 BND2| & \verb|j COL KIND d BND1 BND2| \\
\verb|j COL s BND|       & \verb|j COL KIND s BND|       \\
\end{tabular}

The lower-case letter \verb|j| specifies that this is the column
descriptor line.

The \verb|COL| field specifies the column ordinal number, an integer
between 1 and $n$, where $n$ is the number of columns in the problem
instance.

The \verb|KIND| field is used only for MIP problems and specifies the
column kind as follows:

\verb|c| --- continuous column;

\verb|i| --- integer column;

\verb|b| --- binary column (in this case all remaining fields must be
omitted).

The next lower-case letter specifies the column type as follows:

\verb|f| --- free (unbounded) column: $-\infty<x<+\infty$;

\verb|l| --- column with lower bound: $x\geq l$;

\verb|u| --- column with upper bound: $x\leq u$;

\verb|d| --- double-bounded column: $l\leq x\leq u$;

\verb|s| --- fixed column: $x=s$.

The \verb|BND| field contains a floating-point value that specifies the
column bound. The \verb|BND1| and \verb|BND2| fields contain
floating-point values specifying, respectively, the lower and upper
bounds for the double-bounded column.

If for some column its descriptor line does not appear in the file, by
default that column is assumed to be non-negative (in case of LP class)
or binary (in case of MIP class).

\para{Coefficient descriptors.} There must be exactly one coefficient
descriptor line in the data file for each non-zero objective or
constraint coefficient. This line has the following format:

\begin{verbatim}
   a ROW COL VAL
\end{verbatim}

The lower-case letter \verb|a| specifies that this is the coefficient
descriptor line.

For objective coefficients the \verb|ROW| field must contain 0. For
constraint coefficients the \verb|ROW| field specifies the row ordinal
number, an integer between 1 and $m$, where $m$ is the number of rows
in the problem instance.

The \verb|COL| field specifies the column ordinal number, an integer
between 1 and $n$, where $n$ is the number of columns in the problem
instance.

If both the \verb|ROW| and \verb|COL| fields contain 0, the line
specifies the constant term (``shift'') of the objective function
rather than objective coefficient.

The \verb|VAL| field contains a floating-point coefficient value (it is
allowed to specify zero value in this field).

The number of constraint coefficient descriptor lines must be exactly
the same as specified in the field \verb|NONZ| of the problem line.

\para{Symbolic name descriptors.} There must be at most one symbolic
name descriptor line for the problem instance, objective function, each
row (constraint), and each column (variable). This line has one of the
following formats:

\begin{verbatim}
   n p NAME
   n z NAME
   n i ROW NAME
   n j COL NAME
\end{verbatim}

The lower-case letter \verb|n| specifies that this is the symbolic name
descriptor line.

The next lower-case letter specifies which object should be assigned a
symbolic name:

\verb|p| --- problem instance;

\verb|z| --- objective function;

\verb|i| --- row (constraint);

\verb|j| --- column (variable).

The \verb|ROW| field specifies the row ordinal number, an integer
between 1 and $m$, where $m$ is the number of rows in the problem
instance.

The \verb|COL| field specifies the column ordinal number, an integer
between 1 and $n$, where $n$ is the number of columns in the problem
instance.

The \verb|NAME| field contains the symbolic name, a sequence from 1 to
255 arbitrary graphic ASCII characters, assigned to corresponding
object.

\para{End line.} There must be exactly one end line in the data file.
This line must appear last in the file and has the following format:

\begin{verbatim}
   e
\end{verbatim}

The lower-case letter \verb|e| specifies that this is the end line.
Anything that follows the end line is ignored by GLPK routines.

\newpage

\para{Example of data file in GLPK LP/MIP format}

The following example of a data file in GLPK LP/MIP format specifies
the same LP problem as in Subsection ``Example of MPS file''.

\bigskip

\begin{center}
\footnotesize\tt
\begin{tabular}{l@{\hspace*{50pt}}}
p lp min 8 7 48   \\
n p PLAN          \\
n z VALUE         \\
i 1 f             \\
n i 1 VALUE       \\
i 2 s 2000        \\
n i 2 YIELD       \\
i 3 u 60          \\
n i 3 FE          \\
i 4 u 100         \\
n i 4 CU          \\
i 5 u 40          \\
n i 5 MN          \\
i 6 u 30          \\
n i 6 MG          \\
i 7 l 1500        \\
n i 7 AL          \\
i 8 d 250 300     \\
n i 8 SI          \\
j 1 d 0 200       \\
n j 1 BIN1        \\
j 2 d 0 2500      \\
n j 2 BIN2        \\
j 3 d 400 800     \\
n j 3 BIN3        \\
j 4 d 100 700     \\
n j 4 BIN4        \\
j 5 d 0 1500      \\
n j 5 BIN5        \\
n j 6 ALUM        \\
n j 7 SILICON     \\
a 0 1 0.03        \\
a 0 2 0.08        \\
a 0 3 0.17        \\
a 0 4 0.12        \\
a 0 5 0.15        \\
a 0 6 0.21        \\
a 0 7 0.38        \\
a 1 1 0.03        \\
a 1 2 0.08        \\
a 1 3 0.17        \\
a 1 4 0.12        \\
a 1 5 0.15        \\
a 1 6 0.21        \\
\end{tabular}
\begin{tabular}{|@{\hspace*{80pt}}l}
a 1 7 0.38        \\
a 2 1 1           \\
a 2 2 1           \\
a 2 3 1           \\
a 2 4 1           \\
a 2 5 1           \\
a 2 6 1           \\
a 2 7 1           \\
a 3 1 0.15        \\
a 3 2 0.04        \\
a 3 3 0.02        \\
a 3 4 0.04        \\
a 3 5 0.02        \\
a 3 6 0.01        \\
a 3 7 0.03        \\
a 4 1 0.03        \\
a 4 2 0.05        \\
a 4 3 0.08        \\
a 4 4 0.02        \\
a 4 5 0.06        \\
a 4 6 0.01        \\
a 5 1 0.02        \\
a 5 2 0.04        \\
a 5 3 0.01        \\
a 5 4 0.02        \\
a 5 5 0.02        \\
a 6 1 0.02        \\
a 6 2 0.03        \\
a 6 5 0.01        \\
a 7 1 0.7         \\
a 7 2 0.75        \\
a 7 3 0.8         \\
a 7 4 0.75        \\
a 7 5 0.8         \\
a 7 6 0.97        \\
a 8 1 0.02        \\
a 8 2 0.06        \\
a 8 3 0.08        \\
a 8 4 0.12        \\
a 8 5 0.02        \\
a 8 6 0.01        \\
a 8 7 0.97        \\
e o f             \\
\\
\end{tabular}
\end{center}

\newpage

\subsection{glp\_write\_prob --- write problem data in GLPK format}

\synopsis

\begin{verbatim}
int glp_write_prob(glp_prob *P, int flags, const char *fname);
\end{verbatim}

\description

The routine \verb|glp_write_prob| writes problem data in the GLPK
LP/MIP format to a text file. (For description of the GLPK LP/MIP
format see Subsection ``Read problem data in GLPK format''.)

The parameter \verb|flags| is reserved for use in the future and should
be specified as zero.

The character string \verb|fname| specifies a name of the text file to
be written out. (If the file name ends with suffix `\verb|.gz|', the
file is assumed to be compressed, in which case the routine
\verb|glp_write_prob| performs automatic compression on writing it.)

\returns

If the operation was successful, the routine \verb|glp_read_prob|
returns zero. Otherwise, it prints an error message and returns
non-zero.

%%%%%%%%%%%%%%%%%%%%%%%%%%%%%%%%%%%%%%%%%%%%%%%%%%%%%%%%%%%%%%%%%%%%%%%%

\newpage

\section{Routines for processing MathProg models}

\subsection{Introduction}

GLPK supports the {\it GNU MathProg modeling language}.\footnote{The
GNU MathProg modeling language is a subset of the AMPL language. For
its detailed description see the document ``Modeling Language GNU
MathProg: Language Reference'' included in the GLPK distribution.}
As a rule, models written in MathProg are solved with the GLPK LP/MIP
stand-alone solver \verb|glpsol| (see Appendix D) and do not need any
programming with API routines. However, for various reasons the user
may need to process MathProg models directly in his/her application
program, in which case he/she may use API routines described in this
section. These routines provide an interface to the {\it MathProg
translator}, a component of GLPK, which translates MathProg models into
an internal code and then interprets (executes) this code.

The processing of a model written in GNU MathProg includes several
steps, which should be performed in the following order:

%\vspace*{-8pt}

%\begin{enumerate}
\Item{1.}{\it Allocating the workspace.}
The translator allocates the workspace, an internal data structure used
on all subsequent steps.

\Item{2.}{\it Reading model section.} The translator reads model
section and, optionally, data section from a specified text file and
translates them into the internal code. If necessary, on this step data
section may be ignored.

\Item{3.}{\it Reading data section(s).} The translator reads one or
more data sections from specified text file(s) and translates them into
the internal code.

\Item{4.}{\it Generating the model.} The translator executes the
internal code to evaluate the content of the model objects such as sets,
parameters, variables, constraints, and objectives. On this step the
execution is suspended at the solve statement.

\Item{5.}{\it Building the problem object.} The translator obtains all
necessary information from the workspace and builds the standard
problem object (that is, the program object of type \verb|glp_prob|).

\Item{6.}{\it Solving the problem.} On this step the problem object
built on the previous step is passed to a solver, which solves the
problem instance and stores its solution back to the problem object.

\Item{7.}{\it Postsolving the model.} The translator copies the
solution from the problem object to the workspace and then executes the
internal code from the solve statement to the end of the model.
(If model has no solve statement, the translator does nothing on this
step.)

\Item{8.}{\it Freeing the workspace.} The translator frees all the
memory allocated to the workspace.
%\end{enumerate}

%\vspace*{-8pt}

Note that the MathProg translator performs no error correction, so if
any of steps 2 to 7 fails (due to errors in the model), the application
program should terminate processing and go to\linebreak step 8.

\newpage

\para{Example 1}

In this example the program reads model and data sections from input
file \verb|egypt.mod|\footnote{This is an example model included in
the GLPK distribution.} and writes the model to output file
\verb|egypt.mps| in free MPS format (see Appendix B). No solution is
performed.

\bigskip

\begin{small}
\begin{verbatim}
/* mplsamp1.c */

#include <stdio.h>
#include <stdlib.h>
#include <glpk.h>

int main(void)
{     glp_prob *lp;
      glp_tran *tran;
      int ret;
      lp = glp_create_prob();
      tran = glp_mpl_alloc_wksp();
      ret = glp_mpl_read_model(tran, "egypt.mod", 0);
      if (ret != 0)
      {  fprintf(stderr, "Error on translating model\n");
         goto skip;
      }
      ret = glp_mpl_generate(tran, NULL);
      if (ret != 0)
      {  fprintf(stderr, "Error on generating model\n");
         goto skip;
      }
      glp_mpl_build_prob(tran, lp);
      ret = glp_write_mps(lp, GLP_MPS_FILE, NULL, "egypt.mps");
      if (ret != 0)
         fprintf(stderr, "Error on writing MPS file\n");
skip: glp_mpl_free_wksp(tran);
      glp_delete_prob(lp);
      return 0;
}

/* eof */
\end{verbatim}
\end{small}

\newpage

\subsubsection*{Example 2}

In this example the program reads model section from file
\verb|sudoku.mod|\footnote{This is an example model which is included
in the GLPK distribution along with alternative data file
{\tt sudoku.dat}.} ignoring data section in this file, reads alternative
data section from file \verb|sudoku.dat|, solves the problem instance
and passes the solution found back to the model.

\bigskip

\begin{small}
\begin{verbatim}
/* mplsamp2.c */

#include <stdio.h>
#include <stdlib.h>
#include <glpk.h>

int main(void)
{     glp_prob *mip;
      glp_tran *tran;
      int ret;
      mip = glp_create_prob();
      tran = glp_mpl_alloc_wksp();
      ret = glp_mpl_read_model(tran, "sudoku.mod", 1);
      if (ret != 0)
      {  fprintf(stderr, "Error on translating model\n");
         goto skip;
      }
      ret = glp_mpl_read_data(tran, "sudoku.dat");
      if (ret != 0)
      {  fprintf(stderr, "Error on translating data\n");
         goto skip;
      }
      ret = glp_mpl_generate(tran, NULL);
      if (ret != 0)
      {  fprintf(stderr, "Error on generating model\n");
         goto skip;
      }
      glp_mpl_build_prob(tran, mip);
      glp_simplex(mip, NULL);
      glp_intopt(mip, NULL);
      ret = glp_mpl_postsolve(tran, mip, GLP_MIP);
      if (ret != 0)
         fprintf(stderr, "Error on postsolving model\n");
skip: glp_mpl_free_wksp(tran);
      glp_delete_prob(mip);
      return 0;
}

/* eof */
\end{verbatim}
\end{small}

\newpage

\subsection{glp\_mpl\_alloc\_wksp --- allocate the translator
workspace}

\synopsis

\begin{verbatim}
   glp_tran *glp_mpl_alloc_wksp(void);
\end{verbatim}

\description

The routine \verb|glp_mpl_alloc_wksp| allocates the MathProg translator
work\-space. (Note that multiple instances of the workspace may be
allocated, if necessary.)

\returns

The routine returns a pointer to the workspace, which should be used in
all subsequent operations.

\subsection{glp\_mpl\_init\_rand --- initialize pseudo-random number
generator}

\synopsis

\begin{verbatim}
   void glp_mpl_init_rand(glp_tran *tran, int seed);
\end{verbatim}

\description

The routine \verb|glp_mpl_init_rand| initializes a pseudo-random number
generator used by the MathProg translator, where the parameter
\verb|seed| may be any integer number.

A call to the routine \verb|glp_mpl_init_rand| should immediately
follow the call to the routine \verb|glp_mpl_alloc_wksp|. However,
using of this routine is optional. If it is not called, the effect is
the same as if it were called with \verb|seed| equal to 1.

\subsection{glp\_mpl\_read\_model --- read and translate model section}

\synopsis

\begin{verbatim}
   int glp_mpl_read_model(glp_tran *tran, const char *fname, int skip);
\end{verbatim}

\description

The routine \verb|glp_mpl_read_model| reads model section and,
optionally, data section, which may follow the model section, from a
text file, whose name is the character string \verb|fname|, performs
translation of model statements and data blocks, and stores all the
information in the workspace.

The parameter \verb|skip| is a flag. If the input file contains the
data section and this flag is non-zero, the data section is not read as
if there were no data section and a warning message is printed. This
allows reading data section(s) from other file(s).

\returns

If the operation is successful, the routine returns zero. Otherwise
the routine prints an error message and returns non-zero.

\newpage

\subsection{glp\_mpl\_read\_data --- read and translate data section}

\synopsis

\begin{verbatim}
   int glp_mpl_read_data(glp_tran *tran, const char *fname);
\end{verbatim}

\description

The routine \verb|glp_mpl_read_data| reads data section from a text
file, whose name is the character string \verb|fname|, performs
translation of data blocks, and stores the data read in the translator
workspace. If necessary, this routine may be called more than once.

\returns

If the operation is successful, the routine returns zero. Otherwise
the routine prints an error message and returns non-zero.

\subsection{glp\_mpl\_generate --- generate the model}

\synopsis

\begin{verbatim}
   int glp_mpl_generate(glp_tran *tran, const char *fname);
\end{verbatim}

\description

The routine \verb|glp_mpl_generate| generates the model using its
description stored in the translator workspace. This operation means
generating all variables, constraints, and objectives, executing check
and display statements, which precede the solve statement (if it is
presented).

The character string \verb|fname| specifies the name of an output text
file, to which output produced by display statements should be written.
If \verb|fname| is \verb|NULL|, the output is sent to the terminal.

\returns

If the operation is successful, the routine returns zero. Otherwise
the routine prints an error message and returns non-zero.

\vspace*{-6pt}

\subsection{glp\_mpl\_build\_prob --- build problem instance from the
model}

\synopsis

\begin{verbatim}
   void glp_mpl_build_prob(glp_tran *tran, glp_prob *P);
\end{verbatim}

\description

The routine \verb|glp_mpl_build_prob| obtains all necessary information
from the translator work\-space and stores it in the specified problem
object \verb|P|. Note that before building the current content of the
problem object is erased with the routine \verb|glp_erase_prob|.

\newpage

\subsection{glp\_mpl\_postsolve --- postsolve the model}

\synopsis

\begin{verbatim}
   int glp_mpl_postsolve(glp_tran *tran, glp_prob *P, int sol);
\end{verbatim}

\description

The routine \verb|glp_mpl_postsolve| copies the solution from the
specified problem object \verb|prob| to the translator workspace and
then executes all the remaining model statements, which follow the
solve statement.

The parameter \verb|sol| specifies which solution should be copied
from the problem object to the workspace as follows:

\verb|GLP_SOL| --- basic solution;

\verb|GLP_IPT| --- interior-point solution;

\verb|GLP_MIP| --- mixed integer solution.

\returns

If the operation is successful, the routine returns zero. Otherwise
the routine prints an error message and returns non-zero.

\subsection{glp\_mpl\_free\_wksp --- free the translator workspace}

\synopsis

\begin{verbatim}
   void glp_mpl_free_wksp(glp_tran *tran);
\end{verbatim}

\description

The routine \verb|glp_mpl_free_wksp| frees all the memory allocated to
the translator workspace. It also frees all other resources, which are
still used by the translator.

%%%%%%%%%%%%%%%%%%%%%%%%%%%%%%%%%%%%%%%%%%%%%%%%%%%%%%%%%%%%%%%%%%%%%%%%

\newpage

\section{Problem solution reading/writing routines}

\subsection{glp\_print\_sol --- write basic solution in printable
format}

\synopsis

\begin{verbatim}
   int glp_print_sol(glp_prob *P, const char *fname);
\end{verbatim}

\description

The routine \verb|glp_print_sol writes| the current basic solution to
an LP problem, which is specified by the pointer \verb|P|, to a text
file, whose name is the character string \verb|fname|, in printable
format.

Information reported by the routine \verb|glp_print_sol| is intended
mainly for visual analysis.

\returns

If no errors occurred, the routine returns zero. Otherwise the routine
prints an error message and returns non-zero.

\subsection{glp\_read\_sol --- read basic solution in GLPK format}

\synopsis

\begin{verbatim}
   int glp_read_sol(glp_prob *P, const char *fname);
\end{verbatim}

\description

The routine \verb|glp_read_sol| reads basic solution from a text file
in the GLPK format. (For description of the format see below.)

The character string \verb|fname| specifies the name of the text file
to be read in. (If the file name ends with suffix `\verb|.gz|', the
file is assumed to be compressed, in which case the routine
\verb|glp_read_sol| decompresses it "on the fly".)

\returns

If the operation was successful, the routine \verb|glp_read_sol|
returns zero. Otherwise, it prints an error message and returns
non-zero.

\para{GLPK basic solution format}

The GLPK basic solution format is a DIMACS-like format.\footnote{The
DIMACS formats were developed by the Center for Discrete Mathematics
and Theoretical Computer Science (DIMACS) to facilitate exchange of
problem data.
For details see: {\tt <http://dimacs.rutgers.edu/Challenges/>}. }
The file in this format is a plain ASCII text file containing lines of
several types described below. A line is terminated with the
end-of-line character. Fields in each line are separated by at least
one blank space. Each line begins with a one-character designator to
identify the line type.

The first line of the solution file must be the solution line (except
optional comment lines, which may precede the problem line). The last
line of the data file must be the end line. Other lines may follow in
arbitrary order, however, duplicate lines are not allowed.

\newpage

\para{Comment lines.} Comment lines give human-readable information
about the solution file and are ignored by GLPK routines. Comment lines
can appear anywhere in the data file. Each comment line begins with the
lower-case character \verb|c|.

\begin{verbatim}
   c This is an example of comment line
\end{verbatim}

\para{Solution line.} There must be exactly one solution line in the
solution file. This line must appear before any other lines except
comment lines and has the following format:

\begin{verbatim}
   s bas ROWS COLS PST DST OBJ
\end{verbatim}

The lower-case letter \verb|s| specifies that this is the solution
line.

The three-character solution designator \verb|bas| identifies the file
as containing a basic solution to the LP problem instance.

The \verb|ROWS| and \verb|COLS| fields contain non-negative integer
values that specify the number of rows (constraints) and columns
(variables), resp., in the LP problem instance.

The \verb|PST| and \verb|DST| fields contain lower-case letters that
specify the primal and dual solution status, resp., as follows:

\verb|u| --- solution is undefined;

\verb|f| --- solution is feasible;

\verb|i| --- solution is infeasible;

\verb|n| --- no feasible solution exists.

The \verb|OBJ| field contains a floating-point number that specifies
the objective function value in the basic solution.

\para{Row solution descriptors.} There must be exactly one row solution
descriptor line in the solution file for each row (constraint). This
line has the following format:

\begin{verbatim}
   i ROW ST PRIM DUAL
\end{verbatim}

The lower-case letter \verb|i| specifies that this is the row solution
descriptor line.

The \verb|ROW| field specifies the row ordinal number, an integer
between 1 and $m$, where $m$ is the number of rows in the problem
instance.

The \verb|ST| field contains one of the following lower-case letters
that specifies the row status in the basic solution:\footnote{The row
status is the status of the associated auxiliary variable.}

\verb|b| --- inactive constraint;

\verb|l| --- inequality constraint active on its lower bound;

\verb|u| --- inequality constraint active on its upper bound;

\verb|f| --- active free (unounded) row;

\verb|s| --- active equality constraint.

The \verb|PRIM| field contains a floating-point number that specifies
the row primal value (the value of the corresponding linear form).

The \verb|DUAL| field contains a floating-point number that specifies
the row dual value (the Lagrangian multiplier for active bound).

\para{Column solution descriptors.} There must be exactly one column
solution descriptor line in the solution file for each column
(variable). This line has the following format:

\begin{verbatim}
   j COL ST PRIM DUAL
\end{verbatim}

The lower-case letter \verb|j| specifies that this is the column
solution descriptor line.

The \verb|COL| field specifies the column ordinal number, an integer
between 1 and $n$, where $n$ is the number of columns in the problem
instance.

The \verb|ST| field contains one of the following lower-case letters
that specifies the column status in the basic solution:

\verb|b| --- basic variable;

\verb|l| --- non-basic variable having its lower bound active;

\verb|u| --- non-basic variable having its upper bound active;

\verb|f| --- non-basic free (unounded) variable;

\verb|s| --- non-basic fixed variable.

The \verb|PRIM| field contains a floating-point number that specifies
the column primal value.

The \verb|DUAL| field contains a floating-point number that specifies
the column dual value (the Lagrangian multiplier for active bound).

\para{End line.} There must be exactly one end line in the solution
file. This line must appear last in the file and has the following
format:

\begin{verbatim}
   e
\end{verbatim}

The lower-case letter \verb|e| specifies that this is the end line.
Anything that follows the end line is ignored by GLPK routines.

\para{Example of solution file in GLPK basic solution format}

The following example of a solution file in GLPK basic solution format
specifies the optimal basic solution to the LP problem instance from
Subsection ``Example of MPS file''.

\bigskip

\begin{center}
\footnotesize\tt
\begin{tabular}{l@{\hspace*{50pt}}}
s bas 7 7 f f 296.216606498195   \\
i 1 s 2000 -0.0135956678700369   \\
i 2 u 60 -2.56823104693141       \\
i 3 b 83.9675090252707 0         \\
i 4 u 40 -0.544404332129962      \\
i 5 b 19.9602888086643 0         \\
i 6 l 1500 0.251985559566788     \\
i 7 l 250 0.48519855595668       \\
\end{tabular}
\begin{tabular}{|@{\hspace*{50pt}}l}
j 1 l 0 0.253624548736462        \\
j 2 b 665.342960288809 0         \\
j 3 b 490.252707581226 0         \\
j 4 b 424.187725631769 0         \\
j 5 l 0 0.0145559566787004       \\
j 6 b 299.638989169676 0         \\
j 7 b 120.57761732852 0          \\
e o f                            \\
\end{tabular}
\end{center}

\newpage

\subsection{glp\_write\_sol --- write basic solution in GLPK format}

\synopsis

\begin{verbatim}
   int glp_write_sol(glp_prob *P, const char *fname);
\end{verbatim}

\description

The routine \verb|glp_write_sol| writes the current basic solution to
a text file in the GLPK format. (For description of the GLPK basic
solution format see Subsection ``Read basic solution in GLPK format.'')

The character string \verb|fname| specifies the name of the text file
to be written. (If the file name ends with suffix `\verb|.gz|', the
routine \verb|glp_write_sol| compresses it "on the fly".)

\returns

If the operation was successful, the routine \verb|glp_write_sol|
returns zero. Otherwise, it prints an error message and returns
non-zero.

\subsection{glp\_print\_ipt --- write interior-point solution in
printable format}

\synopsis

\begin{verbatim}
   int glp_print_ipt(glp_prob *P, const char *fname);
\end{verbatim}

\description

The routine \verb|glp_print_ipt| writes the current interior point
solution to an LP problem, which the parameter \verb|P| points to, to
a text file, whose name is the character string \verb|fname|, in
printable format.

Information reported by the routine \verb|glp_print_ipt| is intended
mainly for visual analysis.

\returns

If no errors occurred, the routine returns zero. Otherwise the routine
prints an error message and returns non-zero.

\subsection{glp\_read\_ipt --- read interior-point solution in GLPK
format}

\synopsis

\begin{verbatim}
   int glp_read_ipt(glp_prob *P, const char *fname);
\end{verbatim}

\description

The routine \verb|glp_read_ipt| reads interior-point solution from
a text file in the GLPK format. (For description of the format see
below.)

The character string \verb|fname| specifies the name of the text file
to be read in. (If the file name ends with suffix `\verb|.gz|', the
file is assumed to be compressed, in which case the routine
\verb|glp_read_ipt| decompresses it "on the fly".)

%\newpage

\returns

If the operation was successful, the routine \verb|glp_read_ipt|
returns zero. Otherwise, it prints an error message and returns
non-zero.

\para{GLPK interior-point solution format}

The GLPK interior-point solution format is a DIMACS-like
format.\footnote{The DIMACS formats were developed by the Center for
Discrete Mathematics and Theoretical Computer Science (DIMACS) to
facilitate exchange of problem data. For details see:
{\tt <http://dimacs.rutgers.edu/Challenges/>}. }
The file in this format is a plain ASCII text file containing lines of
several types described below. A line is terminated with the
end-of-line character. Fields in each line are separated by at least
one blank space. Each line begins with a one-character designator to
identify the line type.

The first line of the solution file must be the solution line (except
optional comment lines, which may precede the problem line). The last
line of the data file must be the end line. Other lines may follow in
arbitrary order, however, duplicate lines are not allowed.

\para{Comment lines.} Comment lines give human-readable information
about the solution file and are ignored by GLPK routines. Comment lines
can appear anywhere in the data file. Each comment line begins with the
lower-case character \verb|c|.

\begin{verbatim}
   c This is an example of comment line
\end{verbatim}

\para{Solution line.} There must be exactly one solution line in the
solution file. This line must appear before any other lines except
comment lines and has the following format:

\begin{verbatim}
   s ipt ROWS COLS SST OBJ
\end{verbatim}

The lower-case letter \verb|s| specifies that this is the solution
line.

The three-character solution designator \verb|ipt| identifies the file
as containing an interior-point solution to the LP problem instance.

The \verb|ROWS| and \verb|COLS| fields contain non-negative integer
values that specify the number of rows (constraints) and columns
(variables), resp., in the LP problem instance.

The \verb|SST| field contains one of the following lower-case letters
that specifies the solution status:

\verb|o| --- solution is optimal;

\verb|i| --- solution is infeasible;

\verb|n| --- no feasible solution exists;

\verb|u| --- solution is undefined.

The \verb|OBJ| field contains a floating-point number that specifies
the objective function value in the interior-point solution.

\para{Row solution descriptors.} There must be exactly one row solution
descriptor line in the solution file for each row (constraint). This
line has the following format:

\begin{verbatim}
   i ROW PRIM DUAL
\end{verbatim}

The lower-case letter \verb|i| specifies that this is the row solution
descriptor line.

%\newpage

The \verb|ROW| field specifies the row ordinal number, an integer
between 1 and $m$, where $m$ is the number of rows in the problem
instance.

The \verb|PRIM| field contains a floating-point number that specifies
the row primal value (the value of the corresponding linear form).

The \verb|DUAL| field contains a floating-point number that specifies
the row dual value (the Lagrangian multiplier for active bound).

\para{Column solution descriptors.} There must be exactly one column
solution descriptor line in the solution file for each column
(variable). This line has the following format:

\begin{verbatim}
   j COL PRIM DUAL
\end{verbatim}

The lower-case letter \verb|j| specifies that this is the column
solution descriptor line.

The \verb|COL| field specifies the column ordinal number, an integer
between 1 and $n$, where $n$ is the number of columns in the problem
instance.

The \verb|PRIM| field contains a floating-point number that specifies
the column primal value.

The \verb|DUAL| field contains a floating-point number that specifies
the column dual value (the Lagrangian multiplier for active bound).

\para{End line.} There must be exactly one end line in the solution
file. This line must appear last in the file and has the following
format:

\begin{verbatim}
   e
\end{verbatim}

The lower-case letter \verb|e| specifies that this is the end line.
Anything that follows the end line is ignored by GLPK routines.

\para{Example of solution file in GLPK interior-point solution format}

The following example of a solution file in GLPK interior-point
solution format specifies the optimal interior-point solution to the LP
problem instance from Subsection ``Example of MPS file''.

\bigskip

\begin{center}
\footnotesize\tt
\begin{tabular}{l@{\hspace*{10pt}}}
s ipt 7 7 o 296.216606851403                 \\
i 1 2000.00000290369 -0.0135956757623443     \\
i 2 60.0000001302903 -2.568231024875         \\
i 3 83.9675094251819 -8.85591445202383e-10   \\
i 4 39.9999999985064 -0.544404310443766      \\
i 5 19.9602886941262 -2.24817803513554e-09   \\
i 6 1500.00000199013 0.251985567257828       \\
i 7 250.000000244896 0.48519856304979        \\
\end{tabular}
\begin{tabular}{|@{\hspace*{10pt}}l}
j 1 3.3482079213784e-07 0.253624547432525    \\
j 2 665.342955760768 6.04613825351601e-11    \\
j 3 490.25271366802 5.8488360240978e-10      \\
j 4 424.187729774275 -2.54144550490434e-11   \\
j 5 1.46067738492801e-06 0.0145559574770786  \\
j 6 299.638985053112 1.49359074902927e-10    \\
j 7 120.577616852015 3.50297708781545e-10    \\
e o f
\end{tabular}
\end{center}

\subsection{glp\_write\_ipt --- write interior-point solution in GLPK
format}

\synopsis

\begin{verbatim}
   int glp_write_ipt(glp_prob *P, const char *fname);
\end{verbatim}

\description

The routine \verb|glp_write_ipt| writes the current interior-point
solution to a text file in the GLPK format. (For description of the
GLPK interior-point solution format see Subsection ``Read
interior-point solution in GLPK format.'')

The character string \verb|fname| specifies the name of the text file
to be written. (If the file name ends with suffix `\verb|.gz|', the
routine \verb|glp_write_ipt| compresses it "on the fly".)

\returns

If the operation was successful, the routine \verb|glp_write_ipt|
returns zero. Otherwise, it prints an error message and returns
non-zero.

\subsection{glp\_print\_mip --- write MIP solution in printable format}

\synopsis

\begin{verbatim}
   int glp_print_mip(glp_prob *P, const char *fname);
\end{verbatim}

\description

The routine \verb|glp_print_mip| writes the current solution to a MIP
problem, which is specified by the pointer \verb|P|, to a text file,
whose name is the character string \verb|fname|, in printable format.

Information reported by the routine \verb|glp_print_mip| is intended
mainly for visual analysis.

\returns

If no errors occurred, the routine returns zero. Otherwise the routine
prints an error message and returns non-zero.

\subsection{glp\_read\_mip --- read MIP solution in GLPK format}

\synopsis

\begin{verbatim}
   int glp_read_mip(glp_prob *P, const char *fname);
\end{verbatim}

\description

The routine \verb|glp_read_mip| reads MIP solution from a text file in
the GLPK format. (For description of the format see below.)

The character string \verb|fname| specifies the name of the text file
to be read in. (If the file name ends with suffix `\verb|.gz|', the
file is assumed to be compressed, in which case the routine
\verb|glp_read_mip| decompresses it "on the fly".)

\returns

If the operation was successful, the routine \verb|glp_read_mip|
returns zero. Otherwise, it prints an error message and returns
non-zero.

\para{GLPK MIP solution format}

The GLPK MIP solution format is a DIMACS-like format.\footnote{The
DIMACS formats were developed by the Center for Discrete Mathematics
and Theoretical Computer Science (DIMACS) to facilitate exchange of
problem data. For details see:
{\tt <http://dimacs.rutgers.edu/Challenges/>}. }
The file in this format is a plain ASCII text file containing lines of
several types described below. A line is terminated with the
end-of-line character. Fields in each line are separated by at least
one blank space. Each line begins with a one-character designator to
identify the line type.

The first line of the solution file must be the solution line (except
optional comment lines, which may precede the problem line). The last
line of the data file must be the end line. Other lines may follow in
arbitrary order, however, duplicate lines are not allowed.

\para{Comment lines.} Comment lines give human-readable information
about the solution file and are ignored by GLPK routines. Comment lines
can appear anywhere in the data file. Each comment line begins with the
lower-case character \verb|c|.

\begin{verbatim}
   c This is an example of comment line
\end{verbatim}

\para{Solution line.} There must be exactly one solution line in the
solution file. This line must appear before any other lines except
comment lines and has the following format:

\begin{verbatim}
   s mip ROWS COLS SST OBJ
\end{verbatim}

The lower-case letter \verb|s| specifies that this is the solution
line.

The three-character solution designator \verb|mip| identifies the file
as containing a solution to the MIP problem instance.

The \verb|ROWS| and \verb|COLS| fields contain non-negative integer
values that specify the number of rows (constraints) and columns
(variables), resp., in the LP problem instance.

The \verb|SST| field contains one of the following lower-case letters
that specifies the solution status:

\verb|o| --- solution is integer optimal;

\verb|f| --- solution is integer feasible (non-optimal);

\verb|n| --- no integer feasible solution exists;

\verb|u| --- solution is undefined.

The \verb|OBJ| field contains a floating-point number that specifies
the objective function value in the MIP solution.

\para{Row solution descriptors.} There must be exactly one row solution
descriptor line in the solution file for each row (constraint). This
line has the following format:

\begin{verbatim}
   i ROW VAL
\end{verbatim}

The lower-case letter \verb|i| specifies that this is the row solution
descriptor line.

The \verb|ROW| field specifies the row ordinal number, an integer
between 1 and $m$, where $m$ is the number of rows in the problem
instance.

The \verb|VAL| field contains a floating-point number that specifies
the row value (the value of the corresponding linear form).

\para{Column solution descriptors.} There must be exactly one column
solution descriptor line in the solution file for each column
(variable). This line has the following format:

\begin{verbatim}
   j COL VAL
\end{verbatim}

The lower-case letter \verb|j| specifies that this is the column
solution descriptor line.

The \verb|COL| field specifies the column ordinal number, an integer
between 1 and $n$, where $n$ is the number of columns in the problem
instance.

The \verb|VAL| field contains a floating-point number that specifies
the column value.

\para{End line.} There must be exactly one end line in the solution
file. This line must appear last in the file and has the following
format:

\begin{verbatim}
   e
\end{verbatim}

The lower-case letter \verb|e| specifies that this is the end line.
Anything that follows the end line is ignored by GLPK routines.

\para{Example of solution file in GLPK MIP solution format}

The following example of a solution file in GLPK MIP solution format
specifies an optimal solution to a MIP problem instance.

\bigskip

\begin{center}
\footnotesize\tt
\begin{tabular}{l@{\hspace*{50pt}}}
s mip 8 8 o 1201500 \\
i 1 60      \\
i 2 8400    \\
i 3 -1200   \\
i 4 0       \\
i 5 9000    \\
i 6 -600    \\
i 7 0       \\
i 8 8000    \\
\end{tabular}
\begin{tabular}{|@{\hspace*{80pt}}l}
j 1 60      \\
j 2 6       \\
j 3 0       \\
j 4 60      \\
j 5 6       \\
j 6 600     \\
j 7 60      \\
j 8 16      \\
e o f       \\
\end{tabular}
\end{center}

\subsection{glp\_write\_mip --- write MIP solution in GLPK format}

\synopsis

\begin{verbatim}
   int glp_write_mip(glp_prob *P, const char *fname);
\end{verbatim}

\description

The routine \verb|glp_write_mip| writes the current MIP solution to
a text file in the GLPK format. (For description of the GLPK MIP
solution format see Subsection ``Read MIP solution in GLPK format.'')

The character string \verb|fname| specifies the name of the text file
to be written. (If the file name ends with suffix `\verb|.gz|', the
routine \verb|glp_write_mip| compresses it "on the fly".)

\returns

If the operation was successful, the routine \verb|glp_write_mip|
returns zero. Otherwise, it prints an error message and returns
non-zero.

%%%%%%%%%%%%%%%%%%%%%%%%%%%%%%%%%%%%%%%%%%%%%%%%%%%%%%%%%%%%%%%%%%%%%%%%

\newpage

\section{Post-optimal analysis routines}

\subsection{glp\_print\_ranges --- print sensitivity analysis report}

\synopsis

{\tt int glp\_print\_ranges(glp\_prob *P, int len, const int list[],
int flags,\\
\hspace*{134pt}const char *fname);}

\description

The routine \verb|glp_print_ranges| performs sensitivity analysis of
current optimal basic solution and writes the analysis report in
human-readable format to a text file, whose name is the character
string {\it fname}. (Detailed description of the report structure is
given below.)

The parameter {\it len} specifies the length of the row/column list.

The array {\it list} specifies ordinal number of rows and columns to be
analyzed. The ordinal numbers should be passed in locations
{\it list}[1], {\it list}[2], \dots, {\it list}[{\it len}]. Ordinal
numbers from 1 to $m$ refer to rows, and ordinal numbers from $m+1$ to
$m+n$ refer to columns, where $m$ and $n$ are, resp., the total number
of rows and columns in the problem object. Rows and columns appear in
the analysis report in the same order as they follow in the array list.

It is allowed to specify $len=0$, in which case the array {\it list} is
not used (so it can be specified as \verb|NULL|), and the routine
performs analysis for all rows and columns of the problem object.

The parameter {\it flags} is reserved for use in the future and must be
specified as zero.

On entry to the routine \verb|glp_print_ranges| the current basic
solution must be optimal and the basis factorization must exist.
The application program can check that with the routine
\verb|glp_bf_exists|, and if the factorization does
not exist, compute it with the routine \verb|glp_factorize|. Note that
if the LP preprocessor is not used, on normal exit from the simplex
solver routine \verb|glp_simplex| the basis factorization always exists.

\returns

If the operation was successful, the routine \verb|glp_print_ranges|
returns zero. Otherwise, it prints an error message and returns
non-zero.

\para{Analysis report example}

An example of the sensitivity analysis report is shown on the next two
pages. This example corresponds to the example of LP problem described
in Subsection ``Example of MPS file''.

\para{Structure of the analysis report}

For each row and column specified in the array {\it list} the routine
prints two lines containing generic information and analysis
information, which depends on the status of corresponding row or column.

Note that analysis of a row is analysis of its auxiliary variable,
which is equal to the row linear form $\sum a_jx_j$, and analysis of
a column is analysis of corresponding structural variable. Therefore,
formally, on performing the sensitivity analysis there is no difference
between rows and columns.

\newpage

\begin{landscape}
\begin{footnotesize}
\begin{verbatim}
         GLPK 4.42 - SENSITIVITY ANALYSIS REPORT                                                                         Page   1

         Problem:    PLAN
         Objective:  VALUE = 296.2166065 (MINimum)

            No. Row name     St      Activity         Slack   Lower bound       Activity      Obj coef  Obj value at Limiting
                                                   Marginal   Upper bound          range         range   break point variable
         ------ ------------ -- ------------- ------------- -------------  ------------- ------------- ------------- ------------
              1 VALUE        BS     296.21661    -296.21661          -Inf      299.25255      -1.00000        .      MN
                                                     .               +Inf      296.21661          +Inf          +Inf

              2 YIELD        NS    2000.00000        .         2000.00000     1995.06864          -Inf     296.28365 BIN3
                                                    -.01360    2000.00000     2014.03479          +Inf     296.02579 CU

              3 FE           NU      60.00000        .               -Inf       55.89016          -Inf     306.77162 BIN4
                                                   -2.56823      60.00000       62.69978       2.56823     289.28294 BIN3

              4 CU           BS      83.96751      16.03249          -Inf       93.88467       -.30613     270.51157 MN
                                                     .          100.00000       79.98213        .21474     314.24798 BIN5

              5 MN           NU      40.00000        .               -Inf       34.42336          -Inf     299.25255 BIN4
                                                    -.54440      40.00000       41.68691        .54440     295.29825 BIN3

              6 MG           BS      19.96029      10.03971          -Inf       24.74427      -1.79618     260.36433 BIN1
                                                     .           30.00000        9.40292        .28757     301.95652 MN

              7 AL           NL    1500.00000        .         1500.00000     1485.78425       -.25199     292.63444 CU
                                                     .25199          +Inf     1504.92126          +Inf     297.45669 BIN3

              8 SI           NL     250.00000      50.00000     250.00000      235.32871       -.48520     289.09812 CU
                                                     .48520     300.00000      255.06073          +Inf     298.67206 BIN3
\end{verbatim}
\end{footnotesize}
\end{landscape}

\newpage

\begin{landscape}
\begin{footnotesize}
\begin{verbatim}
         GLPK 4.42 - SENSITIVITY ANALYSIS REPORT                                                                         Page   2

         Problem:    PLAN
         Objective:  VALUE = 296.2166065 (MINimum)

            No. Column name  St      Activity      Obj coef   Lower bound       Activity      Obj coef  Obj value at Limiting
                                                   Marginal   Upper bound          range         range   break point variable
         ------ ------------ -- ------------- ------------- -------------  ------------- ------------- ------------- ------------
              1 BIN1         NL        .             .03000        .           -28.82475       -.22362     288.90594 BIN4
                                                     .25362     200.00000       33.88040          +Inf     304.80951 BIN4

              2 BIN2         BS     665.34296        .08000        .           802.22222        .01722     254.44822 BIN1
                                                     .         2500.00000      313.43066        .08863     301.95652 MN

              3 BIN3         BS     490.25271        .17000     400.00000      788.61314        .15982     291.22807 MN
                                                     .          800.00000     -347.42857        .17948     300.86548 BIN5

              4 BIN4         BS     424.18773        .12000     100.00000      710.52632        .10899     291.54745 MN
                                                     .          700.00000     -256.15524        .14651     307.46010 BIN1

              5 BIN5         NL        .             .15000        .          -201.78739        .13544     293.27940 BIN3
                                                     .01456    1500.00000       58.79586          +Inf     297.07244 BIN3

              6 ALUM         BS     299.63899        .21000        .           358.26772        .18885     289.87879 AL
                                                     .               +Inf      112.40876        .22622     301.07527 MN

              7 SILICON      BS     120.57762        .38000        .           124.27093        .14828     268.27586 BIN5
                                                     .               +Inf       85.54745        .46667     306.66667 MN

         End of report
\end{verbatim}
\end{footnotesize}
\end{landscape}

\newpage

\noindent
{\it Generic information}

{\tt No.} is the row or column ordinal number in the problem object.
Rows are numbered from 1 to $m$, and columns are numbered from 1 to $n$,
where $m$ and $n$ are, resp., the total number of rows and columns in
the problem object.

{\tt Row name} is the symbolic name assigned to the row. If the row has
no name assigned, this field contains blanks.

{\tt Column name} is the symbolic name assigned to the column. If the
column has no name assigned, this field contains blanks.

{\tt St} is the status of the row or column in the optimal solution:

{\tt BS} --- non-active constraint (row), basic column;

{\tt NL} --- inequality constraint having its lower right-hand side
active (row), non-basic column having its lower bound active;

{\tt NU} --- inequality constraint having its upper right-hand side
active (row), non-basic column having its upper bound active;

{\tt NS} --- active equality constraint (row), non-basic fixed column.

{\tt NF} --- active free row, non-basic free (unbounded) column. (This
case means that the optimal solution is dual degenerate.)

{\tt Activity} is the (primal) value of the auxiliary variable (row) or
structural variable (column) in the optimal solution.

{\tt Slack} is the (primal) value of the row slack variable.

{\tt Obj coef} is the objective coefficient of the column (structural
variable).

{\tt Marginal} is the reduced cost (dual activity) of the auxiliary
variable (row) or structural variable (column).

{\tt Lower bound} is the lower right-hand side (row) or lower bound
(column). If the row or column has no lower bound, this field contains
{\tt -Inf}.

{\tt Upper bound} is the upper right-hand side (row) or upper bound
(column). If the row or column has no upper bound, this field contains
{\tt +Inf}.

\noindent
{\it Sensitivity analysis of active bounds}

The sensitivity analysis of active bounds is performed only for rows,
which are active constraints, and only for non-basic columns, because
inactive constraints and basic columns have no active bounds.

For every auxiliary (row) or structural (column) non-basic variable the
routine starts changing its active bound in both direction. The first
of the two lines in the report corresponds to decreasing, and the
second line corresponds to increasing of the active bound. Since the
variable being analyzed is non-basic, its activity, which is equal to
its active bound, also starts changing. This changing leads to changing
of basic (auxiliary and structural) variables, which depend on the
non-basic variable. The current basis remains primal feasible and
therefore optimal while values of all basic variables are primal
feasible, i.e. are within their bounds. Therefore, if some basic
variable called the {\it limiting variable} reaches its (lower or
upper) bound first, before any other basic variables, it thereby limits
further changing of the non-basic variable, because otherwise the
current basis would become primal infeasible. The point, at which this
happens, is called the {\it break point}. Note that there are two break
points: the lower break point, which corresponds to decreasing of the
non-basic variable, and the upper break point, which corresponds to
increasing of the non-basic variable.

In the analysis report values of the non-basic variable (i.e. of its
active bound) being analyzed at both lower and upper break points are
printed in the field `{\tt Activity range}'. Corresponding values of
the objective function are printed in the field `{\tt Obj value at
break point}', and symbolic names of corresponding limiting basic
variables are printed in the field `{\tt Limiting variable}'.
If the active bound can decrease or/and increase unlimitedly, the field
`{\tt Activity range}' contains {\tt -Inf} or/and {\tt +Inf}, resp.

For example (see the example report above), row SI is a double-sided
constraint, which is active on its lower bound (right-hand side), and
its activity in the optimal solution being equal to the lower bound is
250. The activity range for this row is $[235.32871,255.06073]$. This
means that the basis remains optimal while the lower bound is
increasing up to 255.06073, and further increasing is limited by
(structural) variable BIN3. If the lower bound reaches this upper break
point, the objective value becomes equal to 298.67206.

Note that if the basis does not change, the objective function depends
on the non-basic variable linearly, and the per-unit change of the
objective function is the reduced cost (marginal value) of the
non-basic variable.

\noindent
{\it Sensitivity analysis of objective coefficients at non-basic
variables}

The sensitivity analysis of the objective coefficient at a non-basic
variable is quite simple, because in this case change in the objective
coefficient leads to equivalent change in the reduced cost (marginal
value).

For every auxiliary (row) or structural (column) non-basic variable the
routine starts changing its objective coefficient in both direction.
(Note that auxiliary variables are not included in the objective
function and therefore always have zero objective coefficients.) The
first of the two lines in the report corresponds to decreasing, and the
second line corresponds to increasing of the objective coefficient.
This changing leads to changing of the reduced cost of the non-basic
variable to be analyzed and does affect reduced costs of all other
non-basic variables. The current basis remains dual feasible and
therefore optimal while the reduced cost keeps its sign. Therefore, if
the reduced cost reaches zero, it limits further changing of the
objective coefficient (if only the non-basic variable is non-fixed).

In the analysis report minimal and maximal values of the objective
coefficient, on which the basis remains optimal, are printed in the
field `\verb|Obj coef range|'. If the objective coefficient can
decrease or/and increase unlimitedly, this field contains {\tt -Inf}
or/and {\tt +Inf}, resp.

For example (see the example report above), column BIN5 is non-basic
having its lower bound active. Its objective coefficient is 0.15, and
reduced cost in the optimal solution 0.01456. The column lower bound
remains active while the column reduced cost remains non-negative,
thus, minimal value of the objective coefficient, on which the current
basis still remains optimal, is $0.15-0.01456=0.13644$, that is
indicated in the field `\verb|Obj coef range|'.

%\newpage

%{\parskip=0pt
\noindent
{\it Sensitivity analysis of objective coefficients at basic variables}

%\medskip

To perform sensitivity analysis for every auxiliary (row) or structural
(column) variable the routine starts changing its objective coefficient
in both direction. (Note that auxiliary variables are not included in
the objective function and therefore always have zero objective
coefficients.) The first of the two lines in the report corresponds to
decreasing, and the second line corresponds to increasing of the
objective coefficient. This changing leads to changing of reduced costs
of non-basic variables. The current basis remains dual feasible and
therefore optimal while reduced costs of all non-basic variables
(except fixed variables) keep their signs. Therefore, if the reduced
cost of some non-basic non-fixed variable called the {\it limiting
variable} reaches zero first, before reduced cost of any other
non-basic non-fixed variable, it thereby limits further changing of the
objective coefficient, because otherwise the current basis would become
dual infeasible (non-optimal). The point, at which this happens, is
called the {\it break point}. Note that there are two break points: the
lower break point, which corresponds to decreasing of the objective
coefficient, and the upper break point, which corresponds to increasing
of the objective coefficient. Let the objective coefficient reach its
limit value and continue changing a bit further in the same direction
that makes the current basis dual infeasible (non-optimal). Then the
reduced cost of the non-basic limiting variable becomes ``a bit'' dual
infeasible that forces the limiting variable to enter the basis
replacing there some basic variable, which leaves the basis to keep its
primal feasibility. It should be understood that if we change the
current basis in this way exactly at the break point, both the current
and adjacent bases will be optimal with the same objective value,
because at the break point the limiting variable has zero reduced cost.
On the other hand, in the adjacent basis the value of the limiting
variable changes, because there it becomes basic, that leads to
changing of the value of the basic variable being analyzed. Note that
on determining the adjacent basis the bounds of the analyzed basic
variable are ignored as if it were a free (unbounded) variable, so it
cannot leave the current basis.

In the analysis report lower and upper limits of the objective
coefficient at the basic variable being analyzed, when the basis
remains optimal, are printed in the field `{\tt Obj coef range}'.
Corresponding values of the objective function at both lower and upper
break points are printed in the field `{\tt Obj value at break point}',
symbolic names of corresponding non-basic limiting variables are
printed in the field `{\tt Limiting variable}', and values of the basic
variable, which it would take on in the adjacent bases (as was
explained above) are printed in the field `{\tt Activity range}'.
If the objective coefficient can increase or/and decrease unlimitedly,
the field `{\tt Obj coef range}' contains {\tt -Inf} and/or {\tt +Inf},
resp. It also may happen that no dual feasible adjacent basis exists
(i.e. on entering the basis the limiting variable can increase or
decrease unlimitedly), in which case the field `{\tt Activity range}'
contains {\tt -Inf} and/or {\tt +Inf}.

For example (see the example report above), structural variable
(column) BIN3 is basic, its optimal value is 490.25271, and its
objective coefficient is 0.17. The objective coefficient range for this
column is $[0.15982,0.17948]$. This means that the basis remains
optimal while the objective coefficient is decreasing down to 0.15982,
and further decreasing is limited by (auxiliary) variable MN. If we
make the objective coefficient a bit less than 0.15982, the limiting
variable MN will enter the basis, and in that adjacent basis the
structural variable BIN3 will take on new optimal value 788.61314. At
the lower break point, where the objective coefficient is exactly
0.15982, the objective function takes on the value 291.22807 in both
the current and adjacent bases.

Note that if the basis does not change, the objective function depends
on the objective coefficient at the basic variable linearly, and the
per-unit change of the objective function is the value of the basic
variable.
%}

%* eof *%


%* glpk04.tex *%

\chapter{Advanced API Routines}

\section{Background}
\label{basbgd}

Using vector and matrix notations the LP problem (1.1)---(1.3) (see
Section \ref{seclp}, page \pageref{seclp}) can be stated as follows:

\noindent
\hspace{.5in} minimize (or maximize)
$$z=c^Tx_S+c_0\eqno(3.1)$$
\hspace{.5in} subject to linear constraints
$$x_R=Ax_S\eqno(3.2)$$
\hspace{.5in} and bounds of variables
$$
\begin{array}{l@{\ }c@{\ }l@{\ }c@{\ }l}
l_R&\leq&x_R&\leq&u_R\\
l_S&\leq&x_S&\leq&u_S\\
\end{array}\eqno(3.3)
$$
where:

$x_R=(x_1,\dots,x_m)$ is the vector of auxiliary variables;

$x_S=(x_{m+1},\dots,x_{m+n})$ is the vector of structural variables;

$z$ is the objective function;

$c=(c_1,\dots,c_n)$ is the vector of objective coefficients;

$c_0$ is the constant term (``shift'') of the objective function;

$A=(a_{11},\dots,a_{mn})$ is the constraint matrix;

$l_R=(l_1,\dots,l_m)$ is the vector of lower bounds of auxiliary
variables;

$u_R=(u_1,\dots,u_m)$ is the vector of upper bounds of auxiliary
variables;

$l_S=(l_{m+1},\dots,l_{m+n})$ is the vector of lower bounds of
structural variables;

$u_S=(u_{m+1},\dots,u_{m+n})$ is the vector of upper bounds of
structural variables.

From the simplex method's standpoint there is no difference between
auxiliary and structural variables. This allows combining all these
variables into one vector that leads to the following problem
statement:

\newpage

\noindent
\hspace{.5in} minimize (or maximize)
$$z=(0\ |\ c)^Tx+c_0\eqno(3.4)$$
\hspace{.5in} subject to linear constraints
$$(I\ |-\!A)x=0\eqno(3.5)$$
\hspace{.5in} and bounds of variables
$$l\leq x\leq u\eqno(3.6)$$
where:

$x=(x_R\ |\ x_S)$ is the $(m+n)$-vector of (all) variables;

$(0\ |\ c)$ is the $(m+n)$-vector of objective
coefficients;\footnote{Subvector 0 corresponds to objective
coefficients at auxiliary variables.}

$(I\ |-\!A)$ is the {\it augmented} constraint
$m\times(m+n)$-matrix;\footnote{Note that due to auxiliary variables
matrix $(I\ |-\!A)$ contains the unity submatrix and therefore has full
rank. This means, in particular, that the system (3.5) has no linearly
dependent constraints.}

$l=(l_R\ |\ l_S)$ is the $(m+n)$-vector of lower bounds of (all)
variables;

$u=(u_R\ |\ u_S)$ is the $(m+n)$-vector of upper bounds of (all)
variables.

By definition an {\it LP basic solution} geometrically is a point in
the space of all variables, which is the intersection of hyperplanes
corresponding to active constraints\footnote{A constraint is called
{\it active} if at a given point it is satisfied as equality, otherwise
it is called {\it inactive}.}. The space of all variables has the
dimension $m+n$, therefore, to define some basic solution we have to
define $m+n$ active constraints. Note that $m$ constraints (3.5) being
linearly independent equalities are always active, so remaining $n$
active constraints can be chosen only from bound constraints (3.6).

A variable is called {\it non-basic}, if its (lower or upper) bound is
active, otherwise it is called {\it basic}. Since, as was said above,
exactly $n$ bound constraints must be active, in any basic solution
there are always $n$ non-basic variables and $m$ basic variables.
(Note that a free variable also can be non-basic. Although such
variable has no bounds, we can think it as the difference between two
non-negative variables, which both are non-basic in this case.)

Now consider how to determine numeric values of all variables for a
given basic solution.

Let $\Pi$ be an appropriate permutation matrix of the order $(m+n)$.
Then we can write:
$$\left(\begin{array}{@{}c@{}}x_B\\x_N\\\end{array}\right)=
\Pi\left(\begin{array}{@{}c@{}}x_R\\x_S\\\end{array}\right)=\Pi x,
\eqno(3.7)$$
where $x_B$ is the vector of basic variables, $x_N$ is the vector of
non-basic variables, $x=(x_R\ |\ x_S)$ is the vector of all variables
in the original order. In this case the system of linear constraints
(3.5) can be rewritten as follows:
$$(I\ |-\!A)\Pi^T\Pi x=0\ \ \ \Rightarrow\ \ \ (B\ |\ N)
\left(\begin{array}{@{}c@{}}x_B\\x_N\\\end{array}\right)=0,\eqno(3.8)$$
where
$$(B\ |\ N)=(I\ |-\!A)\Pi^T.\eqno(3.9)$$

%\newpage

Matrix $B$ is a square non-singular $m\times m$-matrix, which is
composed from columns of the augmented constraint matrix corresponding
to basic variables. It is called the {\it basis matrix} or simply the
{\it basis}. Matrix $N$ is a rectangular $m\times n$-matrix, which is
composed from columns of the augmented constraint matrix corresponding
to non-basic variables.

From (3.8) it follows that:
$$Bx_B+Nx_N=0,\eqno(3.10)$$
therefore,
$$x_B=-B^{-1}Nx_N.\eqno(3.11)$$
Thus, the formula (3.11) shows how to determine numeric values of basic
variables $x_B$ assuming that non-basic variables $x_N$ are fixed on
their active bounds.

The $m\times n$-matrix
$$\Xi=-B^{-1}N,\eqno(3.12)$$
which appears in (3.11), is called the {\it simplex
tableau}.\footnote{This definition corresponds to the GLPK
implementation.} It shows how basic variables depend on non-basic
variables:
$$x_B=\Xi x_N.\eqno(3.13)$$

The system (3.13) is equivalent to the system (3.5) in the sense that
they both define the same set of points in the space of (primal)
variables, which satisfy to these systems. If, moreover, values of all
basic variables satisfy to their bound constraints (3.3), the
corresponding basic solution is called {\it (primal) feasible},
otherwise {\it (primal) infeasible}. It is understood that any (primal)
feasible basic solution satisfy to all constraints (3.2) and (3.3).

The LP theory says that if LP has optimal solution, it has (at least
one) basic feasible solution, which corresponds to the optimum. And the
most natural way to determine whether a given basic solution is optimal
or not is to use the Karush---Kuhn---Tucker optimality conditions.

\def\arraystretch{1.5}

For the problem statement (3.4)---(3.6) the optimality conditions are
the following:\footnote{These conditions can be appiled to any solution,
not only to a basic solution.}
$$(I\ |-\!A)x=0\eqno(3.14)$$
$$(I\ |-\!A)^T\pi+\lambda_l+\lambda_u=\nabla z=(0\ |\ c)^T\eqno(3.15)$$
$$l\leq x\leq u\eqno(3.16)$$
$$\lambda_l\geq 0,\ \ \lambda_u\leq 0\ \ \mbox{(minimization)}
\eqno(3.17)$$
$$\lambda_l\leq 0,\ \ \lambda_u\geq 0\ \ \mbox{(maximization)}
\eqno(3.18)$$
$$(\lambda_l)_k(x_k-l_k)=0,\ \ (\lambda_u)_k(x_k-u_k)=0,\ \ k=1,2,\dots,
m+n\eqno(3.19)$$
where:

$\pi=(\pi_1,\dots,\pi_m)$ is a $m$-vector of Lagrange
multipliers for equality constraints (3.5);

$\lambda_l=[(\lambda_l)_1,\dots,(\lambda_l)_n]$ is a $n$-vector of
Lagrange multipliers for lower bound constraints (3.6);

$\lambda_u=[(\lambda_u)_1,\dots,(\lambda_u)_n]$ is a $n$-vector of
Lagrange multipliers for upper bound constraints (3.6).

%\newpage

Condition (3.14) is the {\it primal} (original) system of equality
constraints (3.5).

Condition (3.15) is the {\it dual} system of equality constraints.
It requires the gradient of the objective function to be a linear
combination of normals to the planes defined by constraints of the
original problem.

Condition (3.16) is the primal (original) system of bound constraints
(3.6).

Condition (3.17) (or (3.18) in case of maximization) is the dual system
of bound constraints.

Condition (3.19) is the {\it complementary slackness condition}. It
requires, for each original (auxiliary or structural) variable $x_k$,
that either its (lower or upper) bound must be active, or zero bound of
the corresponding Lagrange multiplier ($(\lambda_l)_k$ or
$(\lambda_u)_k$) must be active.

In GLPK two multipliers $(\lambda_l)_k$ and $(\lambda_u)_k$ for each
primal variable $x_k$, $k=1,\dots,m+n$, are combined into one
multiplier:
$$\lambda_k=(\lambda_l)_k+(\lambda_u)_k,\eqno(3.20)$$
which is called a {\it dual variable} for $x_k$. This {\it cannot} lead
to an ambiguity, because both lower and upper bounds of $x_k$ cannot be
active at the same time,\footnote{If $x_k$ is a fixed variable, we can
think it as double-bounded variable $l_k\leq x_k\leq u_k$, where
$l_k=u_k.$} so at least one of $(\lambda_l)_k$ and $(\lambda_u)_k$ must
be equal to zero, and because these multipliers have different signs,
the combined multiplier, which is their sum, uniquely defines each of
them.

\def\arraystretch{1}

Using dual variables $\lambda_k$ the dual system of bound constraints
(3.17) and (3.18) can be written in the form of so called {\it ``rule of
signs''} as follows:

\medskip

\begin{center}
\begin{tabular}{|@{\,}c@{$\,$}|@{$\,$}c@{$\,$}|@{$\,$}c@{$\,$}|
@{$\,$}c|c@{$\,$}|@{$\,$}c@{$\,$}|@{$\,$}c@{$\,$}|}
\hline
Original bound&\multicolumn{3}{c|}{Minimization}&\multicolumn{3}{c|}
{Maximization}\\
\cline{2-7}
constraint&$(\lambda_l)_k$&$(\lambda_u)_k$&$(\lambda_l)_k+
(\lambda_u)_k$&$(\lambda_l)_k$&$(\lambda_u)_k$&$(\lambda_l)_k+
(\lambda_u)_k$\\
\hline
$-\infty<x_k<+\infty$&$=0$&$=0$&$\lambda_k=0$&$=0$&$=0$&$\lambda_k=0$\\
$x_k\geq l_k$&$\geq 0$&$=0$&$\lambda_k\geq 0$&$\leq 0$&$=0$&$\lambda_k
\leq0$\\
$x_k\leq u_k$&$=0$&$\leq 0$&$\lambda_k\leq 0$&$=0$&$\geq 0$&$\lambda_k
\geq0$\\
$l_k\leq x_k\leq u_k$&$\geq 0$& $\leq 0$& $-\infty\!<\!\lambda_k\!<
\!+\infty$
&$\leq 0$& $\geq 0$& $-\infty\!<\!\lambda_k\!<\!+\infty$\\
$x_k=l_k=u_k$&$\geq 0$& $\leq 0$& $-\infty\!<\!\lambda_k\!<\!+\infty$&
$\leq 0$&
$\geq 0$& $-\infty\!<\!\lambda_k\!<\!+\infty$\\
\hline
\end{tabular}
\end{center}

\medskip

May note that each primal variable $x_k$ has its dual counterpart
$\lambda_k$ and vice versa. This allows applying the same partition for
the vector of dual variables as (3.7):
$$\left(\begin{array}{@{}c@{}}\lambda_B\\\lambda_N\\\end{array}\right)=
\Pi\lambda,\eqno(3.21)$$
where $\lambda_B$ is a vector of dual variables for basic variables
$x_B$, $\lambda_N$ is a vector of dual variables for non-basic variables
$x_N$.

By definition, bounds of basic variables are inactive constraints, so in
any basic solution $\lambda_B=0$. Corresponding values of dual variables
$\lambda_N$ for non-basic variables $x_N$ can be determined in the
following way. From the dual system (3.15) we have:
$$(I\ |-\!A)^T\pi+\lambda=(0\ |\ c)^T,\eqno(3.22)$$
so multiplying both sides of (3.22) by matrix $\Pi$ gives:
$$\Pi(I\ |-\!A)^T\pi+\Pi\lambda=\Pi(0\ |\ c)^T.\eqno(3.23)$$
From (3.9) it follows that
$$\Pi(I\ |-\!A)^T=[(I\ |-\!A)\Pi^T]^T=(B\ |\ N)^T.\eqno(3.24)$$
Further, we can apply the partition (3.7) also to the vector of
objective coefficients (see (3.4)):
$$\left(\begin{array}{@{}c@{}}c_B\\c_N\\\end{array}\right)=
\Pi\left(\begin{array}{@{}c@{}}0\\c\\\end{array}\right),\eqno(3.25)$$
where $c_B$ is a vector of objective coefficients at basic variables,
$c_N$ is a vector of objective coefficients at non-basic variables.
Now, substituting (3.24), (3.21), and (3.25) into (3.23), leads to:
$$(B\ |\ N)^T\pi+(\lambda_B\ |\ \lambda_N)^T=(c_B\ |\ c_N)^T,
\eqno(3.26)$$
and transposing both sides of (3.26) gives the system:
$$\left(\begin{array}{@{}c@{}}B^T\\N^T\\\end{array}\right)\pi+
\left(\begin{array}{@{}c@{}}\lambda_B\\\lambda_N\\\end{array}\right)=
\left(\begin{array}{@{}c@{}}c_B\\c_T\\\end{array}\right),\eqno(3.27)$$
which can be written as follows:
$$\left\{
\begin{array}{@{\ }r@{\ }c@{\ }r@{\ }c@{\ }l@{\ }}
B^T\pi&+&\lambda_B&=&c_B\\
N^T\pi&+&\lambda_N&=&c_N\\
\end{array}
\right.\eqno(3.28)
$$
Lagrange multipliers $\pi=(\pi_i)$ correspond to equality constraints
(3.5) and therefore can have any sign. This allows resolving the first
subsystem of (3.28) as follows:\footnote{$B^{-T}$ means $(B^T)^{-1}=
(B^{-1})^T$.}
$$\pi=B^{-T}(c_B-\lambda_B)=-B^{-T}\lambda_B+B^{-T}c_B,\eqno(3.29)$$
and substitution of $\pi$ from (3.29) into the second subsystem of
(3.28) gives:
$$\lambda_N=-N^T\pi+c_N=N^TB^{-T}\lambda_B+(c_N-N^TB^{-T}c_B).
\eqno(3.30)$$
The latter system can be written in the following final form:
$$\lambda_N=-\Xi^T\lambda_B+d,\eqno(3.31)$$
where $\Xi$ is the simplex tableau (see (3.12)), and
$$d=c_N-N^TB^{-T}c_B=c_N+\Xi^Tc_B\eqno(3.32)$$
is the vector of so called {\it reduced costs} of non-basic variables.

Above it was said that in any basic solution $\lambda_B=0$, so
$\lambda_N=d$ as it follows from (3.31).

The system (3.31) is equivalent to the system (3.15) in the sense that
they both define the same set of points in the space of dual variables
$\lambda$, which satisfy to these systems. If, moreover, values of all
dual variables $\lambda_N$ (i.e. reduced costs $d$) satisfy to their
bound constraints (i.e. to the ``rule of signs''; see the table above),
the corresponding basic solution is called {\it dual feasible},
otherwise {\it dual infeasible}. It is understood that any dual feasible
solution satisfy to all constraints (3.15) and (3.17) (or (3.18) in case
of maximization).

It can be easily shown that the complementary slackness condition
(3.19) is always satisfied for {\it any} basic solution.\footnote{Until
double-bounded variables appear.} Therefore, a basic
solution\footnote{It is assumed that a complete basic solution has the
form $(x,\lambda)$, i.e. it includes primal as well as dual variables.}
is {\it optimal} if and only if it is primal and dual feasible, because
in this case it satifies to all the optimality conditions
(3.14)---(3.19).

\def\arraystretch{1.5}

The meaning of reduced costs $d=(d_j)$ of non-basic variables can be
explained in the following way. From (3.4), (3.7), and (3.25) it follows
that:
$$z=c_B^Tx_B+c_N^Tx_N+c_0.\eqno(3.33)$$
Substituting $x_B$ from (3.11) into (3.33) we can eliminate basic
variables and express the objective only through non-basic variables:
$$
\begin{array}{r@{\ }c@{\ }l}
z&=&c_B^T(-B^{-1}Nx_N)+c_N^Tx_N+c_0=\\
&=&(c_N^T-c_B^TB^{-1}N)x_N+c_0=\\
&=&(c_N-N^TB^{-T}c_B)^Tx_N+c_0=\\
&=&d^Tx_N+c_0.
\end{array}\eqno(3.34)
$$
From (3.34) it is seen that reduced cost $d_j$ shows how the objective
function $z$ depends on non-basic variable $(x_N)_j$ in the neighborhood
of the current basic solution, i.e. while the current basis remains
unchanged.

%%%%%%%%%%%%%%%%%%%%%%%%%%%%%%%%%%%%%%%%%%%%%%%%%%%%%%%%%%%%%%%%%%%%%%%%

\newpage

\section{LP basis routines}
\label{lpbasis}

\subsection{glp\_bf\_exists --- check if the basis factorization
exists}

\synopsis

\begin{verbatim}
   int glp_bf_exists(glp_prob *P);
\end{verbatim}

\returns

If the basis factorization for the current basis associated with the
specified problem object exists and therefore is available for
computations, the routine \verb|glp_bf_exists| returns non-zero.
Otherwise the routine returns zero.

\para{Comments}

Let the problem object have $m$ rows and $n$ columns. In GLPK the
{\it basis matrix} $B$ is a square non-singular matrix of the order $m$,
whose columns correspond to basic (auxiliary and/or structural)
variables. It is defined by the following main
equality:\footnote{For more details see Subsection \ref{basbgd},
page \pageref{basbgd}.}
$$(B\ |\ N)=(I\ |-\!A)\Pi^T,$$
where $I$ is the unity matrix of the order $m$, whose columns correspond
to auxiliary variables; $A$ is the original constraint
$m\times n$-matrix, whose columns correspond to structural variables;
$(I\ |-\!A)$ is the augmented constraint $m\times(m+n)$-matrix, whose
columns correspond to all (auxiliary and structural) variables
following in the original order; $\Pi$ is a permutation matrix of the
order $m+n$; and $N$ is a rectangular $m\times n$-matrix, whose columns
correspond to non-basic (auxiliary and/or structural) variables.

For various reasons it may be necessary to solve linear systems with
matrix $B$. To provide this possibility the GLPK implementation
maintains an invertable form of $B$ (that is, some representation of
$B^{-1}$) called the {\it basis factorization}, which is an internal
component of the problem object. Typically, the basis factorization is
computed by the simplex solver, which keeps it in the problem object
to be available for other computations.

Should note that any changes in the problem object, which affects the
basis matrix (e.g. changing the status of a row or column, changing
a basic column of the constraint matrix, removing an active constraint,
etc.), invalidates the basis factorization. So before calling any API
routine, which uses the basis factorization, the application program
must make sure (using the routine \verb|glp_bf_exists|) that the
factorization exists and therefore available for computations.

%\newpage

\subsection{glp\_factorize --- compute the basis factorization}

\synopsis

\begin{verbatim}
   int glp_factorize(glp_prob *P);
\end{verbatim}

\description

The routine \verb|glp_factorize| computes the basis factorization for
the current basis associated with the specified problem
object.\footnote{The current basis is defined by the current statuses
of rows (auxiliary variables) and columns (structural variables).}

The basis factorization is computed from ``scratch'' even if it exists,
so the application program may use the routine \verb|glp_bf_exists|,
and, if the basis factorization already exists, not to call the routine
\verb|glp_factorize| to prevent an extra work.

The routine \verb|glp_factorize| {\it does not} compute components of
the basic solution (i.e. primal and dual values).

\returns

\begin{retlist}
0 & The basis factorization has been successfully computed.\\
\verb|GLP_EBADB| & The basis matrix is invalid, because the number of
basic (auxiliary and structural) variables is not the same as the number
of rows in the problem object.\\

\verb|GLP_ESING| & The basis matrix is singular within the working
precision.\\

\verb|GLP_ECOND| & The basis matrix is ill-conditioned, i.e. its
condition number is too large.\\
\end{retlist}

\subsection{glp\_bf\_updated --- check if the basis factorization has
been updated}

\synopsis

\begin{verbatim}
   int glp_bf_updated(glp_prob *P);
\end{verbatim}

\returns

If the basis factorization has been just computed from ``scratch'', the
routine \verb|glp_bf_updated| returns zero. Otherwise, if the
factorization has been updated at least once, the routine returns
non-zero.

\para{Comments}

{\it Updating} the basis factorization means recomputing it to reflect
changes in the basis matrix. For example, on every iteration of the
simplex method some column of the current basis matrix is replaced by
a new column that gives a new basis matrix corresponding to the
adjacent basis. In this case computing the basis factorization for the
adjacent basis from ``scratch'' (as the routine \verb|glp_factorize|
does) would be too time-consuming.

On the other hand, since the basis factorization update is a numeric
computational procedure, applying it many times may lead to
accumulating round-off errors. Therefore the basis is periodically
refactorized (reinverted) from ``scratch'' (with the routine
\verb|glp_factorize|) that allows improving its numerical properties.

The routine \verb|glp_bf_updated| allows determining if the basis
factorization has been updated at least once since it was computed from
``scratch''.

\subsection{glp\_get\_bfcp --- retrieve basis factorization control
parameters}

\synopsis

\begin{verbatim}
   void glp_get_bfcp(glp_prob *P, glp_bfcp *parm);
\end{verbatim}

\description

The routine \verb|glp_get_bfcp| retrieves control parameters, which are
used on computing and updating the basis factorization associated with
the specified problem object.

Current values of the control parameters are stored in
a \verb|glp_bfcp| structure, which the parameter \verb|parm| points to.
For a detailed description of the structure \verb|glp_bfcp| see
comments to the routine \verb|glp_set_bfcp| in the next subsection.

\para{Comments}

The purpose of the routine \verb|glp_get_bfcp| is two-fold. First, it
allows the application program obtaining current values of control
parameters used by internal GLPK routines, which compute and update the
basis factorization.

The second purpose of this routine is to provide proper values for all
fields of the structure \verb|glp_bfcp| in the case when the
application program needs to change some control parameters.

\subsection{glp\_set\_bfcp --- change basis factorization control
parameters}

\synopsis

\begin{verbatim}
   void glp_set_bfcp(glp_prob *P, const glp_bfcp *parm);
\end{verbatim}

\description

The routine \verb|glp_set_bfcp| changes control parameters, which are
used by internal GLPK routines on computing and updating the basis
factorization associated with the specified problem object.

New values of the control parameters should be passed in a structure
\verb|glp_bfcp|, which the parameter \verb|parm| points to. For a
detailed description of the structure \verb|glp_bfcp| see paragraph
``Control parameters'' below.

The parameter \verb|parm| can be specified as \verb|NULL|, in which
case all control parameters are reset to their default values.

\para{Comments}

Before changing some control parameters with the routine
\verb|glp_set_bfcp| the application program should retrieve current
values of all control parameters with the routine \verb|glp_get_bfcp|.
This is needed for backward compatibility, because in the future there
may appear new members in the structure \verb|glp_bfcp|.

Note that new values of control parameters come into effect on a next
computation of the basis factorization, not immediately.

\para{Example}

\begin{footnotesize}
\begin{verbatim}
glp_prob *lp;
glp_bfcp parm;
. . .
/* retrieve current values of control parameters */
glp_get_bfcp(lp, &parm);
/* change the threshold pivoting tolerance */
parm.piv_tol = 0.05;
/* set new values of control parameters */
glp_set_bfcp(lp, &parm);
. . .
\end{verbatim}
\end{footnotesize}

\newpage

\para{Control parameters}

This paragraph describes all basis factorization control parameters
currently used in the package. Symbolic names of control parameters are
names of corresponding members in the structure \verb|glp_bfcp|.

\medskip

{\tt int type} (default: {\tt GLP\_BF\_LUF + GLP\_BF\_FT})

Basis factorization type:

\verb~GLP_BF_LUF + GLP_BF_FT~ --- $LUF$, Forrest--Tomlin update;

\verb~GLP_BF_LUF + GLP_BF_BG~ --- $LUF$, Schur complement,
Bartels--Golub update;

\verb~GLP_BF_LUF + GLP_BF_GR~ --- $LUF$, Schur complement,
Givens rotation update;

\verb~GLP_BF_BTF + GLP_BF_BG~ --- $BTF$, Schur complement,
Bartels--Golub update;

\verb~GLP_BF_BTF + GLP_BF_GR~ --- $BTF$, Schur complement,
Givens rotation update.

In case of \verb|GLP_BF_FT| the update is applied to matrix $U$, while
in cases of \verb|GLP_BF_BG| and \verb|GLP_BF_GR| the update is applied
to the Schur complement.

%\medskip
%
%{\tt int lu\_size} (default: {\tt 0})
%
%The initial size of the Sparse Vector Area, in non-zeros, used on
%computing $LU$-factorization of the basis matrix for the first time.
%If this parameter is set to 0, the initial SVA size is determined
%automatically.

\medskip

{\tt double piv\_tol} (default: {\tt 0.10})

Threshold pivoting (Markowitz) tolerance, 0 $<$ \verb|piv_tol| $<$ 1,
used on computing $LU$-factoriza\-tion of the basis matrix. Element
$u_{ij}$ of the active submatrix of factor $U$ fits to be pivot if it
satisfies to the stability criterion
$|u_{ij}| >= {\tt piv\_tol}\cdot\max|u_{i*}|$, i.e. if it is not very
small in the magnitude among other elements in the same row. Decreasing
this parameter may lead to better sparsity at the expense of numerical
accuracy, and vice versa.

\medskip

{\tt int piv\_lim} (default: {\tt 4})

This parameter is used on computing $LU$-factorization of the basis
matrix and specifies how many pivot candidates needs to be considered
on choosing a pivot element, \verb|piv_lim| $\geq$ 1. If \verb|piv_lim|
candidates have been considered, the pivoting routine prematurely
terminates the search with the best candidate found.

\medskip

{\tt int suhl} (default: {\tt GLP\_ON})

This parameter is used on computing $LU$-factorization of the basis
matrix. Being set to {\tt GLP\_ON} it enables applying the following
heuristic proposed by Uwe Suhl: if a column of the active submatrix has
no eligible pivot candidates, it is no more considered until it becomes
a column singleton. In many cases this allows reducing the time needed
for pivot searching. To disable this heuristic the parameter
\verb|suhl| should be set to {\tt GLP\_OFF}.

\medskip

{\tt double eps\_tol} (default: {\tt 1e-15})

Epsilon tolerance, \verb|eps_tol| $\geq$ 0, used on computing
$LU$-factorization of the basis matrix. If an element of the active
submatrix of factor $U$ is less than \verb|eps_tol| in the magnitude,
it is replaced by exact zero.

%\medskip
%
%{\tt double max\_gro} (default: {\tt 1e+10})
%
%Maximal growth of elements of factor $U$, \verb|max_gro| $\geq$ 1,
%allowable on computing $LU$-factorization of the basis matrix. If on
%some elimination step the ratio $u_{big}/b_{max}$ (where $u_{big}$ is
%the largest magnitude of elements of factor $U$ appeared in its active
%submatrix during all the factorization process, $b_{max}$ is the
%largest magnitude of elements of the basis matrix to be factorized),
%the basis matrix is considered as ill-conditioned.

\medskip

{\tt int nfs\_max} (default: {\tt 100})

Maximal number of additional row-like factors (entries of the eta
file), \verb|nfs_max| $\geq$ 1, which can be added to
$LU$-factorization of the basis matrix on updating it with the
Forrest--Tomlin technique. This parameter is used only once, before
$LU$-factorization is computed for the first time, to allocate working
arrays. As a rule, each update adds one new factor (however, some
updates may need no addition), so this parameter limits the number of
updates between refactorizations.

\medskip

{\tt double upd\_tol} (default: {\tt 1e-6})

Update tolerance, 0 $<$ \verb|upd_tol| $<$ 1, used on updating
$LU$-factorization of the basis matrix with the Forrest--Tomlin
technique. If after updating the magnitude of some diagonal element
$u_{kk}$ of factor $U$ becomes less than
${\tt upd\_tol}\cdot\max(|u_{k*}|, |u_{*k}|)$, the factorization is
considered as inaccurate.

\medskip

{\tt int nrs\_max} (default: {\tt 100})

Maximal number of additional rows and columns, \verb|nrs_max| $\geq$ 1,
which can be added to $LU$-factorization of the basis matrix on
updating it with the Schur complement technique. This parameter is used
only once, before $LU$-factorization is computed for the first time, to
allocate working arrays. As a rule, each update adds one new row and
column (however, some updates may need no addition), so this parameter
limits the number of updates between refactorizations.

%\medskip
%
%{\tt int rs\_size} (default: {\tt 0})
%
%The initial size of the Sparse Vector Area, in non-zeros, used to
%store non-zero elements of additional rows and columns introduced on
%updating $LU$-factorization of the basis matrix with the Schur
%complement technique. If this parameter is set to 0, the initial SVA
%size is determined automatically.

\subsection{glp\_get\_bhead --- retrieve the basis header information}

\synopsis

\begin{verbatim}
   int glp_get_bhead(glp_prob *P, int k);
\end{verbatim}

\description

The routine \verb|glp_get_bhead| returns the basis header information
for the current basis associated with the specified problem object.

\returns

If basic variable $(x_B)_k$, $1\leq k\leq m$, is $i$-th auxiliary
variable ($1\leq i\leq m$), the routine returns $i$. Otherwise, if
$(x_B)_k$ is $j$-th structural variable ($1\leq j\leq n$), the routine
returns $m+j$. Here $m$ is the number of rows and $n$ is the number of
columns in the problem object.

\para{Comments}

Sometimes the application program may need to know which original
(auxiliary and structural) variable correspond to a given basic
variable, or, that is the same, which column of the augmented
constraint matrix $(I\ |-\!A)$ correspond to a given column of the
basis matrix $B$.

\def\arraystretch{1}

The correspondence is defined as follows:\footnote{For more details see
Subsection \ref{basbgd}, page \pageref{basbgd}.}
$$\left(\begin{array}{@{}c@{}}x_B\\x_N\\\end{array}\right)=
\Pi\left(\begin{array}{@{}c@{}}x_R\\x_S\\\end{array}\right)
\ \ \Leftrightarrow
\ \ \left(\begin{array}{@{}c@{}}x_R\\x_S\\\end{array}\right)=
\Pi^T\left(\begin{array}{@{}c@{}}x_B\\x_N\\\end{array}\right),$$
where $x_B$ is the vector of basic variables, $x_N$ is the vector of
non-basic variables, $x_R$ is the vector of auxiliary variables
following in their original order,\footnote{The original order of
auxiliary and structural variables is defined by the ordinal numbers
of corresponding rows and columns in the problem object.} $x_S$ is the
vector of structural variables following in their original order, $\Pi$
is a permutation matrix (which is a component of the basis
factorization).

Thus, if $(x_B)_k=(x_R)_i$ is $i$-th auxiliary variable, the routine
returns $i$, and if $(x_B)_k=(x_S)_j$ is $j$-th structural variable,
the routine returns $m+j$, where $m$ is the number of rows in the
problem object.

\subsection{glp\_get\_row\_bind --- retrieve row index in the basis
header}

\synopsis

\begin{verbatim}
   int glp_get_row_bind(glp_prob *P, int i);
\end{verbatim}

\returns

The routine \verb|glp_get_row_bind| returns the index $k$ of basic
variable $(x_B)_k$, $1\leq k\leq m$, which is $i$-th auxiliary variable
(that is, the auxiliary variable corresponding to $i$-th row),
$1\leq i\leq m$, in the current basis associated with the specified
problem object, where $m$ is the number of rows. However, if $i$-th
auxiliary variable is non-basic, the routine returns zero.

\para{Comments}

The routine \verb|glp_get_row_bind| is an inversion of the routine
\verb|glp_get_bhead|; that is, if \linebreak
\verb|glp_get_bhead|$(P,k)$ returns $i$,
\verb|glp_get_row_bind|$(P,i)$ returns $k$, and vice versa.

\subsection{glp\_get\_col\_bind --- retrieve column index in the basis
header}

\synopsis

\begin{verbatim}
   int glp_get_col_bind(glp_prob *P, int j);
\end{verbatim}

\returns

The routine \verb|glp_get_col_bind| returns the index $k$ of basic
variable $(x_B)_k$, $1\leq k\leq m$, which is $j$-th structural
variable (that is, the structural variable corresponding to $j$-th
column), $1\leq j\leq n$, in the current basis associated with the
specified problem object, where $m$ is the number of rows, $n$ is the
number of columns. However, if $j$-th structural variable is non-basic,
the routine returns zero.

\para{Comments}

The routine \verb|glp_get_col_bind| is an inversion of the routine
\verb|glp_get_bhead|; that is, if \linebreak
\verb|glp_get_bhead|$(P,k)$ returns $m+j$,
\verb|glp_get_col_bind|$(P,j)$ returns $k$, and vice versa.

\subsection{glp\_ftran --- perform forward transformation}

\synopsis

\begin{verbatim}
   void glp_ftran(glp_prob *P, double x[]);
\end{verbatim}

\description

The routine \verb|glp_ftran| performs forward transformation (FTRAN),
i.e. it solves the system $Bx=b$, where $B$ is the basis matrix
associated with the specified problem object, $x$ is the vector of
unknowns to be computed, $b$ is the vector of right-hand sides.

On entry to the routine elements of the vector $b$ should be stored in
locations \verb|x[1]|, \dots, \verb|x[m]|, where $m$ is the number of
rows. On exit the routine stores elements of the vector $x$ in the same
locations.

\subsection{glp\_btran --- perform backward transformation}

\synopsis

\begin{verbatim}
   void glp_btran(glp_prob *P, double x[]);
\end{verbatim}

\description

The routine \verb|glp_btran| performs backward transformation (BTRAN),
i.e. it solves the system $B^Tx=b$, where $B^T$ is a matrix transposed
to the basis matrix $B$ associated with the specified problem object,
$x$ is the vector of unknowns to be computed, $b$ is the vector of
right-hand sides.

On entry to the routine elements of the vector $b$ should be stored in
locations \verb|x[1]|, \dots, \verb|x[m]|, where $m$ is the number of
rows. On exit the routine stores elements of the vector $x$ in the same
locations.

\subsection{glp\_warm\_up --- ``warm up'' LP basis}

\synopsis

\begin{verbatim}
   int glp_warm_up(glp_prob *P);
\end{verbatim}

\description

The routine \verb|glp_warm_up| ``warms up'' the LP basis for the
specified problem object using current statuses assigned to rows and
columns (that is, to auxiliary and structural variables).

This operation includes computing factorization of the basis matrix
(if it does not exist), computing primal and dual components of basic
solution, and determining the solution status.

\returns

\begin{retlist}
0 & The operation has been successfully performed.\\

\verb|GLP_EBADB| & The basis matrix is invalid, because the number of
basic (auxiliary and structural) variables is not the same as the
number of rows in the problem object.\\

\verb|GLP_ESING| & The basis matrix is singular within the working
precision.\\

\verb|GLP_ECOND| & The basis matrix is ill-conditioned, i.e. its
condition number is too large.\\
\end{retlist}

%%%%%%%%%%%%%%%%%%%%%%%%%%%%%%%%%%%%%%%%%%%%%%%%%%%%%%%%%%%%%%%%%%%%%%%%

\newpage

\section{Simplex tableau routines}

\subsection{glp\_eval\_tab\_row --- compute row of the tableau}

\synopsis

\begin{verbatim}
   int glp_eval_tab_row(glp_prob *P, int k, int ind[], double val[]);
\end{verbatim}

\description

The routine \verb|glp_eval_tab_row| computes a row of the current
simplex tableau (see Subsection 3.1.1, formula (3.12)), which (row)
corresponds to some basic variable specified by the parameter $k$ as
follows: if $1\leq k\leq m$, the basic variable is $k$-th auxiliary
variable, and if $m+1\leq k\leq m+n$, the basic variable is $(k-m)$-th
structural variable, where $m$ is the number of rows and $n$ is the
number of columns in the specified problem object. The basis
factorization must exist.

The computed row shows how the specified basic variable depends on
non-basic variables:
$$x_k=(x_B)_i=\xi_{i1}(x_N)_1+\xi_{i2}(x_N)_2+\dots+\xi_{in}(x_N)_n,$$
where $\xi_{i1}$, $\xi_{i2}$, \dots, $\xi_{in}$ are elements of the
simplex table row, $(x_N)_1$, $(x_N)_2$, \dots, $(x_N)_n$ are non-basic
(auxiliary and structural) variables.

The routine stores column indices and corresponding numeric values of
non-zero elements of the computed row in unordered sparse format in
locations \verb|ind[1]|, \dots, \verb|ind[len]| and \verb|val[1]|,
\dots, \verb|val[len]|, respectively, where $0\leq{\tt len}\leq n$ is
the number of non-zero elements in the row returned on exit.

Element indices stored in the array \verb|ind| have the same sense as
index $k$, i.e. indices 1 to $m$ denote auxiliary variables while
indices $m+1$ to $m+n$ denote structural variables (all these variables
are obviously non-basic by definition).

\returns

The routine \verb|glp_eval_tab_row| returns \verb|len|, which is the
number of non-zero elements in the simplex table row stored in the
arrays \verb|ind| and \verb|val|.

\para{Comments}

A row of the simplex table is computed as follows. At first, the
routine checks that the specified variable $x_k$ is basic and uses the
permutation matrix $\Pi$ (3.7) to determine index $i$ of basic variable
$(x_B)_i$, which corresponds to $x_k$.

The row to be computed is $i$-th row of the matrix $\Xi$ (3.12),
therefore:
$$\xi_i=e_i^T\Xi=-e_i^TB^{-1}N=-(B^{-T}e_i)^TN,$$
where $e_i$ is $i$-th unity vector. So the routine performs BTRAN to
obtain $i$-th row of the inverse $B^{-1}$:
$$\varrho_i=B^{-T}e_i,$$
and then computes elements of the simplex table row as inner products:
$$\xi_{ij}=-\varrho_i^TN_j,\ \ j=1,2,\dots,n,$$
where $N_j$ is $j$-th column of matrix $N$ (3.9), which (column)
corresponds to non-basic variable $(x_N)_j$. The permutation matrix
$\Pi$ is used again to convert indices $j$ of non-basic columns to
original ordinal numbers of auxiliary and structural variables.

\subsection{glp\_eval\_tab\_col --- compute column of the tableau}

\synopsis

\begin{verbatim}
   int glp_eval_tab_col(glp_prob *P, int k, int ind[], double val[]);
\end{verbatim}

\description

The routine \verb|glp_eval_tab_col| computes a column of the current
simplex tableau (see Subsection 3.1.1, formula (3.12)), which (column)
corresponds to some non-basic variable specified by the parameter $k$:
if $1\leq k\leq m$, the non-basic variable is $k$-th auxiliary
variable, and if $m+1\leq k\leq m+n$, the non-basic variable is
$(k-m)$-th structural variable, where $m$ is the number of rows and $n$
is the number of columns in the specified problem object. The basis
factorization must exist.

The computed column shows how basic variables depends on the specified
non-basic variable $x_k=(x_N)_j$:
$$
\begin{array}{r@{\ }c@{\ }l@{\ }l}
(x_B)_1&=&\dots+\xi_{1j}(x_N)_j&+\dots\\
(x_B)_2&=&\dots+\xi_{2j}(x_N)_j&+\dots\\
.\ \ .&.&.\ \ .\ \ .\ \ .\ \ .\ \ .\ \ .\\
(x_B)_m&=&\dots+\xi_{mj}(x_N)_j&+\dots\\
\end{array}
$$
where $\xi_{1j}$, $\xi_{2j}$, \dots, $\xi_{mj}$ are elements of the
simplex table column, $(x_B)_1$, $(x_B)_2$, \dots, $(x_B)_m$ are basic
(auxiliary and structural) variables.

The routine stores row indices and corresponding numeric values of
non-zero elements of the computed column in unordered sparse format in
locations \verb|ind[1]|, \dots, \verb|ind[len]| and \verb|val[1]|,
\dots, \verb|val[len]|, respectively, where $0\leq{\tt len}\leq m$ is
the number of non-zero elements in the column returned on exit.

Element indices stored in the array \verb|ind| have the same sense as
index $k$, i.e. indices 1 to $m$ denote auxiliary variables while
indices $m+1$ to $m+n$ denote structural variables (all these variables
are obviously basic by definition).

\returns

The routine \verb|glp_eval_tab_col| returns \verb|len|, which is the
number of non-zero elements in the simplex table column stored in the
arrays \verb|ind| and \verb|val|.

\para{Comments}

A column of the simplex table is computed as follows. At first, the
routine checks that the specified variable $x_k$ is non-basic and uses
the permutation matrix $\Pi$ (3.7) to determine index $j$ of non-basic
variable $(x_N)_j$, which corresponds to $x_k$.

The column to be computed is $j$-th column of the matrix $\Xi$ (3.12),
therefore:
$$\Xi_j=\Xi e_j=-B^{-1}Ne_j=-B^{-1}N_j,$$
where $e_j$ is $j$-th unity vector, $N_j$ is $j$-th column of matrix
$N$ (3.9). So the routine performs FTRAN to transform $N_j$ to the
simplex table column $\Xi_j=(\xi_{ij})$ and uses the permutation matrix
$\Pi$ to convert row indices $i$ to original ordinal numbers of
auxiliary and structural variables.

\newpage

\subsection{glp\_transform\_row --- transform explicitly specified row}

\synopsis

\begin{verbatim}
   int glp_transform_row(glp_prob *P, int len, int ind[], double val[]);
\end{verbatim}

\description

The routine \verb|glp_transform_row| performs the same operation as the
routine \verb|glp_eval_tab_row| with exception that the row to be
transformed is specified explicitly as a sparse vector.

The explicitly specified row may be thought as a linear form:
$$x=a_1x_{m+1}+a_2x_{m+2}+\dots+a_nx_{m+n},$$
where $x$ is an auxiliary variable for this row, $a_j$ are coefficients
of the linear form, $x_{m+j}$ are structural variables.

On entry column indices and numerical values of non-zero coefficients
$a_j$ of the specified row should be placed in locations \verb|ind[1]|,
\dots, \verb|ind[len]| and \verb|val[1]|, \dots, \verb|val[len]|, where
\verb|len| is number of non-zero coefficients.

This routine uses the system of equality constraints and the current
basis in order to express the auxiliary variable $x$ through the current
non-basic variables (as if the transformed row were added to the problem
object and the auxiliary variable $x$ were basic), i.e. the resultant
row has the form:
$$x=\xi_1(x_N)_1+\xi_2(x_N)_2+\dots+\xi_n(x_N)_n,$$
where $\xi_j$ are influence coefficients, $(x_N)_j$ are non-basic
(auxiliary and structural) variables, $n$ is the number of columns in
the problem object.

On exit the routine stores indices and numerical values of non-zero
coefficients $\xi_j$ of the resultant row in locations \verb|ind[1]|,
\dots, \verb|ind[len']| and \verb|val[1]|, \dots, \verb|val[len']|,
where $0\leq{\tt len'}\leq n$ is the number of non-zero coefficients in
the resultant row returned by the routine. Note that indices of
non-basic variables stored in the array \verb|ind| correspond to
original ordinal numbers of variables: indices 1 to $m$ mean auxiliary
variables and indices $m+1$ to $m+n$ mean structural ones.

\returns

The routine \verb|glp_transform_row| returns \verb|len'|, the number of
non-zero coefficients in the resultant row stored in the arrays
\verb|ind| and \verb|val|.

\newpage

\subsection{glp\_transform\_col --- transform explicitly specified
column}

\synopsis

\begin{verbatim}
   int glp_transform_col(glp_prob *P, int len, int ind[], double val[]);
\end{verbatim}

\description

The routine \verb|glp_transform_col| performs the same operation as the
routine \verb|glp_eval_tab_col| with exception that the column to be
transformed is specified explicitly as a sparse vector.

The explicitly specified column may be thought as it were added to
the original system of equality constraints:
$$
\begin{array}{l@{\ }c@{\ }r@{\ }c@{\ }r@{\ }c@{\ }r}
x_1&=&a_{11}x_{m+1}&+\dots+&a_{1n}x_{m+n}&+&a_1x \\
x_2&=&a_{21}x_{m+1}&+\dots+&a_{2n}x_{m+n}&+&a_2x \\
\multicolumn{7}{c}
{.\ \ .\ \ .\ \ .\ \ .\ \ .\ \ .\ \ .\ \ .\ \ .\ \ .\ \ .\ \ .\ \ .}\\
x_m&=&a_{m1}x_{m+1}&+\dots+&a_{mn}x_{m+n}&+&a_mx \\
\end{array}
$$
where $x_i$ are auxiliary variables, $x_{m+j}$ are structural variables
(presented in the problem object), $x$ is a structural variable for the
explicitly specified column, $a_i$ are constraint coefficients at $x$.

On entry row indices and numerical values of non-zero coefficients
$a_i$ of the specified column should be placed in locations
\verb|ind[1]|, \dots, \verb|ind[len]| and \verb|val[1]|, \dots,
\verb|val[len]|, where \verb|len| is number of non-zero coefficients.

This routine uses the system of equality constraints and the current
basis in order to express the current basic variables through the
structural variable $x$ (as if the transformed column were added to the
problem object and the variable $x$ were non-basic):
$$
\begin{array}{l@{\ }c@{\ }r}
(x_B)_1&=\dots+&\xi_{1}x\\
(x_B)_2&=\dots+&\xi_{2}x\\
\multicolumn{3}{c}{.\ \ .\ \ .\ \ .\ \ .\ \ .}\\
(x_B)_m&=\dots+&\xi_{m}x\\
\end{array}
$$
where $\xi_i$ are influence coefficients, $x_B$ are basic (auxiliary
and structural) variables, $m$ is the number of rows in the problem
object.

On exit the routine stores indices and numerical values of non-zero
coefficients $\xi_i$ of the resultant column in locations \verb|ind[1]|,
\dots, \verb|ind[len']| and \verb|val[1]|, \dots, \verb|val[len']|,
where $0\leq{\tt len'}\leq m$ is the number of non-zero coefficients in
the resultant column returned by the routine. Note that indices of basic
variables stored in the array \verb|ind| correspond to original ordinal
numbers of variables, i.e. indices 1 to $m$ mean auxiliary variables,
indices $m+1$ to $m+n$ mean structural ones.

\returns

The routine \verb|glp_transform_col| returns \verb|len'|, the number of
non-zero coefficients in the resultant column stored in the arrays
\verb|ind| and \verb|val|.

\newpage

\subsection{glp\_prim\_rtest --- perform primal ratio test}

\synopsis

\begin{verbatim}
   int glp_prim_rtest(glp_prob *P, int len, const int ind[], const double val[],
                      int dir, double eps);
\end{verbatim}

\description

The routine \verb|glp_prim_rtest| performs the primal ratio test using
an explicitly specified column of the simplex table.

The current basic solution associated with the LP problem object must
be primal feasible.

The explicitly specified column of the simplex table shows how the
basic variables $x_B$ depend on some non-basic variable $x$ (which is
not necessarily presented in the problem object):
$$
\begin{array}{l@{\ }c@{\ }r}
(x_B)_1&=\dots+&\xi_{1}x\\
(x_B)_2&=\dots+&\xi_{2}x\\
\multicolumn{3}{c}{.\ \ .\ \ .\ \ .\ \ .\ \ .}\\
(x_B)_m&=\dots+&\xi_{m}x\\
\end{array}
$$

The column is specifed on entry to the routine in sparse format.
Ordinal numbers of basic variables $(x_B)_i$ should be placed in
locations \verb|ind[1]|, \dots, \verb|ind[len]|, where ordinal number
1 to $m$ denote auxiliary variables, and ordinal numbers $m+1$ to $m+n$
denote structural variables. The corresponding non-zero coefficients
$\xi_i$ should be placed in locations
\verb|val[1]|, \dots, \verb|val[len]|. The arrays \verb|ind| and
\verb|val| are not changed by the routine.

The parameter \verb|dir| specifies direction in which the variable $x$
changes on entering the basis: $+1$ means increasing, $-1$ means
decreasing.

The parameter \verb|eps| is an absolute tolerance (small positive
number, say, $10^{-9}$) used by the routine to skip $\xi_i$'s whose
magnitude is less than \verb|eps|.

The routine determines which basic variable (among those specified in
\verb|ind[1]|, \dots, \verb|ind[len]|) reaches its (lower or upper)
bound first before any other basic variables do, and which, therefore,
should leave the basis in order to keep primal feasibility.

\returns

The routine \verb|glp_prim_rtest| returns the index, \verb|piv|, in the
arrays \verb|ind| and \verb|val| corresponding to the pivot element
chosen, $1\leq$ \verb|piv| $\leq$ \verb|len|. If the adjacent basic
solution is primal unbounded, and therefore the choice cannot be made,
the routine returns zero.

\para{Comments}

If the non-basic variable $x$ is presented in the LP problem object,
the input column can be computed with the routine
\verb|glp_eval_tab_col|; otherwise, it can be computed with the routine
\verb|glp_transform_col|.

\newpage

\subsection{glp\_dual\_rtest --- perform dual ratio test}

\synopsis

\begin{verbatim}
   int glp_dual_rtest(glp_prob *P, int len, const int ind[], const double val[],
                      int dir, double eps);
\end{verbatim}

\description

The routine \verb|glp_dual_rtest| performs the dual ratio test using
an explicitly specified row of the simplex table.

The current basic solution associated with the LP problem object must
be dual feasible.

The explicitly specified row of the simplex table is a linear form
that shows how some basic variable $x$ (which is not necessarily
presented in the problem object) depends on non-basic variables $x_N$:
$$x=\xi_1(x_N)_1+\xi_2(x_N)_2+\dots+\xi_n(x_N)_n.$$

The row is specified on entry to the routine in sparse format. Ordinal
numbers of non-basic variables $(x_N)_j$ should be placed in locations
\verb|ind[1]|, \dots, \verb|ind[len]|, where ordinal numbers 1 to $m$
denote auxiliary variables, and ordinal numbers $m+1$ to $m+n$ denote
structural variables. The corresponding non-zero coefficients $\xi_j$
should be placed in locations \verb|val[1]|, \dots, \verb|val[len]|.
The arrays \verb|ind| and \verb|val| are not changed by the routine.

The parameter \verb|dir| specifies direction in which the variable $x$
changes on leaving the basis: $+1$ means that $x$ goes on its lower
bound, so its reduced cost (dual variable) is increasing (minimization)
or decreasing (maximization); $-1$ means that $x$ goes on its upper
bound, so its reduced cost is decreasing (minimization) or increasing
(maximization).

The parameter \verb|eps| is an absolute tolerance (small positive
number, say, $10^{-9}$) used by the routine to skip $\xi_j$'s whose
magnitude is less than \verb|eps|.

The routine determines which non-basic variable (among those specified
in \verb|ind[1]|, \dots,\linebreak \verb|ind[len]|) should enter the
basis in order to keep dual feasibility, because its reduced cost
reaches the (zero) bound first before this occurs for any other
non-basic variables.

\returns

The routine \verb|glp_dual_rtest| returns the index, \verb|piv|, in the
arrays \verb|ind| and \verb|val| corresponding to the pivot element
chosen, $1\leq$ \verb|piv| $\leq$ \verb|len|. If the adjacent basic
solution is dual unbounded, and therefore the choice cannot be made,
the routine returns zero.

\para{Comments}

If the basic variable $x$ is presented in the LP problem object, the
input row can be computed\linebreak with the routine
\verb|glp_eval_tab_row|; otherwise, it can be computed with the routine
\linebreak \verb|glp_transform_row|.

%%%%%%%%%%%%%%%%%%%%%%%%%%%%%%%%%%%%%%%%%%%%%%%%%%%%%%%%%%%%%%%%%%%%%%%%

\newpage

\section{Post-optimal analysis routines}

\subsection{glp\_analyze\_bound --- analyze active bound of non-basic
variable}

\synopsis

\begin{verbatim}
   void glp_analyze_bound(glp_prob *P, int k, double *limit1, int *var1,
                          double *limit2, int *var2);
\end{verbatim}

\description

The routine \verb|glp_analyze_bound| analyzes the effect of varying the
active bound of specified non-basic variable.

The non-basic variable is specified by the parameter $k$, where
$1\leq k\leq m$ means auxiliary variable of corresponding row, and
$m+1\leq k\leq m+n$ means structural variable (column).

Note that the current basic solution must be optimal, and the basis
factorization must exist.

Results of the analysis have the following meaning.

\verb|value1| is the minimal value of the active bound, at which the
basis still remains primal feasible and thus optimal. \verb|-DBL_MAX|
means that the active bound has no lower limit.

\verb|var1| is the ordinal number of an auxiliary (1 to $m$) or
structural ($m+1$ to $m+n$) basic variable, which reaches its bound
first and thereby limits further decreasing the active bound being
analyzed. if \verb|value1| = \verb|-DBL_MAX|, \verb|var1| is set to 0.

\verb|value2| is the maximal value of the active bound, at which the
basis still remains primal feasible and thus optimal. \verb|+DBL_MAX|
means that the active bound has no upper limit.

\verb|var2| is the ordinal number of an auxiliary (1 to $m$) or
structural ($m+1$ to $m+n$) basic variable, which reaches its bound
first and thereby limits further increasing the active bound being
analyzed. if \verb|value2| = \verb|+DBL_MAX|, \verb|var2| is set to 0.

The parameters \verb|value1|, \verb|var1|, \verb|value2|, \verb|var2|
can be specified as \verb|NULL|, in which case corresponding information
is not stored.

\subsection{glp\_analyze\_coef --- analyze objective coefficient at
basic variable}

\synopsis

\begin{verbatim}
   void glp_analyze_coef(glp_prob *P, int k,
                         double *coef1, int *var1, double *value1,
                         double *coef2, int *var2, double *value2);
\end{verbatim}

\description

The routine \verb|glp_analyze_coef| analyzes the effect of varying the
objective coefficient at specified basic variable.

The basic variable is specified by the parameter $k$, where
$1\leq k\leq m$ means auxiliary variable of corresponding row, and
$m+1\leq k\leq m+n$ means structural variable (column).

Note that the current basic solution must be optimal, and the basis
factorization must exist.

Results of the analysis have the following meaning.

\verb|coef1| is the minimal value of the objective coefficient, at
which the basis still remains dual feasible and thus optimal.
\verb|-DBL_MAX| means that the objective coefficient has no lower
limit.

\verb|var1| is the ordinal number of an auxiliary (1 to $m$) or
structural ($m+1$ to $m+n$) non-basic variable, whose reduced cost
reaches its zero bound first and thereby limits further decreasing the
objective coefficient being analyzed.
If \verb|coef1| = \verb|-DBL_MAX|, \verb|var1| is set to 0.

\verb|value1| is value of the basic variable being analyzed in an
adjacent basis, which is defined as follows. Let the objective
coefficient reach its minimal value (\verb|coef1|) and continue
decreasing. Then the reduced cost of the limiting non-basic variable
(\verb|var1|) becomes dual infeasible and the current basis becomes
non-optimal that forces the limiting non-basic variable to enter the
basis replacing there some basic variable that leaves the basis to keep
primal feasibility. Should note that on determining the adjacent basis
current bounds of the basic variable being analyzed are ignored as if
it were free (unbounded) variable, so it cannot leave the basis. It may
happen that no dual feasible adjacent basis exists, in which case
\verb|value1| is set to \verb|-DBL_MAX| or \verb|+DBL_MAX|.

\verb|coef2| is the maximal value of the objective coefficient, at
which the basis still remains dual feasible and thus optimal.
\verb|+DBL_MAX| means that the objective coefficient has no upper
limit.

\verb|var2| is the ordinal number of an auxiliary (1 to $m$) or
structural ($m+1$ to $m+n$) non-basic variable, whose reduced cost
reaches its zero bound first and thereby limits further increasing the
objective coefficient being analyzed.
If \verb|coef2| = \verb|+DBL_MAX|, \verb|var2| is set to 0.

\verb|value2| is value of the basic variable being analyzed in an
adjacent basis, which is defined exactly in the same way as
\verb|value1| above with exception that now the objective coefficient
is increasing.

The parameters \verb|coef1|, \verb|var1|, \verb|value1|, \verb|coef2|,
\verb|var2|, \verb|value2| can be specified as \verb|NULL|, in which
case corresponding information is not stored.

%* eof *%


%* glpk05.tex *%

\chapter{Branch-and-Cut API Routines}

\section{Introduction}

\subsection{Using the callback routine}

The GLPK MIP solver based on the branch-and-cut method allows the
application program to control the solution process. This is attained
by means of the user-defined callback routine, which is called by the
solver at various points of the branch-and-cut algorithm.

The callback routine passed to the MIP solver should be written by the
user and has the following specification:\footnote{The name
{\tt foo\_bar} used here is a placeholder for the callback routine
name.}

\begin{verbatim}
   void foo_bar(glp_tree *T, void *info);
\end{verbatim}

\noindent
where \verb|tree| is a pointer to the data structure \verb|glp_tree|,
which should be used on subsequent calls to branch-and-cut interface
routines, and \verb|info| is a transit pointer passed to the routine
\verb|glp_intopt|, which may be used by the application program to pass
some external data to the callback routine.

The callback routine is passed to the MIP solver through the control
parameter structure \verb|glp_iocp| (see Chapter ``Basic API
Routines'', Section ``Mixed integer programming routines'', Subsection
``Solve MIP problem with the branch-and-cut method'') as follows:

\begin{verbatim}
   glp_prob *mip;
   glp_iocp parm;
   . . .
   glp_init_iocp(&parm);
   . . .
   parm.cb_func = foo_bar;
   parm.cb_info = ... ;
   ret = glp_intopt(mip, &parm);
   . . .
\end{verbatim}

To determine why it is being called by the MIP solver the callback
routine should use the routine \verb|glp_ios_reason| (described in this
section below), which returns a code indicating the reason for calling.
Depending on the reason the callback routine may perform necessary
actions to control the solution process.

The reason codes, which correspond to various point of the
branch-and-cut algorithm implemented in the MIP solver, are described
in Subsection ``Reasons for calling the callback routine'' below.

To ignore calls for reasons, which are not processed by the callback
routine, it should simply return to the MIP solver doing nothing. For
example:

\begin{verbatim}
void foo_bar(glp_tree *T, void *info)
{     . . .
      switch (glp_ios_reason(T))
      {  case GLP_IBRANCH:
            . . .
            break;
         case GLP_ISELECT:
            . . .
            break;
         default:
            /* ignore call for other reasons */
            break;
      }
      return;
}
\end{verbatim}

To control the solution process as well as to obtain necessary
information the callback routine may use the branch-and-cut API
routines described in this chapter. Names of all these routines begin
with `\verb|glp_ios_|'.

\subsection{Branch-and-cut algorithm}

This section gives a schematic description of the branch-and-cut
algorithm as it is implemented in the GLPK MIP solver.

{\it 1. Initialization}

Set $L:=\{P_0\}$, where $L$ is the {\it active list} (i.e. the list of
active subproblems), $P_0$ is the original MIP problem to be solved.

Set $z^{\it best}:=+\infty$ (in case of minimization) or
$z^{\it best}:=-\infty$ (in case of maximization), where $z^{\it best}$
is {\it incumbent value}, i.e. an upper (minimization) or lower
(maximization) global bound for $z^{\it opt}$, the optimal objective
value for $P^0$.

{\it 2. Subproblem selection}

If $L=\varnothing$ then GO TO 9.

Select $P\in L$, i.e. make active subproblem $P$ current.

%\newpage

{\it 3. Solving LP relaxation}

Solve $P^{\it LP}$, which is LP relaxation of $P$.

If $P^{\it LP}$ has no primal feasible solution then GO TO 8.

Let $z^{\it LP}$ be the optimal objective value for $P^{\it LP}$.

If $z^{\it LP}\geq z^{\it best}$ (minimization) or
$z^{\it LP}\leq z^{\rm best}$ (), GO TO 8.

{\it 4. Adding ``lazy'' constraints}

Let $x^{\it LP}$ be the optimal solution to $P^{\it LP}$.

If there are ``lazy'' constraints (i.e. essential constraints not
included in the original MIP problem $P_0$), which are violated at the
optimal point $x^{\it LP}$, add them to $P$, and GO TO 3.

{\it 5. Check for integrality}

Let $x_j$ be a variable, which is required to be integer, and let
$x^{\it LP}_j\in x^{\it LP}$ be its value in the optimal solution to
$P^{\it LP}$.

If $x^{\it LP}_j$ are integral for all integer variables, then a better
integer feasible solution is found. Store its components, set
$z^{\it best}:=z^{\it LP}$, and GO TO 8.

{\it 6. Adding cutting planes}

If there are cutting planes (i.e. valid constraints for $P$),
which are violated at the optimal point $x^{\it LP}$, add them to $P$,
and GO TO 3.

{\it 7. Branching}

Select {\it branching variable} $x_j$, i.e. a variable, which is
required to be integer, and whose value $x^{\it LP}_j\in x^{\it LP}$ is
fractional in the optimal solution to $P^{\it LP}$.

Create new subproblem $P^D$ (so called {\it down branch}), which is
identical to the current subproblem $P$ with exception that the upper
bound of $x_j$ is replaced by $\lfloor x^{\it LP}_j\rfloor$. (For
example, if $x^{\it LP}_j=3.14$, the new upper bound of $x_j$ in the
down branch will be $\lfloor 3.14\rfloor=3$.)

Create new subproblem $P^U$ (so called {\it up branch}), which is
identical to the current subproblem $P$ with exception that the lower
bound of $x_j$ is replaced by $\lceil x^{\it LP}_j\rceil$. (For example,
if $x^{\it LP}_j=3.14$, the new lower bound of $x_j$ in the up branch
will be $\lceil 3.14\rceil=4$.)

Set $L:=(L\backslash\{P\})\cup\{P^D,P^U\}$, i.e. remove the current
subproblem $P$ from the active list $L$ and add two new subproblems
$P^D$ and $P^U$ to it. Then GO TO 2.

{\it 8. Pruning}

Remove from the active list $L$ all subproblems (including the current
one), whose local bound $\widetilde{z}$ is not better than the global
bound $z^{\it best}$, i.e. set $L:=L\backslash\{P\}$ for all $P$, where
$\widetilde{z}\geq z^{\it best}$ (in case of minimization) or
$\widetilde{z}\leq z^{\it best}$ (in case of maximization), and then
GO TO 2.

The local bound $\widetilde{z}$ for subproblem $P$ is an lower
(minimization) or upper (maximization) bound for integer optimal
solution to {\it this} subproblem (not to the original problem). This
bound is local in the sense that only subproblems in the subtree rooted
at node $P$ cannot have better integer feasible solutions. Note that
the local bound is not necessarily the optimal objective value to LP
relaxation $P^{\it LP}$.

{\it 9. Termination}

If $z^{\it best}=+\infty$ (in case of minimization) or
$z^{\it best}=-\infty$ (in case of maximization), the original problem
$P_0$ has no integer feasible solution. Otherwise, the last integer
feasible solution stored on step 5 is the integer optimal solution to
the original problem $P_0$ with $z^{\it opt}=z^{\it best}$. STOP.

\subsection{The search tree}

On the branching step of the branch-and-cut algorithm the current
subproblem is divided into two\footnote{In more general cases the
current subproblem may be divided into more than two subproblems.
However, currently such feature is not used in GLPK.} new subproblems,
so the set of all subproblems can be represented in the form of a rooted
tree, which is called the {\it search} or {\it branch-and-bound} tree.
An example of the search tree is shown on Fig.~1. Each node of the
search tree corresponds to a subproblem, so the terms `node' and
`subproblem' may be used synonymously.

\begin{figure}[t]
\noindent\hfil
\xymatrix @R=20pt @C=10pt
{&&&&&&*+<14pt>[o][F=]{A}\ar@{-}[dllll]\ar@{-}[dr]\ar@{-}[drrrr]&&&&\\
&&*+<14pt>[o][F=]{B}\ar@{-}[dl]\ar@{-}[dr]&&&&&*+<14pt>[o][F=]{C}
\ar@{-}[dll]\ar@{-}[dr]\ar@{-}[drrr]&&&*+<14pt>[o][F-]{\times}\\
&*+<14pt>[o][F-]{\times}\ar@{-}[dl]\ar@{-}[d]\ar@{-}[dr]&&
*+<14pt>[o][F-]{D}&&*+<14pt>[o][F=]{E}\ar@{-}[dl]\ar@{-}[dr]&&&
*+<14pt>[o][F=]{F}\ar@{-}[dl]\ar@{-}[dr]&&*+<14pt>[o][F-]{G}\\
*+<14pt>[o][F-]{\times}&*+<14pt>[o][F-]{\times}&*+<14pt>[o][F-]{\times}
&&*+<14pt>[][F-]{H}&&*+<14pt>[o][F-]{I}&*+<14pt>[o][F-]{\times}&&
*+<14pt>[o][F-]{J}&\\}

\bigskip

\noindent\hspace{.8in}
\xymatrix @R=11pt
{*+<20pt>[][F-]{}&*\txt{\makebox[1in][l]{Current}}&&
*+<20pt>[o][F-]{}&*\txt{\makebox[1in][l]{Active}}\\
*+<20pt>[o][F=]{}&*\txt{\makebox[1in][l]{Non-active}}&&
*+<14pt>[o][F-]{\times}&*\txt{\makebox[1in][l]{Fathomed}}\\
}

\bigskip

\begin{center}
Fig. 1. An example of the search tree.
\end{center}
\end{figure}

In GLPK each node may have one of the following four statuses:

%\vspace*{-8pt}

%\begin{itemize}
\Item{---}{\it current node} is the active node currently being
processed;

\Item{---}{\it active node} is a leaf node, which still has to be
processed;

\Item{---}{\it non-active node} is a node, which has been processed,
but not fathomed;

\Item{---}{\it fathomed node} is a node, which has been processed and
fathomed.
%\end{itemize}

%\vspace*{-8pt}

In the data structure representing the search tree GLPK keeps only
current, active, and non-active nodes. Once a node has been fathomed,
it is removed from the tree data structure.

Being created each node of the search tree is assigned a distinct
positive integer called the {\it subproblem reference number}, which
may be used by the application program to specify a particular node of
the tree. The root node corresponding to the original problem to be
solved is always assigned the reference number 1.

\subsection{Current subproblem}

The current subproblem is a MIP problem corresponding to the current
node of the search tree. It is represented as the GLPK problem object
(\verb|glp_prob|) that allows the application program using API
routines to access its content in the standard way. If the MIP
presolver is not used, it is the original problem object passed to the
routine \verb|glp_intopt|; otherwise, it is an internal problem object
built by the MIP presolver.

Note that the problem object is used by the MIP solver itself during
the solution process for various purposes (to solve LP relaxations, to
perfom branching, etc.), and even if the MIP presolver is not used, the
current content of the problem object may differ from its original
content. For example, it may have additional rows, bounds of some rows
and columns may be changed, etc. In particular, LP segment of the
problem object corresponds to LP relaxation of the current subproblem.
However, on exit from the MIP solver the content of the problem object
is restored to its original state.

To obtain information from the problem object the application program
may use any API routines, which do not change the object. Using API
routines, which change the problem object, is restricted to stipulated
cases.

\subsection{The cut pool}

The {\it cut pool} is a set of cutting plane constraints maintained by
the MIP solver. It is used by the GLPK cut generation routines and may
be used by the application program in the same way, i.e. rather than
to add cutting plane constraints directly to the problem object the
application program may store them to the cut pool. In the latter case
the solver looks through the cut pool, selects efficient constraints,
and adds them to the problem object.

\subsection{Reasons for calling the callback routine}

The callback routine may be called by the MIP solver for the following
reasons.

\para{Request for subproblem selection}

The callback routine is called with the reason code \verb|GLP_ISELECT|
if the current subproblem has been fathomed and therefore there is no
current subproblem.

In response the callback routine may select some subproblem from the
active list and pass its reference number to the solver using the
routine \verb|glp_ios_select_node|, in which case the solver continues
the search from the specified active subproblem. If no selection is
made by the callback routine, the solver uses a backtracking technique
specified by the control parameter \verb|bt_tech|.

To explore the active list (i.e. active nodes of the branch-and-bound
tree) the callback routine may use the routines \verb|glp_ios_next_node|
and \verb|glp_ios_prev_node|.

\para{Request for preprocessing}

The callback routine is called with the reason code \verb|GLP_IPREPRO|
if the current subproblem has just been selected from the active list
and its LP relaxation is not solved yet.

In response the callback routine may perform some preprocessing of the
current subproblem like tightening bounds of some variables or removing
bounds of some redundant constraints.

\para{Request for row generation}

The callback routine is called with the reason code \verb|GLP_IROWGEN|
if LP relaxation of the current subproblem has just been solved to
optimality and its objective value is better than the best known
integer feasible solution.

In response the callback routine may add one or more ``lazy''
constraints (rows), which are violated by the current optimal solution
of LP relaxation, using API routines \verb|glp_add_rows|,
\verb|glp_set_row_name|, \verb|glp_set_row_bnds|, and
\verb|glp_set_mat_row|, in which case the solver will perform
re-optimization of LP relaxation. If there are no violated constraints,
the callback routine should just return.

Note that components of optimal solution to LP relaxation can be
obtained with API\linebreak routines \verb|glp_get_obj_val|,
\verb|glp_get_row_prim|, \verb|glp_get_row_dual|,
\verb|glp_get_col_prim|, and\linebreak \verb|glp_get_col_dual|.

\para{Request for heuristic solution}

The callback routine is called with the reason code \verb|GLP_IHEUR|
if LP relaxation of the current subproblem being solved to optimality
is integer infeasible (i.e. values of some structural variables of
integer kind are fractional), though its objective value is better than
the best known integer feasible solution.

In response the callback routine may try applying a primal heuristic
to find an integer feasible solution,\footnote{Integer feasible to the
original MIP problem, not to the current subproblem.} which is better
than the best known one. In case of success the callback routine may
store such better solution in the problem object using the routine
\verb|glp_ios_heur_sol|.

\para{Request for cut generation}

The callback routine is called with the reason code \verb|GLP_ICUTGEN|
if LP relaxation of the current subproblem being solved to optimality
is integer infeasible (i.e. values of some structural variables of
integer kind are fractional), though its objective value is better than
the best known integer feasible solution.

In response the callback routine may reformulate the {\it current}
subproblem (before it will be splitted up due to branching) by adding
to the problem object one or more {\it cutting plane constraints},
which cut off the fractional optimal point from the MIP
polytope.\footnote{Since these constraints are added to the current
subproblem, they may be globally as well as locally valid.}

Adding cutting plane constraints may be performed in two ways.
One way is the same as for the reason code \verb|GLP_IROWGEN| (see
above), in which case the callback routine adds new rows corresponding
to cutting plane constraints directly to the current subproblem.

The other way is to add cutting plane constraints to the
{\it cut pool}, a set of cutting plane constraints maintained by the
solver, rather than directly to the current subproblem. In this case
after return from the callback routine the solver looks through the
cut pool, selects efficient cutting plane constraints, adds them to the
current subproblem, drops other constraints, and then performs
re-optimization.

\para{Request for branching}

The callback routine is called with the reason code \verb|GLP_IBRANCH|
if LP relaxation of the current subproblem being solved to optimality
is integer infeasible (i.e. values of some structural variables of
integer kind are fractional), though its objective value is better than
the best known integer feasible solution.

In response the callback routine may choose some variable suitable for
branching (i.e. integer variable, whose value in optimal solution to
LP relaxation of the current subproblem is fractional) and pass its
ordinal number to the solver using the routine
\verb|glp_ios_branch_upon|, in which case the solver splits the current
subproblem in two new subproblems and continues the search.
If no choice is made by the callback routine, the solver uses
a branching technique specified by the control parameter \verb|br_tech|.

\para{Better integer solution found}

The callback routine is called with the reason code \verb|GLP_IBINGO|
if LP relaxation of the current subproblem being solved to optimality
is integer feasible (i.e. values of all structural variables of integer
kind are integral within the working precision) and its objective value
is better than the best known integer feasible solution.

Optimal solution components for LP relaxation can be obtained in the
same way as for the reason code \verb|GLP_IROWGEN| (see above).

Components of the new MIP solution can be obtained with API routines
\verb|glp_mip_obj_val|, \verb|glp_mip_row_val|, and
\verb|glp_mip_col_val|. Note, however, that due to row/cut generation
there may be additional rows in the problem object.

The difference between optimal solution to LP relaxation and
corresponding MIP solution is that in the former case some structural
variables of integer kind (namely, basic variables) may have values,
which are close to nearest integers within the working precision, while
in the latter case all such variables have exact integral values.

The reason \verb|GLP_IBINGO| is intended only for informational
purposes, so the callback routine should not modify the problem object
in this case.

%%%%%%%%%%%%%%%%%%%%%%%%%%%%%%%%%%%%%%%%%%%%%%%%%%%%%%%%%%%%%%%%%%%%%%%%

\newpage

\section{Basic routines}

\subsection{glp\_ios\_reason --- determine reason for calling the
callback routine}

\synopsis

\begin{verbatim}
   int glp_ios_reason(glp_tree *T);
\end{verbatim}

\returns

The routine \verb|glp_ios_reason| returns a code, which indicates why
the user-defined callback routine is being called:

\verb|GLP_ISELECT| --- request for subproblem selection;

\verb|GLP_IPREPRO| --- request for preprocessing;

\verb|GLP_IROWGEN| --- request for row generation;

\verb|GLP_IHEUR  | --- request for heuristic solution;

\verb|GLP_ICUTGEN| --- request for cut generation;

\verb|GLP_IBRANCH| --- request for branching;

\verb|GLP_IBINGO | --- better integer solution found.

\subsection{glp\_ios\_get\_prob --- access the problem object}

\synopsis

\begin{verbatim}
   glp_prob *glp_ios_get_prob(glp_tree *T);
\end{verbatim}

\description

The routine \verb|glp_ios_get_prob| can be called from the user-defined
callback routine to access the problem object, which is used by the MIP
solver. It is the original problem object passed to the routine
\verb|glp_intopt| if the MIP presolver is not used; otherwise it is an
internal problem object built by the presolver.

\returns

The routine \verb|glp_ios_get_prob| returns a pointer to the problem
object used by the MIP solver.

\para{Comments}

To obtain various information about the problem instance the callback
routine can access the problem object (i.e. the object of type
\verb|glp_prob|) using the routine \verb|glp_ios_get_prob|. It is the
original problem object passed to the routine \verb|glp_intopt| if the
MIP presolver is not used; otherwise it is an internal problem object
built by the presolver.

\newpage

\subsection{glp\_ios\_row\_attr --- determine additional row
attributes}

\synopsis

\begin{verbatim}
   void glp_ios_row_attr(glp_tree *T, int i, glp_attr *attr);
\end{verbatim}

\description

The routine \verb|glp_ios_row_attr| retrieves additional attributes of
$i$-th row of the current subproblem and stores them in the structure
\verb|glp_attr|, which the parameter \verb|attr| points to.

The structure \verb|glp_attr| has the following fields:

\medskip

{\tt int level}

Subproblem level at which the row was created. (If \verb|level| = 0,
the row was added either to the original problem object passed to the
routine \verb|glp_intopt| or to the root subproblem on generating
``lazy'' or/and cutting plane constraints.)

\medskip

{\tt int origin}

The row origin flag:

\verb|GLP_RF_REG | --- regular constraint;

\verb|GLP_RF_LAZY| --- ``lazy'' constraint;

\verb|GLP_RF_CUT | --- cutting plane constraint.

\medskip

{\tt int klass}

The row class descriptor, which is a number passed to the routine
\verb|glp_ios_add_row| as its third parameter. If the row is a cutting
plane constraint generated by the solver, its class may be the
following:

\verb|GLP_RF_GMI | --- Gomory's mixed integer cut;

\verb|GLP_RF_MIR | --- mixed integer rounding cut;

\verb|GLP_RF_COV | --- mixed cover cut;

\verb|GLP_RF_CLQ | --- clique cut.

\subsection{glp\_ios\_mip\_gap --- compute relative MIP gap}

\synopsis

\begin{verbatim}
   double glp_ios_mip_gap(glp_tree *T);
\end{verbatim}

\description

The routine \verb|glp_ios_mip_gap| computes the relative MIP gap (also
called {\it duality gap}) with the following formula:
$${\tt gap} = \frac{|{\tt best\_mip} - {\tt best\_bnd}|}
{|{\tt best\_mip}| + {\tt DBL\_EPSILON}}$$
where \verb|best_mip| is the best integer feasible solution found so
far, \verb|best_bnd| is the best (global) bound. If no integer feasible
solution has been found yet, \verb|gap| is set to \verb|DBL_MAX|.

\newpage

\returns

The routine \verb|glp_ios_mip_gap| returns the relative MIP gap.

\para{Comments}

The relative MIP gap is used to measure the quality of the best integer
feasible solution found so far, because the optimal solution value
$z^*$ for the original MIP problem always lies in the range
$${\tt best\_bnd}\leq z^*\leq{\tt best\_mip}$$
in case of minimization, or in the range
$${\tt best\_mip}\leq z^*\leq{\tt best\_bnd}$$
in case of maximization.

To express the relative MIP gap in percents the value returned by the
routine \verb|glp_ios_mip_gap| should be multiplied by 100\%.

\subsection{glp\_ios\_node\_data --- access application-specific data}

\synopsis

\begin{verbatim}
   void *glp_ios_node_data(glp_tree *T, int p);
\end{verbatim}

\description

The routine \verb|glp_ios_node_data| allows the application accessing
a memory block allocated for the subproblem (which may be active or
inactive), whose reference number is $p$.

The size of the block is defined by the control parameter
\verb|cb_size| passed to the routine \verb|glp_intopt|. The block is
initialized by binary zeros on creating corresponding subproblem, and
its contents is kept until the subproblem will be removed from the
tree.

The application may use these memory blocks to store specific data for
each subproblem.

\returns

The routine \verb|glp_ios_node_data| returns a pointer to the memory
block for the specified subproblem. Note that if \verb|cb_size| = 0,
the routine returns a null pointer.

\subsection{glp\_ios\_select\_node --- select subproblem to continue
the search}

\synopsis

\begin{verbatim}
   void glp_ios_select_node(glp_tree *T, int p);
\end{verbatim}

\description

The routine \verb|glp_ios_select_node| can be called from the
user-defined callback routine in response to the reason
\verb|GLP_ISELECT| to select an active subproblem, whose reference
number\linebreak is $p$. The search will be continued from the
subproblem selected.

\newpage

\subsection{glp\_ios\_heur\_sol --- provide solution found by
heuristic}

\synopsis

\begin{verbatim}
   int glp_ios_heur_sol(glp_tree *T, const double x[]);
\end{verbatim}

\description

The routine \verb|glp_ios_heur_sol| can be called from the user-defined
callback routine in response to the reason \verb|GLP_IHEUR| to provide
an integer feasible solution found by a primal heuristic.

Primal values of {\it all} variables (columns) found by the heuristic
should be placed in locations $x[1]$, \dots, $x[n]$, where $n$ is the
number of columns in the original problem object. Note that the routine
\verb|glp_ios_heur_sol| does {\it not} check primal feasibility of the
solution provided.

Using the solution passed in the array $x$ the routine computes value
of the objective function. If the objective value is better than the
best known integer feasible solution, the routine computes values of
auxiliary variables (rows) and stores all solution components in the
problem object.

\returns

If the provided solution is accepted, the routine
\verb|glp_ios_heur_sol| returns zero. Otherwise, if the provided
solution is rejected, the routine returns non-zero.

\vspace*{-5pt}

\subsection{glp\_ios\_can\_branch --- check if can branch upon
specified variable}

\synopsis

\begin{verbatim}
   int glp_ios_can_branch(glp_tree *T, int j);
\end{verbatim}

\returns

If $j$-th variable (column) can be used to branch upon, the routine
returns non-zero, otherwise zero.

\vspace*{-5pt}

\subsection{glp\_ios\_branch\_upon --- choose variable to branch upon}

\synopsis

\begin{verbatim}
   void glp_ios_branch_upon(glp_tree *T, int j, int sel);
\end{verbatim}

\description

The routine \verb|glp_ios_branch_upon| can be called from the
user-defined callback routine in response to the reason
\verb|GLP_IBRANCH| to choose a branching variable, whose ordinal number
\linebreak is $j$. Should note that only variables, for which the
routine \verb|glp_ios_can_branch| returns non-zero, can be used to
branch upon.

The parameter \verb|sel| is a flag that indicates which branch
(subproblem) should be selected next to continue the search:

\verb|GLP_DN_BRNCH| --- select down-branch;

\verb|GLP_UP_BRNCH| --- select up-branch;

\verb|GLP_NO_BRNCH| --- use general selection technique.

\newpage

\para{Comments}

On branching the solver removes the current active subproblem from the
active list and creates two new subproblems ({\it down-} and {\it
up-branches}), which are added to the end of the active list. Note that
the down-branch is created before the up-branch, so the last active
subproblem will be the up-branch.

The down- and up-branches are identical to the current subproblem with
exception that in the down-branch the upper bound of $x_j$, the variable
chosen to branch upon, is replaced by $\lfloor x_j^*\rfloor$, while in
the up-branch the lower bound of $x_j$ is replaced by
$\lceil x_j^*\rceil$, where $x_j^*$ is the value of $x_j$ in optimal
solution to LP relaxation of the current subproblem. For example, if
$x_j^*=3.14$, the new upper bound of $x_j$ in the down-branch is
$\lfloor 3.14\rfloor=3$, and the new lower bound in the up-branch is
$\lceil 3.14\rceil=4$.)

Additionally the callback routine may select either down- or up-branch,
from which the solver will continue the search. If none of the branches
is selected, a general selection technique will be used.

\subsection{glp\_ios\_terminate --- terminate the solution process}

\synopsis

\begin{verbatim}
   void glp_ios_terminate(glp_tree *T);
\end{verbatim}

\description

The routine \verb|glp_ios_terminate| sets a flag indicating that the
MIP solver should prematurely terminate the search.

%%%%%%%%%%%%%%%%%%%%%%%%%%%%%%%%%%%%%%%%%%%%%%%%%%%%%%%%%%%%%%%%%%%%%%%%

\newpage

\section{The search tree exploring routines}

\subsection{glp\_ios\_tree\_size --- determine size of the search tree}

\synopsis

\begin{verbatim}
   void glp_ios_tree_size(glp_tree *T, int *a_cnt, int *n_cnt, int *t_cnt);
\end{verbatim}

\description

The routine \verb|glp_ios_tree_size| stores the following three counts
which characterize the current size of the search tree:

\verb|a_cnt| is the current number of active nodes, i.e. the current
size of the active list;

\verb|n_cnt| is the current number of all (active and inactive) nodes;

\verb|t_cnt| is the total number of nodes including those which have
been already removed from the tree. This count is increased whenever
a new node appears in the tree and never decreased.

If some of the parameters \verb|a_cnt|, \verb|n_cnt|, \verb|t_cnt| is
a null pointer, the corresponding count is not stored.

\subsection{glp\_ios\_curr\_node --- determine current active
subproblem}

\synopsis

\begin{verbatim}
   int glp_ios_curr_node(glp_tree *T);
\end{verbatim}

\returns

The routine \verb|glp_ios_curr_node| returns the reference number of
the current active subproblem. However, if the current subproblem does
not exist, the routine returns zero.

\subsection{glp\_ios\_next\_node --- determine next active subproblem}

\synopsis

\begin{verbatim}
   int glp_ios_next_node(glp_tree *T, int p);
\end{verbatim}

\returns

If the parameter $p$ is zero, the routine \verb|glp_ios_next_node|
returns the reference number of the first active subproblem. However,
if the tree is empty, zero is returned.

If the parameter $p$ is not zero, it must specify the reference number
of some active subproblem, in which case the routine returns the
reference number of the next active subproblem. However, if there is
no next active subproblem in the list, zero is returned.

All subproblems in the active list are ordered chronologically, i.e.
subproblem $A$ precedes subproblem $B$ if $A$ was created before $B$.

\newpage

\subsection{glp\_ios\_prev\_node --- determine previous active
subproblem}

\synopsis

\begin{verbatim}
   int glp_ios_prev_node(glp_tree *T, int p);
\end{verbatim}

\returns

If the parameter $p$ is zero, the routine \verb|glp_ios_prev_node|
returns the reference number of the last active subproblem. However, if
the tree is empty, zero is returned.

If the parameter $p$ is not zero, it must specify the reference number
of some active subproblem, in which case the routine returns the
reference number of the previous active subproblem. However, if there
is no previous active subproblem in the list, zero is returned.

All subproblems in the active list are ordered chronologically, i.e.
subproblem $A$ precedes subproblem $B$ if $A$ was created before $B$.

\subsection{glp\_ios\_up\_node --- determine parent subproblem}

\synopsis

\begin{verbatim}
   int glp_ios_up_node(glp_tree *T, int p);
\end{verbatim}

\returns

The parameter $p$ must specify the reference number of some (active or
inactive) subproblem, in which case the routine \verb|iet_get_up_node|
returns the reference number of its parent subproblem. However, if the
specified subproblem is the root of the tree and, therefore, has
no parent, the routine returns zero.

\subsection{glp\_ios\_node\_level --- determine subproblem level}

\synopsis

\begin{verbatim}
   int glp_ios_node_level(glp_tree *T, int p);
\end{verbatim}

\returns

The routine \verb|glp_ios_node_level| returns the level of the
subproblem, whose reference number is $p$, in the branch-and-bound
tree. (The root subproblem has level 0, and the level of any other
subproblem is the level of its parent plus one.)

\subsection{glp\_ios\_node\_bound --- determine subproblem local bound}

\synopsis

\begin{verbatim}
   double glp_ios_node_bound(glp_tree *T, int p);
\end{verbatim}

\returns

The routine \verb|glp_ios_node_bound| returns the local bound for
(active or inactive) subproblem, whose reference number is $p$.

\newpage

\para{Comments}

The local bound for subproblem $p$ is an lower (minimization) or upper
(maximization) bound for integer optimal solution to {\it this}
subproblem (not to the original problem). This bound is local in the
sense that only subproblems in the subtree rooted at node $p$ cannot
have better integer feasible solutions.

On creating a subproblem (due to the branching step) its local bound is
inherited from its parent and then may get only stronger (never weaker).
For the root subproblem its local bound is initially set to
\verb|-DBL_MAX| (minimization) or \verb|+DBL_MAX| (maximization) and
then improved as the root LP relaxation has been solved.

Note that the local bound is not necessarily the optimal objective
value to corresponding LP relaxation.

\subsection{glp\_ios\_best\_node --- find active subproblem with best
local bound}

\synopsis

\begin{verbatim}
   int glp_ios_best_node(glp_tree *T);
\end{verbatim}

\returns

The routine \verb|glp_ios_best_node| returns the reference number of
the active subproblem, whose local bound is best (i.e. smallest in case
of minimization or largest in case of maximization). However, if the
tree is empty, the routine returns zero.

\para{Comments}

The best local bound is an lower (minimization) or upper (maximization)
bound for integer optimal solution to the original MIP problem.

%%%%%%%%%%%%%%%%%%%%%%%%%%%%%%%%%%%%%%%%%%%%%%%%%%%%%%%%%%%%%%%%%%%%%%%%

\newpage

\section{The cut pool routines}

\subsection{glp\_ios\_pool\_size --- determine current size of the cut
pool}

\synopsis

\begin{verbatim}
   int glp_ios_pool_size(glp_tree *T);
\end{verbatim}

\returns

The routine \verb|glp_ios_pool_size| returns the current size of the
cut pool, that is, the number of cutting plane constraints currently
added to it.

\subsection{glp\_ios\_add\_row --- add constraint to the cut pool}

\synopsis

\begin{verbatim}
   int glp_ios_add_row(glp_tree *T, const char *name, int klass, int flags,
       int len, const int ind[], const double val[], int type, double rhs);
\end{verbatim}

\description

The routine \verb|glp_ios_add_row| adds specified row (cutting plane
constraint) to the cut pool.

The cutting plane constraint should have the following format:
$$\sum_{j\in J}a_jx_j\left\{\begin{array}{@{}c@{}}\geq\\\leq\\
\end{array}\right\}b,$$
where $J$ is a set of indices (ordinal numbers) of structural
variables, $a_j$ are constraint coefficients, $x_j$ are structural
variables, $b$ is the right-hand side.

The parameter \verb|name| specifies a symbolic name assigned to the
constraint (1 up to 255 characters). If it is \verb|NULL| or an empty
string, no name is assigned.

The parameter \verb|klass| specifies the constraint class, which must
be either zero or a number in the range from 101 to 200.
The application may use this attribute to distinguish between cutting
plane constraints of different classes.\footnote{Constraint classes
numbered from 1 to 100 are reserved for GLPK cutting plane generators.}

The parameter \verb|flags| currently is not used and must be zero.

Ordinal numbers of structural variables (i.e. column indices) $j\in J$
and numerical values of corresponding constraint coefficients $a_j$
should be placed in locations \verb|ind[1]|, \dots, \verb|ind[len]| and
\verb|val[1]|, \dots, \verb|val[len]|, respectively, where
${\tt len}=|J|$ is the number of constraint coefficients,
$0\leq{\tt len}\leq n$, and $n$ is the number of columns in the problem
object. Coefficients with identical column indices are not allowed.
Zero coefficients are allowed, however, they are ignored.

The parameter \verb|type| specifies the constraint type as follows:

\verb|GLP_LO| means inequality constraint $\Sigma a_jx_j\geq b$;

\verb|GLP_UP| means inequality constraint $\Sigma a_jx_j\leq b$;

\newpage

The parameter \verb|rhs| specifies the right-hand side $b$.

All cutting plane constraints in the cut pool are identified by their
ordinal numbers 1, 2, \dots, $size$, where $size$ is the current size
of the cut pool. New constraints are always added to the end of the cut
pool, thus, ordinal numbers of previously added constraints are not
changed.

\returns

The routine \verb|glp_ios_add_row| returns the ordinal number of the
cutting plane constraint added, which is the new size of the cut pool.

\para{Example}

\begin{verbatim}
/* generate triangle cutting plane:
   x[i] + x[j] + x[k] <= 1 */
. . .
/* add the constraint to the cut pool */
ind[1] = i, val[1] = 1.0;
ind[2] = j, val[2] = 1.0;
ind[3] = k, val[3] = 1.0;
glp_ios_add_row(tree, NULL, TRIANGLE_CUT, 0, 3, ind, val, GLP_UP, 1.0);
\end{verbatim}

\para{Comments}

Cutting plane constraints added to the cut pool are intended to be then
added only to the {\it current} subproblem, so these constraints can be
globally as well as locally valid. However, adding a constraint to the
cut pool does not mean that it will be added to the current
subproblem---it depends on the solver's decision: if the constraint
seems to be efficient, it is moved from the pool to the current
subproblem, otherwise it is simply dropped.\footnote{Globally valid
constraints could be saved and then re-used for other subproblems, but
currently such feature is not implemented.}

Normally, every time the callback routine is called for cut generation,
the cut pool is empty. On the other hand, the solver itself can
generate cutting plane constraints (like Gomory's or mixed integer
rounding cuts), in which case the cut pool may be non-empty.

\subsection{glp\_ios\_del\_row --- remove constraint from the cut pool}

\synopsis

\begin{verbatim}
   void glp_ios_del_row(glp_tree *T, int i);
\end{verbatim}

\description

The routine \verb|glp_ios_del_row| deletes $i$-th row (cutting plane
constraint) from the cut pool, where $1\leq i\leq size$ is the ordinal
number of the constraint in the pool, $size$ is the current size of the
cut pool.

Note that deleting a constraint from the cut pool leads to changing
ordinal numbers of other constraints remaining in the pool. New ordinal
numbers of the remaining constraints are assigned under assumption that
the original order of constraints is not changed. Let, for example,
there be four constraints $a$, $b$, $c$ and $d$ in the cut pool, which
have ordinal numbers 1, 2, 3 and 4, respectively, and let constraint
$b$ have been deleted. Then after deletion the remaining constraint $a$,
$c$ and $d$ are assigned new ordinal numbers 1, 2 and 3, respectively.

To find the constraint to be deleted the routine \verb|glp_ios_del_row|
uses ``smart'' linear search, so it is recommended to remove
constraints in a natural or reverse order and avoid removing them in
a random order.

\para{Example}

\begin{verbatim}
/* keep first 10 constraints in the cut pool and remove other
   constraints */
while (glp_ios_pool_size(tree) > 10)
   glp_ios_del_row(tree, glp_ios_pool_size(tree));
\end{verbatim}

\subsection{glp\_ios\_clear\_pool --- remove all constraints from the
cut pool}

\synopsis

\begin{verbatim}
   void glp_ios_clear_pool(glp_tree *T);
\end{verbatim}

\description

The routine \verb|glp_ios_clear_pool| makes the cut pool empty deleting
all existing rows (cutting plane constraints) from it.

%* eof *%


%* glpk06.tex *%

\chapter{Miscellaneous API Routines}

\section{GLPK environment routines}

\subsection{glp\_init\_env --- initialize GLPK environment}

\synopsis

\begin{verbatim}
   int glp_init_env(void);
\end{verbatim}

\description

The routine \verb|glp_init_env| initializes the GLPK environment.
Normally the application program does not need to call this routine,
because it is called automatically on the first call to any API
routine.

\returns

\begin{retlist}
0 & initialization successful;\\
1 & environment is already initialized;\\
2 & initialization failed (insufficient memory);\\
3 & initialization failed (unsupported programming model).\\
\end{retlist}

\subsection{glp\_version --- determine library version}

\synopsis

\begin{verbatim}
   const char *glp_version(void);
\end{verbatim}

\returns

The routine \verb|glp_version| returns a pointer to a null-terminated
character string, which specifies the version of the GLPK library in
the form \verb|"X.Y"|, where `\verb|X|' is the major version number,
and `\verb|Y|' is the minor version number, for example, \verb|"4.16"|.

\newpage

\subsection{glp\_free\_env --- free GLPK environment}

\synopsis

\begin{verbatim}
   int glp_free_env(void);
\end{verbatim}

\description

The routine \verb|glp_free_env| frees all resources used by GLPK
routines (memory blocks, etc.) which are currently still in use.

Normally the application program does not need to call this routine,
because GLPK routines always free all unused resources. However, if
the application program even has deleted all problem objects, there
will be several memory blocks still allocated for the internal library
needs. For some reasons the application program may want GLPK to free
this memory, in which case it should call \verb|glp_free_env|.

Note that a call to \verb|glp_free_env| invalidates all problem objects
which still exist.

\returns

\begin{retlist}
0 & termination successful;\\
1 & environment is inactive (was not initialized).\\
\end{retlist}

\subsection{glp\_printf --- write formatted output to terminal}

\synopsis

\begin{verbatim}
   void glp_printf(const char *fmt, ...);
\end{verbatim}

\description

The routine \verb|glp_printf| uses the format control string
\verb|fmt| to format its parameters and writes the formatted output to
the terminal.

This routine is a replacement of the standard C function
\verb|printf| and used by all GLPK routines to perform terminal
output. The application program may use \verb|glp_printf| for the same
purpose that allows controlling its terminal output with the routines
\verb|glp_term_out| and \verb|glp_term_hook|.

\subsection{glp\_vprintf --- write formatted output to terminal}

\synopsis

\begin{verbatim}
   void glp_vprintf(const char *fmt, va_list arg);
\end{verbatim}

\description

The routine \verb|glp_vprintf| uses the format control string
\verb|fmt| to format its parameters specified by the list \verb|arg|
and writes the formatted output to the terminal.

This routine is a replacement of the standard C function
\verb|vprintf| and used by all GLPK routines to perform terminal
output. The application program may use \verb|glp_vprintf| for the same
purpose that allows controlling its terminal output with the routines
\verb|glp_term_out| and \verb|glp_term_hook|.

\newpage

\subsection{glp\_term\_out --- enable/disable terminal output}

\synopsis

\begin{verbatim}
   int glp_term_out(int flag);
\end{verbatim}

\description

Depending on the parameter flag the routine \verb|glp_term_out| enables
or disables terminal output performed by glpk routines:

\verb|GLP_ON | --- enable terminal output;

\verb|GLP_OFF| --- disable terminal output.

\returns

The routine \verb|glp_term_out| returns the previous value of the
terminal output flag.

\subsection{glp\_term\_hook --- intercept terminal output}

\synopsis

\begin{verbatim}
   void glp_term_hook(int (*func)(void *info, const char *s), void *info);
\end{verbatim}

\description

The routine \verb|glp_term_hook| installs the user-defined hook routine
to intercept all terminal output performed by GLPK routines.

%This feature can be used to redirect the terminal output to other
%destination, for example, to a file or a text window.

The parameter {\it func} specifies the user-defined hook routine. It is
called from an internal printing routine, which passes to it two
parameters: {\it info} and {\it s}. The parameter {\it info} is a
transit pointer specified in corresponding call to the routine
\verb|glp_term_hook|; it may be used to pass some additional information
to the hook routine. The parameter {\it s} is a pointer to the null
terminated character string, which is intended to be written to the
terminal. If the hook routine returns zero, the printing routine writes
the string {\it s} to the terminal in a usual way; otherwise, if the
hook routine returns non-zero, no terminal output is performed.

To uninstall the hook routine both parameters {\it func} and {\it info}
should be specified as \verb|NULL|.

\para{Example}

\begin{footnotesize}
\begin{verbatim}
static int hook(void *info, const char *s)
{     FILE *foo = info;
      fputs(s, foo);
      return 1;
}

int main(void)
{     FILE *foo;
      . . .
      glp_term_hook(hook, foo); /* redirect terminal output */
      . . .
      glp_term_hook(NULL, NULL); /* resume terminal output */
      . . .
}
\end{verbatim}
\end{footnotesize}

\newpage

\subsection{glp\_open\_tee --- start copying terminal output}

\synopsis

\begin{verbatim}
   int glp_open_tee(const char *fname);
\end{verbatim}

\description

The routine \verb|glp_open_tee| starts copying all the terminal output
to an output text file, whose name is specified by the character string
\verb|fname|.

\returns

\begin{retlist}
0 & operation successful;\\
1 & copying terminal output is already active;\\
2 & unable to create output file.\\
\end{retlist}

\subsection{glp\_close\_tee --- stop copying terminal output}

\synopsis

\begin{verbatim}
   int glp_close_tee(void);
\end{verbatim}

\description

The routine \verb|glp_close_tee| stops copying the terminal output to
the output text file previously open by the routine \verb|glp_open_tee|
closing that file.

\returns

\begin{retlist}
0 & operation successful;\\
1 & copying terminal output was not started.\\
\end{retlist}

\subsection{glp\_error --- display error message and terminate
execution}

\synopsis

\begin{verbatim}
   void glp_error(const char *fmt, ...);
\end{verbatim}

\description

The routine \verb|glp_error| (implemented as a macro) formats its
parameters using the format control string \verb|fmt|, writes the
formatted message to the terminal, and then abnormally terminates the
program.

\newpage

\subsection{glp\_at\_error --- check for error state}

\synopsis

\begin{verbatim}
   int glp_at_error(void);
\end{verbatim}

\description

The routine \verb|glp_at_error| checks if the GLPK environment is at
error state, i.~e.~if the call to the routine is (indirectly) made from
the \verb|glp_error| routine via an user-defined hook routine.

This routine can be used, for example, by a custom output handler
(installed with the routine \verb|glp_term_hook|) to determine whether
or not the message to be displayed is an error message.

\returns

If the GLPK environment is at error state, the routine returns
non-zero, otherwise zero.

\subsection{glp\_assert --- check logical condition}

\synopsis

\begin{verbatim}
   void glp_assert(int expr);
\end{verbatim}

\description

The routine \verb|glp_assert| (implemented as a macro) checks
a logical condition specified by the expression \verb|expr|. If the
condition is true (non-zero), the routine does nothing; otherwise, if
the condition is false (zero), the routine prints an error message and
abnormally terminates the program.

This routine is a replacement of the standard C function \verb|assert|
and used by all GLPK routines to check program logic. The application
program may use \verb|glp_assert| for the same purpose.

\subsection{glp\_error\_hook --- install hook to intercept abnormal
termination}

\synopsis

\begin{verbatim}
   void glp_error_hook(void (*func)(void *info), void *info);
\end{verbatim}

\description

The routine \verb|glp_error_hook| installs a user-defined hook routine
to intercept abnormal termination.

The parameter \verb|func| specifies the user-defined hook routine. It
is called from the routine \verb|glp_error| before the latter calls the
abort function to abnormally terminate the application program because
of fatal error. The parameter \verb|info| is a transit pointer,
specified in the corresponding call to the routine
\verb|glp_error_hook|; it may be used to pass some information to the
hook routine.

To uninstall the hook routine the parameters \verb|func| and \verb|info|
should be specified as \verb|NULL|.

If the hook routine returns, the application program is abnormally
terminated. To prevent abnormal termnation the hook routine may perform
a global jump using the standard function \verb|longjmp|, in which case
the application program {\it must} call the routine \verb|glp_free_env|.

\subsection{glp\_alloc --- allocate memory block}

\synopsis

\begin{verbatim}
   void *glp_alloc(int n, int size);
\end{verbatim}

\description

The routine \verb|glp_alloc| dynamically allocates a memory block of
\verb|n|$\times$\verb|size| bytes long. Note that:

1) the parameters \verb|n| and \verb|size| must be positive;

2) having been allocated the memory block contains arbitrary data, that
is, it is {\it not} initialized by binary zeros;

3) if the block cannot be allocated due to insufficient memory, the
routine prints an error message and abnormally terminates the program.

This routine is a replacement of the standard C function \verb|malloc|
and used by GLPK routines for dynamic memory allocation. The
application program may use \verb|glp_alloc| for the same purpose.

\returns

The routine \verb|glp_alloc| returns a pointer to the memory block
allocated. To free this block the routine \verb|glp_free| (not the
standard C function \verb|free|!) should be used.

\subsection{glp\_realloc --- reallocate memory block}

\synopsis

\begin{verbatim}
   void *glp_realloc(void *ptr, int n, int size);
\end{verbatim}

\description

The routine \verb|glp_realloc| dynamically reallocates a memory block
pointed to by \verb|ptr|, which was previously allocated by the routine
\verb|glp_alloc| or reallocated by this routine. Note that the pointer
\verb|ptr| must be valid and must not be \verb|NULL|. The new size of
the memory block is \verb|n|$\times$\verb|size| bytes long. Note that:

1) both parameters \verb|n| and \verb|size| must be positive;

2) if the block cannot be reallocated due to insufficient memory, the
routine prints an error message and abnormally terminates the program.

This routine is a replacement of the standard C function \verb|realloc|
and used by GLPK routines for dynamic memory allocation. The
application program may use \verb|glp_realloc| for the same purpose.

\returns

The routine \verb|glp_realloc| returns a pointer to the memory block
reallocated. To free this block the routine \verb|glp_free| (not the
standard C function \verb|free|!) should be used.

\newpage

\subsection{glp\_free --- free memory block}

\synopsis

\begin{verbatim}
   void glp_free(void *ptr);
\end{verbatim}

\description

The routine \verb|glp_free| deallocates a memory block pointed to by
\verb|ptr|, which was previously allocated by the routine
\verb|glp_malloc| or reallocated by the routine \verb|glp_realloc|.
Note that the pointer \verb|ptr| must be valid and must not be
\verb|NULL|.

This routine is a replacement of the standard C function \verb|free|
and used by GLPK routines for dynamic memory allocation. The
application program may use \verb|glp_free| for the same purpose.

\subsection{glp\_mem\_usage --- get memory usage information}

\synopsis

\begin{verbatim}
   void glp_mem_usage(int *count, int *cpeak, size_t *total, size_t *tpeak);
\end{verbatim}

\description

The routine \verb|glp_mem_usage| reports some information about
utilization of the memory by the routines \verb|glp_malloc|,
\verb|glp_calloc|, and \verb|glp_free|. Information is stored to
locations specified by corresponding parameters (see below). Any
parameter can be specified as \verb|NULL|, in which case corresponding
information is not stored.

\verb|*count| is the number of currently allocated memory blocks.

\verb|*cpeak| is the peak value of \verb|*count| reached since the
initialization of the GLPK library environment.

\verb|*total| is the total amount, in bytes, of currently allocated
memory blocks.

\verb|*tpeak| is the peak value of \verb|*total| reached since the
initialization of the GLPK library envirionment.

\para{Example}

\begin{footnotesize}
\begin{verbatim}
glp_mem_usage(&count, NULL, NULL, NULL);
printf("%d memory block(s) are still allocated\n", count);
\end{verbatim}
\end{footnotesize}

\subsection{glp\_mem\_limit --- set memory usage limit}

\synopsis

\begin{verbatim}
   void glp_mem_limit(int limit);
\end{verbatim}

\description

The routine \verb|glp_mem_limit| limits the amount of memory available
for dynamic allocation (with the routines \verb|glp_malloc| and
\verb|glp_calloc|) to \verb|limit| megabytes.

%* eof *%


\appendix

%* glpk07.tex *%

\chapter{Installing GLPK on Your Computer}
\label{install}

\section{Downloading the distribution tarball}

The distribution tarball of the most recent version of the GLPK
package can be found on \url{http://ftp.gnu.org/gnu/glpk/} [via http]
and \url{ftp://ftp.gnu.org/gnu/glpk/} [via FTP]. It can also be found
on one of the FTP mirrors; see \url{http://www.gnu.org/prep/ftp.html}.
Please use a mirror if possible.

To make sure that the GLPK distribution tarball you have downloaded is
intact you need to download the corresponding `\verb|.sig|' file and
run a command like this:

\begin{verbatim}
   gpg --verify glpk-4.38.tar.gz.sig
\end{verbatim}

\noindent
If that command fails because you do not have the required public key,
run the following command to import it:

\begin{verbatim}
   gpg --keyserver keys.gnupg.net --recv-keys 5981E818
\end{verbatim}

\noindent
and then re-run the previous command.

\section{Unpacking the distribution tarball}

The GLPK package (like all other GNU software) is distributed in the
form of packed archive. This is one file named \verb|glpk-X.Y.tar.gz|,
where {\it X} is the major version number and {\it Y} is the minor
version number.

In order to prepare the distribution for installation you need to copy
the GLPK distribution file to a working subdirectory and then unpack
and unarchive the distribution file with the following command:

\begin{verbatim}
   tar zx < glpk-X.Y.tar
\end{verbatim}

\newpage

\section{Configuring the package}

After unpacking and unarchiving the GLPK distribution you should
configure the package,\linebreak i.e. automatically tune it for your
platform.

Normally, you should just \verb|cd| to the subdirectory \verb|glpk-X.Y|
and run the configure script, e.g.

\begin{verbatim}
   ./configure
\end{verbatim}

The `\verb|configure|' shell script attempts to guess correct values
for various system-dependent variables used during compilation. It uses
those values to create a `\verb|Makefile|' in each directory of the
package. It also creates file `\verb|config.h|' containing
platform-dependent definitions. Finally, it creates a shell script
`\verb|config.status|' that you can run in the future to recreate the
current configuration, a file `\verb|config.cache|' that saves the
results of its tests to speed up reconfiguring, and a file
`\verb|config.log|' containing compiler output (useful mainly for
debugging `\verb|configure|').

Running `\verb|configure|' takes about a minute. While it is running,
it displays some informational messages that tell you what it
is doing. If you don't want to see these messages, run
`\verb|configure|' with its standard output redirected to
`\verb|dev/null|'; for example, `\verb|./configure > /dev/null|'.

By default both static and shared versions of the GLPK library will be
compiled. Compilation of the shared librariy can be turned off by
specifying the `\verb|--disable-shared|' option to `\verb|configure|':

\begin{verbatim}
   ./configure --disable-shared
\end{verbatim}

\noindent
If you encounter problems building the library try using the above
option, because some platforms do not support shared libraries.

The GLPK package has some optional features listed below. By default
all these features are disabled. To enable a feature the corresponding
option should be passed to the configure script.

\verb|--with-gmp         | Enable using the GNU MP bignum library

This feature allows the exact simplex solver to use the GNU MP bignum
library. If it is disabled, the exact simplex solver uses the GLPK
bignum module, which provides the same functionality as GNU MP, however,
it is much less efficient.

For details about the GNU MP bignum library see its web page at
\url{http://gmplib.org/}.

\verb|--enable-dl        | The same as `\verb|--enable-dl=ltdl|'

\verb|--enable-dl=ltdl   | Enable shared library support (GNU)

\verb|--enable-dl=dlfcn  | Enable shared library support (POSIX)

Currently this feature is only needed to provide dynamic linking to
ODBC and MySQL shared libraries (see below).

For details about the GNU shared library support see the manual at
\url{http://www.gnu.org/software/libtool/manual/}.

\verb|--enable-odbc      |
Enable using ODBC table driver (\verb|libiodbc|)

\verb|--enable-odbc=unix |
Enable using ODBC table driver (\verb|libodbc|)

This feature allows transmitting data between MathProg model objects
and relational databases accessed through ODBC.

For more details about this feature see the supplement ``Using Data
Tables in the GNU MathProg Modeling Language'' (\verb|doc/tables.pdf|).

\verb|--enable-mysql     |
Enable using MySQL table driver (\verb|libmysql|)

This feature allows transmitting data between MathProg model objects
and MySQL relational databases.

For more details about this feature see the supplement ``Using Data
Tables in the GNU MathProg Modeling Language'' (\verb|doc/tables.pdf|).

\section{Compiling the package}

Normally, you can compile (build) the package by typing the command:

\begin{verbatim}
   make
\end{verbatim}

\noindent
It reads `\verb|Makefile|' generated by `\verb|configure|' and performs
all necessary jobs.

If you want, you can override the `\verb|make|' variables \verb|CFLAGS|
and \verb|LDFLAGS| like this:

\begin{verbatim}
   make CFLAGS=-O2 LDFLAGS=-s
\end{verbatim}

To compile the package in a different directory from the one containing
the source code, you must use a version of `\verb|make|' that supports
`\verb|VPATH|' variable, such as GNU `\verb|make|'. `\verb|cd|' to the
directory where you want the object files and executables to go and run
the `\verb|configure|' script. `\verb|configure|' automatically checks
for the source code in the directory that `\verb|configure|' is in and
in `\verb|..|'. If for some reason `\verb|configure|' is not in the
source code directory that you are configuring, then it will report
that it can't find the source code. In that case, run `\verb|configure|'
with the option `\verb|--srcdir=DIR|', where \verb|DIR| is the
directory that contains the source code.

Some systems require unusual options for compilation or linking that
the `\verb|configure|' script does not know about. You can give
`\verb|configure|' initial values for variables by setting them in the
environment. Using a Bourne-compatible shell, you can do that on the
command line like this:

\begin{verbatim}
   CC=c89 CFLAGS=-O2 LIBS=-lposix ./configure
\end{verbatim}

\noindent
Or on systems that have the `\verb|env|' program, you can do it like
this:

\begin{verbatim}
   env CPPFLAGS=-I/usr/local/include LDFLAGS=-s ./configure
\end{verbatim}

Here are the `\verb|make|' variables that you might want to override
with environment variables when running `\verb|configure|'.

For these variables, any value given in the environment overrides the
value that `\verb|configure|' would choose:

\verb|CC      | C compiler program. The default is `\verb|cc|'.

\verb|INSTALL | Program used to install files. The default value is
`\verb|install|' if you have it,\\
\verb|           | otherwise `\verb|cp|'.

For these variables, any value given in the environment is added to the
value that `\verb|configure|' chooses:

\verb|DEFS    | Configuration options, in the form
`\verb|-Dfoo -Dbar| \dots'.

\verb|LIBS    | Libraries to link with, in the form
`\verb|-lfoo -lbar| \dots'.

\section{Checking the package}

To check the package, i.e. to run some tests included in the package,
you can use the following command:

\begin{verbatim}
   make check
\end{verbatim}

\section{Installing the package}

Normally, to install the GLPK package you should type the following
command:

\begin{verbatim}
   make install
\end{verbatim}

\noindent
By default, `\verb|make install|' will install the package's files in
`\verb|usr/local/bin|', `\verb|usr/local/lib|', etc. You can specify an
installation prefix other than `\verb|/usr/local|' by giving
`\verb|configure|' the option `\verb|--prefix=PATH|'. Alternately, you
can do so by consistently giving a value for the `\verb|prefix|'
variable when you run `\verb|make|', e.g.

\begin{verbatim}
   make prefix=/usr/gnu
   make prefix=/usr/gnu install
\end{verbatim}

After installing you can remove the program binaries and object files
from the source directory by typing `\verb|make clean|'. To remove all
files that `\verb|configure|' created (`\verb|Makefile|',
`\verb|config.status|', etc.), just type `\verb|make distclean|'.

The file `\verb|configure.ac|' is used to create `\verb|configure|' by
a program called `\verb|autoconf|'. You only need it if you want to
remake `\verb|configure|' using a newer version of `\verb|autoconf|'.

\section{Uninstalling the package}

To uninstall the GLPK package, i.e. to remove all the package's files
from the system places, you can use the following command:

\begin{verbatim}
   make uninstall
\end{verbatim}

%* eof *%


%* glpk08.tex *%

\chapter{MPS Format}
\label{champs}

\section{Fixed MPS Format}
\label{secmps}

The MPS format\footnote{The MPS format was developed in 1960's by IBM
as input format for their mathematical programming system MPS/360.
Today the MPS format is a most widely used format understood by most
mathematical programming packages. This appendix describes only the
features of the MPS format, which are implemented in the GLPK package.}
is intended for coding LP/MIP problem data. This format assumes the
formulation of LP/MIP problem (1.1)---(1.3) (see Section \ref{seclp},
page \pageref{seclp}).

{\it MPS file} is a text file, which contains two types of
cards\footnote{In 1960's MPS file was a deck of 80-column punched
cards, so the author decided to keep the word ``card'', which may be
understood as ``line of text file''.}: indicator cards and data cards.

Indicator cards determine a kind of succeeding data. Each indicator
card has one word in uppercase letters beginning in column 1.

Data cards contain problem data. Each data card is divided into six
fixed fields:

\begin{center}
\begin{tabular}{lcccccc}
& Field 1 & Field 2 & Field 3 & Field 4 & Field 5 & Field 6 \\
\hline
Columns & 2---3 & 5---12 & 15---22 & 25---36 & 40---47 & 50---61 \\
Contents & Code & Name & Name & Number & Name & Number \\
\end{tabular}
\end{center}

On a particular data card some fields may be optional.

Names are used to identify rows, columns, and some vectors (see below).

Aligning the indicator code in the field 1 to the left margin is
optional.

All names specified in the fields 2, 3, and 5 should contain from 1 up
to 8 arbitrary characters (except control characters). If a name is
placed in the field 3 or 5, its first character should not be the dollar
sign `\verb|$|'. If a name contains spaces, the spaces are ignored.

All numerical values in the fields 4 and 6 should be coded in the form
$sxx$\verb|E|$syy$, where $s$ is the plus `\verb|+|' or the minus
`\verb|-|' sign, $xx$ is a real number with optional decimal point,
$yy$ is an integer decimal exponent. Any number should contain up to 12
characters. If the sign $s$ is omitted, the plus sign is assumed. The
exponent part is optional. If a number contains spaces, the spaces are
ignored.

%\newpage

If a card has the asterisk `\verb|*|' in the column 1, this card is
considered as a comment and ignored. Besides, if the first character in
the field 3 or 5 is the dollar sign `\verb|$|', all characters from the
dollar sign to the end of card are considered as a comment and ignored.

MPS file should contain cards in the following order:

%\vspace*{-8pt}

%\begin{itemize}
\Item{---}NAME indicator card;

\Item{---}ROWS indicator card;

\Item{---}data cards specifying rows (constraints);

\Item{---}COLUMNS indicator card;

\Item{---}data cards specifying columns (structural variables) and
constraint coefficients;

\Item{---}RHS indicator card;

\Item{---}data cards specifying right-hand sides of constraints;

\Item{---}RANGES indicator card;

\Item{---}data cards specifying ranges for double-bounded constraints;

\Item{---}BOUNDS indicator card;

\Item{---}data cards specifying types and bounds of structural
variables;

\Item{---}ENDATA indicator card.
%\end{itemize}

%\vspace*{-8pt}

{\it Section} is a group of cards consisting of an indicator card and
data cards succeeding this indicator card. For example, the ROWS section
consists of the ROWS indicator card and data cards specifying rows.

The sections RHS, RANGES, and BOUNDS are optional and may be omitted.

\section{Free MPS Format}

{\it Free MPS format} is an improved version of the standard (fixed)
MPS format described above.\footnote{This format was developed in the
beginning of 1990's by IBM as an alternative to the standard fixed MPS
format for Optimization Subroutine Library (OSL).} Note that all
changes in free MPS format concern only the coding of data while the
structure of data is the same for both fixed and free versions of the
MPS format.

In free MPS format indicator and data records\footnote{{\it Record} in
free MPS format has the same meaning as {\it card} in fixed MPS format.}
may have arbitrary length not limited to 80 characters. Fields of data
records have no predefined positions, i.e. the fields may begin in any
position, except position 1, which must be blank, and must be separated
from each other by one or more blanks. However, the fields must appear
in the same order as in fixed MPS format.

%\newpage

Symbolic names in fields 2, 3, and 5 may be longer than 8
characters\footnote{GLPK allows symbolic names having up to 255
characters.} and must not contain embedded blanks.

Numeric values in fields 4 and 6 are limited to 12 characters and must
not contain embedded blanks.

Only six fields on each data record are used. Any other fields are
ignored.

If the first character of any field (not necessarily fields 3 and 5)
is the dollar sign (\$), all characters from the dollar sign to the end
of record are considered as a comment and ignored.

\newpage

\section{NAME indicator card}

The NAME indicator card should be the first card in the MPS file
(except optional comment cards, which may precede the NAME card). This
card should contain the word \verb|NAME| in the columns 1---4 and the
problem name in the field 3. The problem name is optional and may be
omitted.

\section{ROWS section}
\label{secrows}

The ROWS section should start with the indicator card, which contains
the word \verb|ROWS| in the columns 1---4.

Each data card in the ROWS section specifies one row (constraint) of
the problem. All these data cards have the following format.

`\verb|N|' in the field 1 means that the row is free (unbounded):
$$-\infty < x_i = a_{i1}x_{m+1} + a_{i2}x_{m+2} + \dots + a_{in}x_{m+n}
< +\infty;$$

`\verb|L|' in the field 1 means that the row is of ``less than or equal
to'' type:
$$-\infty < x_i = a_{i1}x_{m+1} + a_{i2}x_{m+2} + \dots + a_{in}x_{m+n}
\leq b_i;$$

`\verb|G|' in the field 1 means that the row is of ``greater than or
equal to'' type:
$$b_i \leq x_i = a_{i1}x_{m+1} + a_{i2}x_{m+2} + \dots + a_{in}x_{m+n}
< +\infty;$$

`\verb|E|' in the field 1 means that the row is of ``equal to'' type:
$$x_i = a_{i1}x_{m+1} + a_{i2}x_{m+2} + \dots + a_{in}x_{m+n} \leq
b_i,$$
where $b_i$ is a right-hand side. Note that each constraint has a
corresponding implictly defined auxiliary variable ($x_i$ above), whose
value is a value of the corresponding linear form, therefore row bounds
can be considered as bounds of such auxiliary variable.

The filed 2 specifies a row name (which is considered as the name of
the corresponding auxiliary variable).

%\newpage

The fields 3, 4, 5, and 6 are not used and should be empty.

Numerical values of all non-zero right-hand sides $b_i$ should be
specified in the RHS section (see below). All double-bounded (ranged)
constraints should be specified in the RANGES section (see below).

\section{COLUMNS section}

The COLUMNS section should start with the indicator card, which
contains the word \verb|COLUMNS| in the columns 1---7.

Each data card in the COLUMNS section specifies one or two constraint
coefficients $a_{ij}$ and also introduces names of columns, i.e. names
of structural variables. All these data cards have the following
format.

The field 1 is not used and should be empty.

The field 2 specifies a column name. If this field is empty, the column
name from the immediately preceeding data card is assumed.

The field 3 specifies a row name defined in the ROWS section.

The field 4 specifies a numerical value of the constraint coefficient
$a_{ij}$, which is placed in the corresponding row and column.

The fields 5 and 6 are optional. If they are used, they should contain
a second pair ``row name---constraint coefficient'' for the same column.

Elements of the constraint matrix (i.e. constraint coefficients) should
be enumerated in the column wise manner: all elements for the current
column should be specified before elements for the next column. However,
the order of rows in the COLUMNS section may differ from the order of
rows in the ROWS section.

Constraint coefficients not specified in the COLUMNS section are
considered as zeros. Therefore zero coefficients may be omitted,
although it is allowed to explicitly specify them.

\section{RHS section}

The RHS section should start with the indicator card, which contains the
word \verb|RHS| in the columns 1---3.

Each data card in the RHS section specifies one or two right-hand sides
$b_i$ (see Section \ref{secrows}, page \pageref{secrows}). All these
data cards have the following format.

The field 1 is not used and should be empty.

The field 2 specifies a name of the right-hand side (RHS)
vector\footnote{This feature allows the user to specify several RHS
vectors in the same MPS file. However, before solving the problem a
particular RHS vector should be chosen.}. If this field is empty, the
RHS vector name from the immediately preceeding data card is assumed.

%\newpage

The field 3 specifies a row name defined in the ROWS section.

The field 4 specifies a right-hand side $b_i$ for the row, whose name is
specified in the field 3. Depending on the row type $b_i$ is a lower
bound (for the row of \verb|G| type), an upper bound (for the row of
\verb|L| type), or a fixed value (for the row of \verb|E|
type).\footnote{If the row is of {\tt N} type, $b_i$ is considered as
a constant term of the corresponding linear form. Should note, however,
this convention is non-standard.}

The fields 5 and 6 are optional. If they are used, they should contain
a second pair ``row name---right-hand side'' for the same RHS vector.

All right-hand sides for the current RHS vector should be specified
before right-hand sides for the next RHS vector. However, the order of
rows in the RHS section may differ from the order of rows in the ROWS
section.

Right-hand sides not specified in the RHS section are considered as
zeros. Therefore zero right-hand sides may be omitted, although it is
allowed to explicitly specify them.

\newpage

\section{RANGES section}

The RANGES section should start with the indicator card, which contains
the word \verb|RANGES| in the columns 1---6.

Each data card in the RANGES section specifies one or two ranges for
double-side constraints, i.e. for constraints that are of the types
\verb|L| and \verb|G| at the same time:
$$l_i \leq x_i = a_{i1}x_{m+1} + a_{i2}x_{m+2} + \dots + a_{in}x_{m+n}
\leq u_i,$$
where $l_i$ is a lower bound, $u_i$ is an upper bound. All these data
cards have the following format.

The field 1 is not used and should be empty.

The field 2 specifies a name of the range vector\footnote{This feature
allows the user to specify several range vectors in the same MPS file.
However, before solving the problem a particular range vector should be
chosen.}. If this field is empty, the range vector name from the
immediately preceeding data card is assumed.

The field 3 specifies a row name defined in the ROWS section.

The field 4 specifies a range value $r_i$ (see the table below) for the
row, whose name is specified in the field 3.

The fields 5 and 6 are optional. If they are used, they should contain
a second pair ``row name---range value'' for the same range vector.

All range values for the current range vector should be specified before
range values for the next range vector. However, the order of rows in
the RANGES section may differ from the order of rows in the ROWS
section.

For each double-side constraint specified in the RANGES section its
lower and upper bounds are determined as follows:

%\newpage

\begin{center}
\begin{tabular}{cccc}
Row type & Sign of $r_i$ & Lower bound & Upper bound \\
\hline
{\tt G} & $+$ or $-$ & $b_i$ & $b_i + |r_i|$ \\
{\tt L} & $+$ or $-$ & $b_i - |r_i|$ & $b_i$ \\
{\tt E} & $+$ & $b_i$ & $b_i + |r_i|$ \\
{\tt E} & $-$ & $b_i - |r_i|$ & $b_i$ \\
\end{tabular}
\end{center}

\noindent
where $b_i$ is a right-hand side specified in the RHS section (if $b_i$
is not specified, it is considered as zero), $r_i$ is a range value
specified in the RANGES section.

\section{BOUNDS section}
\label{secbounds}

The BOUNDS section should start with the indicator card, which contains
the word \verb|BOUNDS| in the columns 1---6.

Each data card in the BOUNDS section specifies one (lower or upper)
bound for one structural variable (column). All these data cards have
the following format.

The indicator in the field 1 specifies the bound type:

\verb|LO| --- lower bound;

\verb|UP| --- upper bound;

\verb|FX| --- fixed variable (lower and upper bounds are equal);

\verb|FR| --- free variable (no bounds);

\verb|MI| --- no lower bound (lower bound is ``minus infinity'');

\verb|PL| --- no upper bound (upper bound is ``plus infinity'').

The field 2 specifies a name of the bound vector\footnote{This feature
allows the user to specify several bound vectors in the same MPS file.
However, before solving the problem a particular bound vector should be
chosen.}. If this field is empty, the bound vector name from the
immediately preceeding data card is assumed.

The field 3 specifies a column name defined in the COLUMNS section.

The field 4 specifies a bound value. If the bound type in the field 1
differs from \verb|LO|, \verb|UP|, and \verb|FX|, the value in the field
4 is ignored and may be omitted.

The fields 5 and 6 are not used and should be empty.

All bound values for the current bound vector should be specified before
bound values for the next bound vector. However, the order of columns in
the BOUNDS section may differ from the order of columns in the COLUMNS
section. Specification of a lower bound should precede specification of
an upper bound for the same column (if both the lower and upper bounds
are explicitly specified).

By default, all columns (structural variables) are non-negative, i.e.
have zero lower bound and no upper bound. Lower ($l_j$) and upper
($u_j$) bounds of some column (structural variable $x_j$) are set in the
following way, where $s_j$ is a corresponding bound value explicitly
specified in the BOUNDS section:

%\newpage

\verb|LO| sets $l_j$ to $s_j$;

\verb|UP| sets $u_j$ to $s_j$;

\verb|FX| sets both $l_j$ and $u_j$ to $s_j$;

\verb|FR| sets $l_j$ to $-\infty$ and $u_j$ to $+\infty$;

\verb|MI| sets $l_j$ to $-\infty$;

\verb|PL| sets $u_j$ to $+\infty$.

\section{ENDATA indicator card}

The ENDATA indicator card should be the last card of MPS file (except
optional comment cards, which may follow the ENDATA card). This card
should contain the word \verb|ENDATA| in the columns 1---6.

\section{Specifying objective function}

It is impossible to explicitly specify the objective function and
optimization direction in the MPS file. However, the following implicit
rule is used by default: the first row of \verb|N| type is considered
as a row of the objective function (i.e. the objective function is the
corresponding auxiliary variable), which should be {\it minimized}.

GLPK also allows specifying a constant term of the objective function
as a right-hand side of the corresponding row in the RHS section.

\section{Example of MPS file}
\label{secmpsex}

To illustrate what the MPS format is, consider the following example of
LP problem:

\def\arraystretch{1.2}

\noindent\hspace{.5in}minimize
$$
value = .03\ bin_1 + .08\ bin_2 + .17\ bin_3 + .12\ bin_4 + .15\ bin_5
+ .21\ al + .38\ si
$$
\noindent\hspace{.5in}subject to linear constraints
$$
\begin{array}{@{}l@{\:}l@{}}
yield &= \ \ \ \ \;bin_1 + \ \ \ \ \;bin_2 + \ \ \ \ \;bin_3 +
         \ \ \ \ \;bin_4 + \ \ \ \ \;bin_5 + \ \ \ \ \;al +
         \ \ \ \ \;si \\
FE    &= .15\ bin_1 + .04\ bin_2 + .02\ bin_3 + .04\ bin_4 + .02\ bin_5
         + .01\ al + .03\ si \\
CU    &= .03\ bin_1 + .05\ bin_2 + .08\ bin_3 + .02\ bin_4 + .06\ bin_5
         + .01\ al \\
MN    &= .02\ bin_1 + .04\ bin_2 + .01\ bin_3 + .02\ bin_4 + .02\ bin_5
         \\
MG    &= .02\ bin_1 + .03\ bin_2
\ \ \ \ \ \ \ \ \ \ \ \ \ \ \ \ \ \ \ \ \ \ \ \ \ \ \ \ + .01\ bin_5 \\
AL    &= .70\ bin_1 + .75\ bin_2 + .80\ bin_3 + .75\ bin_4 + .80\ bin_5
         + .97\ al \\
SI    &= .02\ bin_1 + .06\ bin_2 + .08\ bin_3 + .12\ bin_4 + .02\ bin_5
         + .01\ al + .97\ si \\
\end{array}
$$
\noindent\hspace{.5in}and bounds of (auxiliary and structural)
variables
$$
\begin{array}{r@{\ }l@{\ }l@{\ }l@{\ }rcr@{\ }l@{\ }l@{\ }l@{\ }r}
&&yield&=&2000&&0&\leq&bin_1&\leq&200\\
-\infty&<&FE&\leq&60&&0&\leq&bin_2&\leq&2500\\
-\infty&<&CU&\leq&100&&400&\leq&bin_3&\leq&800\\
-\infty&<&MN&\leq&40&&100&\leq&bin_4&\leq&700\\
-\infty&<&MG&\leq&30&&0&\leq&bin_5&\leq&1500\\
1500&\leq&AL&<&+\infty&&0&\leq&al&<&+\infty\\
250&\leq&SI&\leq&300&&0&\leq&si&<&+\infty\\
\end{array}
$$

\def\arraystretch{1}

A complete MPS file which specifies data for this example is shown
below (the first two comment lines show card positions).

\newpage

\begin{footnotesize}
\begin{verbatim}
*000000001111111111222222222233333333334444444444555555555566
*234567890123456789012345678901234567890123456789012345678901
NAME          PLAN
ROWS
 N  VALUE
 E  YIELD
 L  FE
 L  CU
 L  MN
 L  MG
 G  AL
 L  SI
COLUMNS
    BIN1      VALUE           .03000   YIELD          1.00000
              FE              .15000   CU              .03000
              MN              .02000   MG              .02000
              AL              .70000   SI              .02000
    BIN2      VALUE           .08000   YIELD          1.00000
              FE              .04000   CU              .05000
              MN              .04000   MG              .03000
              AL              .75000   SI              .06000
    BIN3      VALUE           .17000   YIELD          1.00000
              FE              .02000   CU              .08000
              MN              .01000   AL              .80000
              SI              .08000
    BIN4      VALUE           .12000   YIELD          1.00000
              FE              .04000   CU              .02000
              MN              .02000   AL              .75000
              SI              .12000
    BIN5      VALUE           .15000   YIELD          1.00000
              FE              .02000   CU              .06000
              MN              .02000   MG              .01000
              AL              .80000   SI              .02000
    ALUM      VALUE           .21000   YIELD          1.00000
              FE              .01000   CU              .01000
              AL              .97000   SI              .01000
    SILICON   VALUE           .38000   YIELD          1.00000
              FE              .03000   SI              .97000
RHS
    RHS1      YIELD       2000.00000   FE            60.00000
              CU           100.00000   MN            40.00000
              SI           300.00000
              MG            30.00000   AL          1500.00000
RANGES
    RNG1      SI            50.00000
BOUNDS
 UP BND1      BIN1         200.00000
 UP           BIN2        2500.00000
 LO           BIN3         400.00000
 UP           BIN3         800.00000
 LO           BIN4         100.00000
 UP           BIN4         700.00000
 UP           BIN5        1500.00000
ENDATA
\end{verbatim}
\end{footnotesize}

%\vspace*{-6pt}

\section{MIP features}

%\vspace*{-4pt}

The MPS format provides two ways for introducing integer variables into
the problem.

The first way is most general and based on using special marker cards
INTORG and INTEND. These marker cards are placed in the COLUMNS section.
The INTORG card indicates the start of a group of integer variables
(columns), and the card INTEND indicates the end of the group. The MPS
file may contain arbitrary number of the marker cards.

The marker cards have the same format as the data cards (see Section
\ref{secmps}, page \pageref{secmps}).

The fields 1, 2, and 6 are not used and should be empty.

The field 2 should contain a marker name. This name may be arbitrary.

The field 3 should contain the word \verb|'MARKER'| (including
apostrophes).

The field 5 should contain either the word \verb|'INTORG'| (including
apostrophes) for the marker card, which begins a group of integer
columns, or the word \verb|'INTEND'| (including apostrophes) for the
marker card, which ends the group.

The second way is less general but more convenient in some cases. It
allows the user declaring integer columns using three additional types
of bounds, which are specified in the field 1 of data cards in the
BOUNDS section (see Section \ref{secbounds}, page \pageref{secbounds}):

\verb|LI| --- lower integer. This bound type specifies that the
corresponding column (structural variable), whose name is specified in
field 3, is of integer kind. In this case an lower bound of the
column should be specified in field 4 (like in the case of \verb|LO|
bound type).

\verb|UI| --- upper integer. This bound type specifies that the
corresponding column (structural variable), whose name is specified in
field 3, is of integer kind. In this case an upper bound of the
column should be specified in field 4 (like in the case of \verb|UP|
bound type).

\verb|BV| --- binary variable. This bound type specifies that the
corresponding column (structural variable), whose name is specified in
the field 3, is of integer kind, its lower bound is zero, and its upper
bound is one (thus, such variable being of integer kind can have only
two values zero and one). In this case a numeric value specified in the
field 4 is ignored and may be omitted.

Consider the following example of MIP problem:

\noindent
\hspace{1in} minimize
$$Z = 3 x_1 + 7 x_2 - x_3 + x4$$
\hspace{1in} subject to linear constraints
$$
\begin{array}{c}
\nonumber r_1 = 2   x_1 - \ \ x_2 + \ \ x_3 - \ \;x_4 \\
\nonumber r_2 = \ \;x_1 - \ \;x_2 - 6   x_3 + 4   x_4 \\
\nonumber r_3 = 5   x_1 +   3 x_2 \ \ \ \ \ \ \ \ \ + \ \ x_4 \\
\end{array}
$$
\hspace{1in} and bound of variables
$$
\begin{array}{cccl}
\nonumber 1 \leq r_1 < +\infty && 0 \leq x_1 \leq 4 &{\rm(continuous)}\\
\nonumber 8 \leq r_2 < +\infty && 2 \leq x_2 \leq 5 &{\rm(integer)}   \\
\nonumber 5 \leq r_3 < +\infty && 0 \leq x_3 \leq 1 &{\rm(integer)}   \\
\nonumber                      && 3 \leq x_4 \leq 8 &{\rm(continuous)}\\
\end{array}
$$

The corresponding MPS file may look like follows:

\newpage

\begin{footnotesize}
\begin{verbatim}
NAME          SAMP1
ROWS
 N  Z
 G  R1
 G  R2
 G  R3
COLUMNS
    X1        R1                2.0    R2                 1.0
    X1        R3                5.0    Z                  3.0
    MARK0001  'MARKER'                 'INTORG'
    X2        R1               -1.0    R2                -1.0
    X2        R3                3.0    Z                  7.0
    X3        R1                1.0    R2                -6.0
    X3        Z                -1.0
    MARK0002  'MARKER'                 'INTEND'
    X4        R1               -1.0    R2                 4.0
    X4        R3                1.0    Z                  1.0
RHS
    RHS1      R1                1.0
    RHS1      R2                8.0
    RHS1      R3                5.0
BOUNDS
 UP BND1      X1                4.0
 LO BND1      X2                2.0
 UP BND1      X2                5.0
 UP BND1      X3                1.0
 LO BND1      X4                3.0
 UP BND1      X4                8.0
ENDATA
\end{verbatim}
\end{footnotesize}

%\newpage
\vspace{-3pt}

The same example may be coded without INTORG/INTEND markers using the
bound type UI for the variable $x_2$ and the bound type BV for the
variable $x_3$:

%\medskip

\begin{footnotesize}
\begin{verbatim}
NAME          SAMP2
ROWS
 N  Z
 G  R1
 G  R2
 G  R3
COLUMNS
    X1        R1                2.0    R2                 1.0
    X1        R3                5.0    Z                  3.0
    X2        R1               -1.0    R2                -1.0
    X2        R3                3.0    Z                  7.0
    X3        R1                1.0    R2                -6.0
    X3        Z                -1.0
    X4        R1               -1.0    R2                 4.0
    X4        R3                1.0    Z                  1.0
RHS
    RHS1      R1                1.0
    RHS1      R2                8.0
    RHS1      R3                5.0
BOUNDS
 UP BND1      X1                4.0
 LO BND1      X2                2.0
 UI BND1      X2                5.0
 BV BND1      X3
 LO BND1      X4                3.0
 UP BND1      X4                8.0
ENDATA
\end{verbatim}
\end{footnotesize}

%\section{Specifying predefined basis}
%\label{secbas}
%
%The MPS format can also be used to specify some predefined basis for an
%LP problem, i.e. to specify which rows and columns are basic and which
%are non-basic.
%
%The order of a basis file in the MPS format is:
%
%$\bullet$ NAME indicator card;
%
%$\bullet$ data cards (can appear in arbitrary order);
%
%$\bullet$ ENDATA indicator card.
%
%Each data card specifies either a pair "basic column---non-basic row"
%or a non-basic column. All the data cards have the following format.
%
%`\verb|XL|' in the field 1 means that a column, whose name is given in
%the field 2, is basic, and a row, whose name is given in the field 3,
%is non-basic and placed on its lower bound.
%
%`\verb|XU|' in the field 1 means that a column, whose name is given in
%the field 2, is basic, and a row, whose name is given in the field 3,
%is non-basic and placed on its upper bound.
%
%`\verb|LL|' in the field 1 means that a column, whose name is given in
%the field 3, is non-basic and placed on its lower bound.
%
%`\verb|UL|' in the field 1 means that a column, whose name is given in
%the field 3, is non-basic and placed on its upper bound.
%
%The field 2 contains a column name.
%
%If the indicator given in the field 1 is `\verb|XL|' or `\verb|XU|',
%the field 3 contains a row name. Otherwise, if the indicator is
%`\verb|LL|' or `\verb|UL|', the field 3 is not used and should be
%empty.
%
%The field 4, 5, and 6 are not used and should be empty.
%
%A basis file in the MPS format acts like a patch: it doesn't specify
%a basis completely, instead that it is just shows in what a given basis
%differs from the "standard" basis, where all rows (auxiliary variables)
%are assumed to be basic and all columns (structural variables) are
%assumed to be non-basic.
%
%As an example here is a basis file that specifies an optimal basis
%for the example LP problem given in Section \ref{secmpsex},
%Page \pageref{secmpsex}:
%
%\pagebreak
%
%\begin{verbatim}
%*000000001111111111222222222233333333334444444444555555555566
%*234567890123456789012345678901234567890123456789012345678901
%NAME          PLAN
% XL BIN2      YIELD
% XL BIN3      FE
% XL BIN4      MN
% XL ALUM      AL
% XL SILICON   SI
% LL BIN1
% LL BIN5
%ENDATA
%\end{verbatim}

%* eof *%


%* glpk09.tex *%

\chapter{CPLEX LP Format}
\label{chacplex}

\section{Prelude}

The CPLEX LP format\footnote{The CPLEX LP format was developed in
the end of 1980's by CPLEX Optimization, Inc. as an input format for
the CPLEX linear programming system. Although the CPLEX LP format is
not as widely used as the MPS format, being row-oriented it is more
convenient for coding mathematical programming models by human. This
appendix describes only the features of the CPLEX LP format which are
implemented in the GLPK package.} is intended for coding LP/MIP problem
data. It is a row-oriented format that assumes the formulation of
LP/MIP problem (1.1)---(1.3) (see Section \ref{seclp}, page
\pageref{seclp}).

CPLEX LP file is a plain text file written in CPLEX LP format. Each
text line of this file may contain up to 255 characters\footnote{GLPK
allows text lines of arbitrary length.}. Blank lines are ignored.
If a line contains the backslash character ($\backslash$), this
character and everything that follows it until the end of line are
considered as a comment and also ignored.

An LP file is coded by the user using the following elements: keywords,
symbolic names, numeric constants, delimiters, and blanks.

{\it Keywords} which may be used in the LP file are the following:

\begin{verbatim}
      minimize        minimum        min
      maximize        maximum        max
      subject to      such that      s.t.      st.      st
      bounds          bound
      general         generals       gen
      integer         integers       int
      binary          binaries       bin
      infinity        inf
      free
      end
\end{verbatim}

\noindent
All the keywords are case insensitive. Keywords given above on the same
line are equivalent. Any keyword (except \verb|infinity|, \verb|inf|,
and \verb|free|) being used in the LP file must start at the beginning
of a text line.

\newpage

{\it Symbolic names} are used to identify the objective function,
constraints (rows), and variables (columns). All symbolic names are case
sensitive and may contain up to 16 alphanumeric characters\footnote{GLPK
allows symbolic names having up to 255 characters.} (\verb|a|, \dots,
\verb|z|, \verb|A|, \dots, \verb|Z|, \verb|0|, \dots, \verb|9|) as well
as the following characters:

\begin{verbatim}
      !  "  #  $  %  &  (  )  /  ,  .  ;  ?  @  _  `  '  {  }  |  ~
\end{verbatim}

\noindent
with exception that no symbolic name can begin with a digit or
a period.

{\it Numeric constants} are used to denote constraint and objective
coefficients, right-hand sides of constraints, and bounds of variables.
They are coded in the standard form $xx$\verb|E|$syy$, where $xx$ is
a real number with optional decimal point, $s$ is a sign (\verb|+| or
\verb|-|), $yy$ is an integer decimal exponent. Numeric constants may
contain arbitrary number of characters. The exponent part is optional.
The letter `\verb|E|' can be coded as `\verb|e|'. If the sign $s$ is
omitted, plus is assumed.

{\it Delimiters} that may be used in the LP file are the following:

\begin{verbatim}
      :
      +
      -
      <   <=   =<
      >   >=   =>
      =
\end{verbatim}

\noindent
Delimiters given above on the same line are equivalent. The meaning of
the delimiters will be explained below.

{\it Blanks} are non-significant characters. They may be used freely to
improve readability of the LP file. Besides, blanks should be used to
separate elements from each other if there is no other way to do that
(for example, to separate a keyword from a following symbolic name).

The order of an LP file is the following:

%\vspace*{-8pt}

%\begin{itemize}
\Item{---}objective function definition;

\Item{---}constraints section;

\Item{---}bounds section;

\Item{---}general, integer, and binary sections (can appear in
arbitrary order);

\Item{---}end keyword.
%\end{itemize}

%\vspace*{-8pt}

These components are discussed in following sections.

\section{Objective function definition}

The objective function definition must appear first in the LP file.
It defines the objective function and specifies the optimization
direction.

The objective function definition has the following form:
$$
\left\{
\begin{array}{@{}c@{}}
{\tt minimize} \\ {\tt maximize}
\end{array}
\right\}\ f\ {\tt :}\ s\ c\ x\ \dots\ s\ c\ x
$$
where $f$ is a symbolic name of the objective function, $s$ is a sign
\verb|+| or \verb|-|, $c$ is a numeric constant that denotes an
objective coefficient, $x$ is a symbolic name of a variable.

If necessary, the objective function definition can be continued on as
many text lines as desired.

The name of the objective function is optional and may be omitted
(together with the semicolon that follows it). In this case the default
name `\verb|obj|' is assigned to the objective function.

If the very first sign $s$ is omitted, the sign plus is assumed. Other
signs cannot be omitted.

If some objective coefficient $c$ is omitted, 1 is assumed.

Symbolic names $x$ used to denote variables are recognized by context
and therefore needn't to be declared somewhere else.

Here is an example of the objective function definition:

\begin{verbatim}
   Minimize Z : - x1 + 2 x2 - 3.5 x3 + 4.997e3x(4) + x5 + x6 +
      x7 - .01x8
\end{verbatim}

\section{Constraints section}

The constraints section must follow the objective function definition.
It defines a system of equality and/or inequality constraints.

The constraint section has the following form:

\begin{center}
\begin{tabular}{l}
\verb|subject to| \\
{\it constraint}$_1$ \\
\hspace{20pt}\dots \\
{\it constraint}$_m$ \\
\end{tabular}
\end{center}

\noindent where {\it constraint}$_i, i=1,\dots,m,$ is a particular
constraint definition.

Each constraint definition can be continued on as many text lines as
desired. However, each constraint definition must begin on a new line
except the very first constraint definition which can begin on the same
line as the keyword `\verb|subject to|'.

Constraint definitions have the following form:
$$
r\ {\tt:}\ s\ c\ x\ \dots\ s\ c\ x
\ \left\{
\begin{array}{@{}c@{}}
\mbox{\tt<=} \\ \mbox{\tt>=} \\ \mbox{\tt=}
\end{array}
\right\}\ b
$$
where $r$ is a symbolic name of a constraint, $s$ is a sign \verb|+| or
\verb|-|, $c$ is a numeric constant that denotes a constraint
coefficient, $x$ is a symbolic name of a variable, $b$ is a right-hand
side.

The name $r$ of a constraint (which is the name of the corresponding
auxiliary variable) is optional and may be omitted together with the
semicolon that follows it. In the latter case the default names like
`\verb|r.nnn|' are assigned to unnamed constraints.

The linear form $s\ c\ x\ \dots\ s\ c\ x$ in the left-hand side of
a constraint definition has exactly the same meaning as in the case of
the objective function definition (see above).

After the linear form one of the following delimiters that indicates
the constraint sense must be specified:

\verb|<=| \ means `less than or equal to'

\verb|>=| \ means `greater than or equal to'

\verb|= | \ means `equal to'

The right hand side $b$ is a numeric constant with an optional sign.

Here is an example of the constraints section:

\begin{verbatim}
   Subject To
      one: y1 + 3 a1 - a2 - b >= 1.5
      y2 + 2 a3 + 2
         a4 - b >= -1.5
      two : y4 + 3 a1 + 4 a5 - b <= +1
      .20y5 + 5 a2 - b = 0
      1.7 y6 - a6 + 5 a777 - b >= 1
\end{verbatim}

Should note that it is impossible to express ranged constraints in the
CPLEX LP format. Each a ranged constraint can be coded as two
constraints with identical linear forms in the left-hand side, one of
which specifies a lower bound and other does an upper one of the
original ranged constraint. Another way is to introduce a slack
double-bounded variable; for example, the
constraint
$$10\leq x+2y+3z\leq 50$$
can be written as follows:
$$x+2y+3z+t=50,$$
where $0\leq t\leq 40$ is a slack variable.

\section{Bounds section}

The bounds section is intended to define bounds of variables. This
section is optional; if it is specified, it must follow the constraints
section. If the bound section is omitted, all variables are assumed to
be non-negative (i.e. that they have zero lower bound and no upper
bound).

The bounds section has the following form:

\begin{center}
\begin{tabular}{l}
\verb|bounds| \\
{\it definition}$_1$ \\
\hspace{20pt}\dots \\
{\it definition}$_p$ \\
\end{tabular}
\end{center}

\noindent
where {\it definition}$_k, k=1,\dots,p,$ is a particular bound
definition.

Each bound definition must begin on a new line\footnote{The GLPK
implementation allows several bound definitions to be placed on the
same line.} except the very first bound definition which can begin on
the same line as the keyword `\verb|bounds|'.

%\newpage

Syntactically constraint definitions can have one of the following six
forms:

\begin{center}
\begin{tabular}{ll}
$x$ \verb|>=| $l$ &              specifies a lower bound \\
$l$ \verb|<=| $x$ &              specifies a lower bound \\
$x$ \verb|<=| $u$ &              specifies an upper bound \\
$l$ \verb|<=| $x$ \verb|<=| $u$ &specifies both lower and upper bounds\\
$x$ \verb|=| $t$                &specifies a fixed value \\
$x$ \verb|free|                 &specifies free variable
\end{tabular}
\end{center}

\noindent
where $x$ is a symbolic name of a variable, $l$ is a numeric constant
with an optional sign that defines a lower bound of the variable or
\verb|-inf| that means that the variable has no lower bound, $u$ is
a~numeric constant with an optional sign that defines an upper bound of
the variable or \verb|+inf| that means that the variable has no upper
bound, $t$ is a numeric constant with an optional sign that defines a
fixed value of the variable.

By default all variables are non-negative, i.e. have zero lower bound
and no upper bound. Therefore definitions of these default bounds can
be omitted in the bounds section.

Here is an example of the bounds section:

\begin{verbatim}
   Bounds
      -inf <= a1 <= 100
      -100 <= a2
      b <= 100
      x2 = +123.456
      x3 free
\end{verbatim}

\section{General, integer, and binary sections}

The general, integer, and binary sections are intended to define
some variables as integer or binary. All these sections are optional
and needed only in case of MIP problems. If they are specified, they
must follow the bounds section or, if the latter is omitted, the
constraints section.

All the general, integer, and binary sections have the same form as
follows:

\begin{center}
\begin{tabular}{l}
$
\left\{
\begin{array}{@{}l@{}}
\verb|general| \\
\verb|integer| \\
\verb|binary | \\
\end{array}
\right\}
$ \\
\hspace{10pt}$x_1$ \\
\hspace{10pt}\dots \\
\hspace{10pt}$x_q$ \\
\end{tabular}
\end{center}

\noindent
where $x_k$ is a symbolic name of variable, $k=1,\dots,q$.

Each symbolic name must begin on a new line\footnote{The GLPK
implementation allows several symbolic names to be placed on the same
line.} except the very first symbolic name which can begin on the same
line as the keyword `\verb|general|', `\verb|integer|', or
`\verb|binary|'.

%\newpage

If a variable appears in the general or the integer section, it is
assumed to be general integer variable. If a variable appears in the
binary section, it is assumed to be binary variable, i.e. an integer
variable whose lower bound is zero and upper bound is one. (Note that
if bounds of a variable are specified in the bounds section and then
the variable appears in the binary section, its previously specified
bounds are ignored.)

Here is an example of the integer section:

\begin{verbatim}
   Integer
      z12
      z22
      z35
\end{verbatim}

\newpage

\section{End keyword}

The keyword `\verb|end|' is intended to end the LP file. It must begin
on a separate line and no other elements (except comments and blank
lines) must follow it. Although this keyword is optional, it is strongly
recommended to include it in the LP file.

\section{Example of CPLEX LP file}

Here is a complete example of CPLEX LP file that corresponds to the
example given in Section \ref{secmpsex}, page \pageref{secmpsex}.

\medskip

\begin{footnotesize}
\begin{verbatim}
\* plan.lp *\

Minimize
   value: .03 bin1 + .08 bin2 + .17 bin3 + .12 bin4 + .15 bin5 +
          .21 alum + .38 silicon

Subject To
   yield:     bin1 +     bin2 +     bin3 +     bin4 +     bin5 +
              alum +     silicon                                 =  2000

   fe:    .15 bin1 + .04 bin2 + .02 bin3 + .04 bin4 + .02 bin5 +
          .01 alum + .03 silicon                                 <=   60

   cu:    .03 bin1 + .05 bin2 + .08 bin3 + .02 bin4 + .06 bin5 +
          .01 alum                                               <=  100

   mn:    .02 bin1 + .04 bin2 + .01 bin3 + .02 bin4 + .02 bin5   <=   40

   mg:    .02 bin1 + .03 bin2                       + .01 bin5   <=   30

   al:    .70 bin1 + .75 bin2 + .80 bin3 + .75 bin4 + .80 bin5 +
          .97 alum                                               >= 1500

   si1:   .02 bin1 + .06 bin2 + .08 bin3 + .12 bin4 + .02 bin5 +
          .01 alum + .97 silicon                                 >=  250

   si2:   .02 bin1 + .06 bin2 + .08 bin3 + .12 bin4 + .02 bin5 +
          .01 alum + .97 silicon                                 <=  300

Bounds
          bin1 <=  200
          bin2 <= 2500
   400 <= bin3 <=  800
   100 <= bin4 <=  700
          bin5 <= 1500

End

\* eof *\
\end{verbatim}
\end{footnotesize}

%* eof *%


%* glpk10.tex *%

\chapter{Stand-alone LP/MIP Solver}
\label{chaglpsol}

The GLPK package includes the program \verb|glpsol|, which is a
stand-alone LP/MIP solver. This program can be invoked from the command
line to read LP/MIP problem data in any format supported by GLPK, solve
the problem, and write its solution to an output text file.

\para{Usage}

\verb|glpsol| [{\it options\dots}] [{\it filename}]

\para{General options}

\begin{verbatim}
   --mps             read LP/MIP problem in fixed MPS format
   --freemps         read LP/MIP problem in free MPS format (default)
   --lp              read LP/MIP problem in CPLEX LP format
   --glp             read LP/MIP problem in GLPK format
   --math            read LP/MIP model written in GNU MathProg modeling
                     language
   -m filename, --model filename
                     read model section and optional data section from
                     filename (same as --math)
   -d filename, --data filename
                     read data section from filename (for --math only);
                     if model file also has data section, it is ignored
   -y filename, --display filename
                     send display output to filename (for --math only);
                     by default the output is sent to terminal
   --seed value      initialize pseudo-random number generator used in
                     MathProg model with specified seed (any integer);
                     if seed value is ?, some random seed will be used
   --mincost         read min-cost flow problem in DIMACS format
   --maxflow         read maximum flow problem in DIMACS format
   --cnf             read CNF-SAT problem in DIMACS format
   --simplex         use simplex method (default)
   --interior        use interior point method (LP only)
   -r filename, --read filename
                     read solution from filename rather to find it with
                     the solver
   --min             minimization
   --max             maximization
   --scale           scale problem (default)
   --noscale         do not scale problem
   -o filename, --output filename
                     write solution to filename in printable format
   -w filename, --write filename
                     write solution to filename in plain text format
   --ranges filename
                     write sensitivity analysis report to filename in
                     printable format (simplex only)
   --tmlim nnn       limit solution time to nnn seconds
   --memlim nnn      limit available memory to nnn megabytes
   --check           do not solve problem, check input data only
   --name probname   change problem name to probname
   --wmps filename   write problem to filename in fixed MPS format
   --wfreemps filename
                     write problem to filename in free MPS format
   --wlp filename    write problem to filename in CPLEX LP format
   --wglp filename   write problem to filename in GLPK format
   --wcnf filename   write problem to filename in DIMACS CNF-SAT format
   --log filename    write copy of terminal output to filename
   -h, --help        display this help information and exit
   -v, --version     display program version and exit
\end{verbatim}

\para{LP basis factorization options}

\begin{verbatim}
   --luf             plain LU factorization (default)
   --btf             block triangular LU factorization
   --ft              Forrest-Tomlin update (requires --luf; default)
   --cbg             Schur complement + Bartels-Golub update
   --cgr             Schur complement + Givens rotation update
\end{verbatim}

\para{Options specific to the simplex solver}

\begin{verbatim}
   --primal          use primal simplex (default)
   --dual            use dual simplex
   --std             use standard initial basis of all slacks
   --adv             use advanced initial basis (default)
   --bib             use Bixby's initial basis
   --ini filename    use as initial basis previously saved with -w
                     (disables LP presolver)
   --steep           use steepest edge technique (default)
   --nosteep         use standard "textbook" pricing
   --relax           use Harris' two-pass ratio test (default)
   --norelax         use standard "textbook" ratio test
   --presol          use presolver (default; assumes --scale and --adv)
   --nopresol        do not use presolver
   --exact           use simplex method based on exact arithmetic
   --xcheck          check final basis using exact arithmetic
\end{verbatim}

\para{Options specific to the interior-point solver}

\begin{verbatim}
   --nord            use natural (original) ordering
   --qmd             use quotient minimum degree ordering
   --amd             use approximate minimum degree ordering (default)
   --symamd          use approximate minimum degree ordering
\end{verbatim}

\para{Options specific to the MIP solver}

\begin{verbatim}
   --nomip           consider all integer variables as continuous
                     (allows solving MIP as pure LP)
   --first           branch on first integer variable
   --last            branch on last integer variable
   --mostf           branch on most fractional variable
   --drtom           branch using heuristic by Driebeck and Tomlin
                     (default)
   --pcost           branch using hybrid pseudocost heuristic (may be
                     useful for hard instances)
   --dfs             backtrack using depth first search
   --bfs             backtrack using breadth first search
   --bestp           backtrack using the best projection heuristic
   --bestb           backtrack using node with best local bound
                     (default)
   --intopt          use MIP presolver (default)
   --nointopt        do not use MIP presolver
   --binarize        replace general integer variables by binary ones
                     (assumes --intopt)
   --fpump           apply feasibility pump heuristic
   --proxy [nnn]     apply proximity search heuristic (nnn is time limit
                     in seconds; default is 60)
   --gomory          generate Gomory's mixed integer cuts
   --mir             generate MIR (mixed integer rounding) cuts
   --cover           generate mixed cover cuts
   --clique          generate clique cuts
   --cuts            generate all cuts above
   --mipgap tol      set relative mip gap tolerance to tol
   --minisat         translate integer feasibility problem to CNF-SAT
                     and solve it with MiniSat solver
   --objbnd bound    add inequality obj <= bound (minimization) or
                     obj >= bound (maximization) to integer feasibility
                     problem (assumes --minisat)
\end{verbatim}

For description of the MPS format see Appendix \ref{champs}, page
\pageref{champs}.

For description of the CPLEX LP format see Appendix \ref{chacplex},
page \pageref{chacplex}.

For description of the modeling language see the document ``Modeling
Language GNU MathProg: Language Reference'' included in the GLPK
distribution.

For description of the DIMACS min-cost flow problem format and DIMACS
maximum flow problem format see the document ``GLPK: Graph and Network
Routines'' included in the GLPK distribution.

%* eof *%


%* glpk11.tex *%

\chapter{External Software Used In GLPK}

In the GLPK package there are used some external software listed in
this Appendix. Note that these software are {\it not} part of GLPK, but
are used with GLPK and included in the distribution.

%%%%%%%%%%%%%%%%%%%%%%%%%%%%%%%%%%%%%%%%%%%%%%%%%%%%%%%%%%%%%%%%%%%%%%%%

\section{AMD}

\noindent
AMD Version 2.2, Copyright {\copyright} 2007 by Timothy A. Davis,
Patrick R. Amestoy, and Iain S. Duff.  All Rights Reserved.

\para{Description}

AMD is a set of routines for pre-ordering sparse matrices prior to
Cholesky or LU factorization, using the approximate minimum degree
ordering algorithm.

\para{License}

This library is free software; you can redistribute it and/or
modify it under the terms of the GNU Lesser General Public License
as published by the Free Software Foundation; either version 2.1 of
the License, or (at your option) any later version.

This library is distributed in the hope that it will be useful,
but WITHOUT ANY WARRANTY; without even the implied warranty of
MERCHANTABILITY or FITNESS FOR A PARTICULAR PURPOSE.  See the GNU
Lesser General Public License for more details.

You should have received a copy of the GNU Lesser General Public
License along with this library; if not, write to the Free Software
Foundation, Inc., 51 Franklin St, Fifth Floor, Boston, MA 02110-1301
USA.

Permission is hereby granted to use or copy this program under the
terms of the GNU LGPL, provided that the Copyright, this License,
and the Availability of the original version is retained on all
copies.  User documentation of any code that uses this code or any
modified version of this code must cite the Copyright, this License,
the Availability note, and ``Used by permission.''  Permission to
modify the code and to distribute modified code is granted, provided
the Copyright, this License, and the Availability note are retained,
and a notice that the code was modified is included.

AMD is available under alternate licences; contact T. Davis for
details.

\para{Availability}

\noindent
\url{http://www.cise.ufl.edu/research/sparse/amd}

%%%%%%%%%%%%%%%%%%%%%%%%%%%%%%%%%%%%%%%%%%%%%%%%%%%%%%%%%%%%%%%%%%%%%%%%

\section{COLAMD/SYMAMD}

\noindent
COLAMD/SYMAMD Version 2.7, Copyright {\copyright} 1998-2007, Timothy A.
Davis, All Rights Reserved.

\para{Description}

colamd: an approximate minimum degree column ordering algorithm, for
LU factorization of symmetric or unsymmetric matrices, QR factorization,
least squares, interior point methods for linear programming problems,
and other related problems.

symamd: an approximate minimum degree ordering algorithm for Cholesky
factorization of symmetric matrices.

\para{Authors}

The authors of the code itself are Stefan I. Larimore and Timothy A.
Davis (davis at cise.ufl.edu), University of Florida.  The algorithm
was developed in collaboration with John Gilbert, Xerox PARC, and
Esmond Ng, Oak Ridge National Laboratory.

\para{License}

This library is free software; you can redistribute it and/or
modify it under the terms of the GNU Lesser General Public License
as published by the Free Software Foundation; either version 2.1 of
the License, or (at your option) any later version.

This library is distributed in the hope that it will be useful,
but WITHOUT ANY WARRANTY; without even the implied warranty of
MERCHANTABILITY or FITNESS FOR A PARTICULAR PURPOSE.  See the GNU
Lesser General Public License for more details.

You should have received a copy of the GNU Lesser General Public
License along with this library; if not, write to the Free Software
Foundation, Inc., 51 Franklin St, Fifth Floor, Boston, MA 02110-1301
USA.

Permission is hereby granted to use or copy this program under the
terms of the GNU LGPL, provided that the Copyright, this License,
and the Availability of the original version is retained on all
copies.  User documentation of any code that uses this code or any
modified version of this code must cite the Copyright, this License,
the Availability note, and ``Used by permission.''  Permission to
modify the code and to distribute modified code is granted, provided
the Copyright, this License, and the Availability note are retained,
and a notice that the code was modified is included.

COLAMD is also available under alternate licenses, contact T. Davis for
details.

\para{Availability}

\noindent
\url{http://www.cise.ufl.edu/research/sparse/colamd}

%%%%%%%%%%%%%%%%%%%%%%%%%%%%%%%%%%%%%%%%%%%%%%%%%%%%%%%%%%%%%%%%%%%%%%%%

%\newpage

\section{MiniSat}

\noindent
MiniSat-C v1.14.1, Copyright {\copyright} 2005, Niklas Sorensson.

\para{Description}

MiniSat is a minimalistic implementation of a Chaff-like SAT solver
based on the two-literal watch scheme for fast BCP and clause learning
by conflict analysis.

\newpage

\para{License}

Permission is hereby granted, free of charge, to any person obtaining a
copy of this software and associated documentation files (the
"Software"), to deal in the Software without restriction, including
without limitation the rights to use, copy, modify, merge, publish,
distribute, sublicense, and/or sell copies of the Software, and to
permit persons to whom the Software is furnished to do so, subject to
the following conditions:

The above copyright notice and this permission notice shall be included
in all copies or substantial portions of the Software.

THE SOFTWARE IS PROVIDED "AS IS", WITHOUT WARRANTY OF ANY KIND, EXPRESS
OR IMPLIED, INCLUDING BUT NOT LIMITED TO THE WARRANTIES OF
MERCHANTABILITY, FITNESS FOR A PARTICULAR PURPOSE AND
NONINFRINGEMENT. IN NO EVENT SHALL THE AUTHORS OR COPYRIGHT HOLDERS BE
LIABLE FOR ANY CLAIM, DAMAGES OR OTHER LIABILITY, WHETHER IN AN ACTION
OF CONTRACT, TORT OR OTHERWISE, ARISING FROM, OUT OF OR IN CONNECTION
WITH THE SOFTWARE OR THE USE OR OTHER DEALINGS IN THE SOFTWARE.

\para{Availability}

\noindent
\url{http://www.cs.chalmers.se/Cs/Research/FormalMethods/MiniSat}

%%%%%%%%%%%%%%%%%%%%%%%%%%%%%%%%%%%%%%%%%%%%%%%%%%%%%%%%%%%%%%%%%%%%%%%%

\section{zlib}

\noindent
zlib version 1.2.5, Copyright {\copyright} 1995--2010 Jean-loup Gailly
and Mark Adler.

\para{Description}

zlib is a general purpose data compression library. All the code is
thread safe. The data format used by the zlib library is described by
RFCs (Request for Comments) 1950 to 1952 in the files
\verb|rfc1950.txt| (zlib format), \verb|rfc1951.txt| (deflate format)
and \verb|rfc1952.txt| (gzip format).

\para{License}

This software is provided 'as-is', without any express or implied
warranty. In no event will the authors be held liable for any damages
arising from the use of this software.

Permission is granted to anyone to use this software for any purpose,
including commercial applications, and to alter it and redistribute it
freely, subject to the following restrictions:

1. The origin of this software must not be misrepresented; you must not
   claim that you wrote the original software. If you use this software
   in a product, an acknowledgment in the product documentation would
   be appreciated but is not required.

2. Altered source versions must be plainly marked as such, and must not
   be misrepresented as being the original software.

3. This notice may not be removed or altered from any source
   distribution.

\hfill Jean-loup Gailly

\hfill Mark Adler

\para{Availability}

\noindent
\url{http://www.zlib.net/}

%* eof *%


%* glpk12.tex *%

\begin{footnotesize}

\chapter*{\sf\bfseries GNU General Public License}
\addcontentsline{toc}{chapter}{GNU General Public License}

\begin{center}
{\bf Version 3, 29 June 2007}
\end{center}

\begin{quotation}
\noindent
Copyright {\copyright} 2007 Free Software Foundation, Inc.
\verb|<http://fsf.org/>|
\end{quotation}

\begin{quotation}
\noindent
Everyone is permitted to copy and distribute verbatim copies
of this license document, but changing it is not allowed.
\end{quotation}

\medskip\para{\normalsize Preamble}

  The GNU General Public License is a free, copyleft license for
software and other kinds of works.

  The licenses for most software and other practical works are designed
to take away your freedom to share and change the works.  By contrast,
the GNU General Public License is intended to guarantee your freedom to
share and change all versions of a program--to make sure it remains free
software for all its users.  We, the Free Software Foundation, use the
GNU General Public License for most of our software; it applies also to
any other work released this way by its authors.  You can apply it to
your programs, too.

  When we speak of free software, we are referring to freedom, not
price.  Our General Public Licenses are designed to make sure that you
have the freedom to distribute copies of free software (and charge for
them if you wish), that you receive source code or can get it if you
want it, that you can change the software or use pieces of it in new
free programs, and that you know you can do these things.

  To protect your rights, we need to prevent others from denying you
these rights or asking you to surrender the rights.  Therefore, you have
certain responsibilities if you distribute copies of the software, or if
you modify it: responsibilities to respect the freedom of others.

  For example, if you distribute copies of such a program, whether
gratis or for a fee, you must pass on to the recipients the same
freedoms that you received.  You must make sure that they, too, receive
or can get the source code.  And you must show them these terms so they
know their rights.

  Developers that use the GNU GPL protect your rights with two steps:
(1) assert copyright on the software, and (2) offer you this License
giving you legal permission to copy, distribute and/or modify it.

  For the developers' and authors' protection, the GPL clearly explains
that there is no warranty for this free software.  For both users' and
authors' sake, the GPL requires that modified versions be marked as
changed, so that their problems will not be attributed erroneously to
authors of previous versions.

  Some devices are designed to deny users access to install or run
modified versions of the software inside them, although the manufacturer
can do so.  This is fundamentally incompatible with the aim of
protecting users' freedom to change the software.  The systematic
pattern of such abuse occurs in the area of products for individuals to
use, which is precisely where it is most unacceptable.  Therefore, we
have designed this version of the GPL to prohibit the practice for those
products.  If such problems arise substantially in other domains, we
stand ready to extend this provision to those domains in future versions
of the GPL, as needed to protect the freedom of users.

  Finally, every program is threatened constantly by software patents.
States should not allow patents to restrict development and use of
software on general-purpose computers, but in those that do, we wish to
avoid the special danger that patents applied to a free program could
make it effectively proprietary.  To prevent this, the GPL assures that
patents cannot be used to render the program non-free.

  The precise terms and conditions for copying, distribution and
modification follow.

\newpage

\medskip\para{\normalsize TERMS AND CONDITIONS}

\medskip\para{\normalsize 0. Definitions.}

  ``This License'' refers to version 3 of the GNU General Public
License.

  ``Copyright'' also means copyright-like laws that apply to other kinds
of works, such as semiconductor masks.

  ``The Program'' refers to any copyrightable work licensed under this
License.  Each licensee is addressed as ``you''.  ``Licensees'' and
``recipients'' may be individuals or organizations.

  To ``modify'' a work means to copy from or adapt all or part of the
work in a fashion requiring copyright permission, other than the making
of an exact copy.  The resulting work is called a ``modified version''
of the earlier work or a work ``based on'' the earlier work.

  A ``covered work'' means either the unmodified Program or a work based
on the Program.

  To ``propagate'' a work means to do anything with it that, without
permission, would make you directly or secondarily liable for
infringement under applicable copyright law, except executing it on a
computer or modifying a private copy.  Propagation includes copying,
distribution (with or without modification), making available to the
public, and in some countries other activities as well.

  To ``convey'' a work means any kind of propagation that enables other
parties to make or receive copies.  Mere interaction with a user through
a computer network, with no transfer of a copy, is not conveying.

  An interactive user interface displays ``Appropriate Legal Notices''
to the extent that it includes a convenient and prominently visible
feature that (1) displays an appropriate copyright notice, and (2)
tells the user that there is no warranty for the work (except to the
extent that warranties are provided), that licensees may convey the
work under this License, and how to view a copy of this License.  If
the interface presents a list of user commands or options, such as a
menu, a prominent item in the list meets this criterion.

\medskip\para{\normalsize 1. Source Code.}

  The ``source code'' for a work means the preferred form of the work
for making modifications to it.  ``Object code'' means any non-source
form of a work.

  A ``Standard Interface'' means an interface that either is an official
standard defined by a recognized standards body, or, in the case of
interfaces specified for a particular programming language, one that
is widely used among developers working in that language.

  The ``System Libraries'' of an executable work include anything, other
than the work as a whole, that (a) is included in the normal form of
packaging a Major Component, but which is not part of that Major
Component, and (b) serves only to enable use of the work with that
Major Component, or to implement a Standard Interface for which an
implementation is available to the public in source code form.  A
``Major Component'', in this context, means a major essential component
(kernel, window system, and so on) of the specific operating system
(if any) on which the executable work runs, or a compiler used to
produce the work, or an object code interpreter used to run it.

  The ``Corresponding Source'' for a work in object code form means all
the source code needed to generate, install, and (for an executable
work) run the object code and to modify the work, including scripts to
control those activities.  However, it does not include the work's
System Libraries, or general-purpose tools or generally available free
programs which are used unmodified in performing those activities but
which are not part of the work.  For example, Corresponding Source
includes interface definition files associated with source files for
the work, and the source code for shared libraries and dynamically
linked subprograms that the work is specifically designed to require,
such as by intimate data communication or control flow between those
subprograms and other parts of the work.

  The Corresponding Source need not include anything that users
can regenerate automatically from other parts of the Corresponding
Source.

  The Corresponding Source for a work in source code form is that
same work.

\medskip\para{\normalsize 2. Basic Permissions.}

  All rights granted under this License are granted for the term of
copyright on the Program, and are irrevocable provided the stated
conditions are met.  This License explicitly affirms your unlimited
permission to run the unmodified Program.  The output from running a
covered work is covered by this License only if the output, given its
content, constitutes a covered work.  This License acknowledges your
rights of fair use or other equivalent, as provided by copyright law.

  You may make, run and propagate covered works that you do not
convey, without conditions so long as your license otherwise remains
in force.  You may convey covered works to others for the sole purpose
of having them make modifications exclusively for you, or provide you
with facilities for running those works, provided that you comply with
the terms of this License in conveying all material for which you do
not control copyright.  Those thus making or running the covered works
for you must do so exclusively on your behalf, under your direction
and control, on terms that prohibit them from making any copies of
your copyrighted material outside their relationship with you.

  Conveying under any other circumstances is permitted solely under
the conditions stated below.  Sublicensing is not allowed; section 10
makes it unnecessary.

\medskip\para{\normalsize 3. Protecting Users' Legal Rights From
Anti-Circumvention Law.}

  No covered work shall be deemed part of an effective technological
measure under any applicable law fulfilling obligations under article
11 of the WIPO copyright treaty adopted on 20 December 1996, or
similar laws prohibiting or restricting circumvention of such
measures.

  When you convey a covered work, you waive any legal power to forbid
circumvention of technological measures to the extent such circumvention
is effected by exercising rights under this License with respect to
the covered work, and you disclaim any intention to limit operation or
modification of the work as a means of enforcing, against the work's
users, your or third parties' legal rights to forbid circumvention of
technological measures.

\medskip\para{\normalsize 4. Conveying Verbatim Copies.}

  You may convey verbatim copies of the Program's source code as you
receive it, in any medium, provided that you conspicuously and
appropriately publish on each copy an appropriate copyright notice;
keep intact all notices stating that this License and any
non-permissive terms added in accord with section 7 apply to the code;
keep intact all notices of the absence of any warranty; and give all
recipients a copy of this License along with the Program.

  You may charge any price or no price for each copy that you convey,
and you may offer support or warranty protection for a fee.

\medskip\para{\normalsize 5. Conveying Modified Source Versions.}

  You may convey a work based on the Program, or the modifications to
produce it from the Program, in the form of source code under the
terms of section 4, provided that you also meet all of these conditions:

    a) The work must carry prominent notices stating that you modified
    it, and giving a relevant date.

    b) The work must carry prominent notices stating that it is
    released under this License and any conditions added under section
    7.  This requirement modifies the requirement in section 4 to
    ``keep intact all notices''.

    c) You must license the entire work, as a whole, under this
    License to anyone who comes into possession of a copy.  This
    License will therefore apply, along with any applicable section 7
    additional terms, to the whole of the work, and all its parts,
    regardless of how they are packaged.  This License gives no
    permission to license the work in any other way, but it does not
    invalidate such permission if you have separately received it.

    d) If the work has interactive user interfaces, each must display
    Appropriate Legal Notices; however, if the Program has interactive
    interfaces that do not display Appropriate Legal Notices, your
    work need not make them do so.

  A compilation of a covered work with other separate and independent
works, which are not by their nature extensions of the covered work,
and which are not combined with it such as to form a larger program,
in or on a volume of a storage or distribution medium, is called an
``aggregate'' if the compilation and its resulting copyright are not
used to limit the access or legal rights of the compilation's users
beyond what the individual works permit.  Inclusion of a covered work
in an aggregate does not cause this License to apply to the other
parts of the aggregate.

\medskip\para{\normalsize 6. Conveying Non-Source Forms.}

  You may convey a covered work in object code form under the terms
of sections 4 and 5, provided that you also convey the
machine-readable Corresponding Source under the terms of this License,
in one of these ways:

    a) Convey the object code in, or embodied in, a physical product
    (including a physical distribution medium), accompanied by the
    Corresponding Source fixed on a durable physical medium
    customarily used for software interchange.

    b) Convey the object code in, or embodied in, a physical product
    (including a physical distribution medium), accompanied by a
    written offer, valid for at least three years and valid for as
    long as you offer spare parts or customer support for that product
    model, to give anyone who possesses the object code either (1) a
    copy of the Corresponding Source for all the software in the
    product that is covered by this License, on a durable physical
    medium customarily used for software interchange, for a price no
    more than your reasonable cost of physically performing this
    conveying of source, or (2) access to copy the
    Corresponding Source from a network server at no charge.

    c) Convey individual copies of the object code with a copy of the
    written offer to provide the Corresponding Source.  This
    alternative is allowed only occasionally and noncommercially, and
    only if you received the object code with such an offer, in accord
    with subsection 6b.

    d) Convey the object code by offering access from a designated
    place (gratis or for a charge), and offer equivalent access to the
    Corresponding Source in the same way through the same place at no
    further charge.  You need not require recipients to copy the
    Corresponding Source along with the object code.  If the place to
    copy the object code is a network server, the Corresponding Source
    may be on a different server (operated by you or a third party)
    that supports equivalent copying facilities, provided you maintain
    clear directions next to the object code saying where to find the
    Corresponding Source.  Regardless of what server hosts the
    Corresponding Source, you remain obligated to ensure that it is
    available for as long as needed to satisfy these requirements.

    e) Convey the object code using peer-to-peer transmission, provided
    you inform other peers where the object code and Corresponding
    Source of the work are being offered to the general public at no
    charge under subsection 6d.

  A separable portion of the object code, whose source code is excluded
from the Corresponding Source as a System Library, need not be
included in conveying the object code work.

  A ``User Product'' is either (1) a ``consumer product'', which means
any tangible personal property which is normally used for personal,
family, or household purposes, or (2) anything designed or sold for
incorporation into a dwelling.  In determining whether a product is a
consumer product, doubtful cases shall be resolved in favor of coverage.
For a particular product received by a particular user, ``normally
used'' refers to a typical or common use of that class of product,
regardless of the status of the particular user or of the way in which
the particular user actually uses, or expects or is expected to use, the
product.  A product is a consumer product regardless of whether the
product has substantial commercial, industrial or non-consumer uses,
unless such uses represent the only significant mode of use of the
product.

  ``Installation Information'' for a User Product means any methods,
procedures, authorization keys, or other information required to install
and execute modified versions of a covered work in that User Product
from a modified version of its Corresponding Source.  The information
must suffice to ensure that the continued functioning of the modified
object code is in no case prevented or interfered with solely because
modification has been made.

  If you convey an object code work under this section in, or with, or
specifically for use in, a User Product, and the conveying occurs as
part of a transaction in which the right of possession and use of the
User Product is transferred to the recipient in perpetuity or for a
fixed term (regardless of how the transaction is characterized), the
Corresponding Source conveyed under this section must be accompanied
by the Installation Information.  But this requirement does not apply
if neither you nor any third party retains the ability to install
modified object code on the User Product (for example, the work has
been installed in ROM).

  The requirement to provide Installation Information does not include a
requirement to continue to provide support service, warranty, or updates
for a work that has been modified or installed by the recipient, or for
the User Product in which it has been modified or installed.  Access to
a network may be denied when the modification itself materially and
adversely affects the operation of the network or violates the rules and
protocols for communication across the network.

  Corresponding Source conveyed, and Installation Information provided,
in accord with this section must be in a format that is publicly
documented (and with an implementation available to the public in
source code form), and must require no special password or key for
unpacking, reading or copying.

\medskip\para{\normalsize 7. Additional Terms.}

  ``Additional permissions'' are terms that supplement the terms of
this License by making exceptions from one or more of its conditions.
Additional permissions that are applicable to the entire Program shall
be treated as though they were included in this License, to the extent
that they are valid under applicable law.  If additional permissions
apply only to part of the Program, that part may be used separately
under those permissions, but the entire Program remains governed by
this License without regard to the additional permissions.

  When you convey a copy of a covered work, you may at your option
remove any additional permissions from that copy, or from any part of
it.  (Additional permissions may be written to require their own
removal in certain cases when you modify the work.)  You may place
additional permissions on material, added by you to a covered work,
for which you have or can give appropriate copyright permission.

  Notwithstanding any other provision of this License, for material you
add to a covered work, you may (if authorized by the copyright holders
of that material) supplement the terms of this License with terms:

    a) Disclaiming warranty or limiting liability differently from the
    terms of sections 15 and 16 of this License; or

    b) Requiring preservation of specified reasonable legal notices or
    author attributions in that material or in the Appropriate Legal
    Notices displayed by works containing it; or

    c) Prohibiting misrepresentation of the origin of that material, or
    requiring that modified versions of such material be marked in
    reasonable ways as different from the original version; or

    d) Limiting the use for publicity purposes of names of licensors or
    authors of the material; or

    e) Declining to grant rights under trademark law for use of some
    trade names, trademarks, or service marks; or

    f) Requiring indemnification of licensors and authors of that
    material by anyone who conveys the material (or modified versions of
    it) with contractual assumptions of liability to the recipient, for
    any liability that these contractual assumptions directly impose on
    those licensors and authors.

  All other non-permissive additional terms are considered ``further
restrictions'' within the meaning of section 10.  If the Program as you
received it, or any part of it, contains a notice stating that it is
governed by this License along with a term that is a further
restriction, you may remove that term.  If a license document contains
a further restriction but permits relicensing or conveying under this
License, you may add to a covered work material governed by the terms
of that license document, provided that the further restriction does
not survive such relicensing or conveying.

  If you add terms to a covered work in accord with this section, you
must place, in the relevant source files, a statement of the
additional terms that apply to those files, or a notice indicating
where to find the applicable terms.

  Additional terms, permissive or non-permissive, may be stated in the
form of a separately written license, or stated as exceptions;
the above requirements apply either way.

\medskip\para{\normalsize 8. Termination.}

  You may not propagate or modify a covered work except as expressly
provided under this License.  Any attempt otherwise to propagate or
modify it is void, and will automatically terminate your rights under
this License (including any patent licenses granted under the third
paragraph of section 11).

  However, if you cease all violation of this License, then your
license from a particular copyright holder is reinstated (a)
provisionally, unless and until the copyright holder explicitly and
finally terminates your license, and (b) permanently, if the copyright
holder fails to notify you of the violation by some reasonable means
prior to 60 days after the cessation.

  Moreover, your license from a particular copyright holder is
reinstated permanently if the copyright holder notifies you of the
violation by some reasonable means, this is the first time you have
received notice of violation of this License (for any work) from that
copyright holder, and you cure the violation prior to 30 days after
your receipt of the notice.

  Termination of your rights under this section does not terminate the
licenses of parties who have received copies or rights from you under
this License.  If your rights have been terminated and not permanently
reinstated, you do not qualify to receive new licenses for the same
material under section 10.

\medskip\para{\normalsize 9. Acceptance Not Required for Having Copies.}

  You are not required to accept this License in order to receive or
run a copy of the Program.  Ancillary propagation of a covered work
occurring solely as a consequence of using peer-to-peer transmission
to receive a copy likewise does not require acceptance.  However,
nothing other than this License grants you permission to propagate or
modify any covered work.  These actions infringe copyright if you do
not accept this License.  Therefore, by modifying or propagating a
covered work, you indicate your acceptance of this License to do so.

\medskip\para{\normalsize 10. Automatic Licensing of Downstream Recipients.}

  Each time you convey a covered work, the recipient automatically
receives a license from the original licensors, to run, modify and
propagate that work, subject to this License.  You are not responsible
for enforcing compliance by third parties with this License.

  An ``entity transaction'' is a transaction transferring control of an
organization, or substantially all assets of one, or subdividing an
organization, or merging organizations.  If propagation of a covered
work results from an entity transaction, each party to that
transaction who receives a copy of the work also receives whatever
licenses to the work the party's predecessor in interest had or could
give under the previous paragraph, plus a right to possession of the
Corresponding Source of the work from the predecessor in interest, if
the predecessor has it or can get it with reasonable efforts.

  You may not impose any further restrictions on the exercise of the
rights granted or affirmed under this License.  For example, you may
not impose a license fee, royalty, or other charge for exercise of
rights granted under this License, and you may not initiate litigation
(including a cross-claim or counterclaim in a lawsuit) alleging that
any patent claim is infringed by making, using, selling, offering for
sale, or importing the Program or any portion of it.

\medskip\para{\normalsize 11. Patents.}

  A ``contributor'' is a copyright holder who authorizes use under this
License of the Program or a work on which the Program is based.  The
work thus licensed is called the contributor's ``contributor version''.

  A contributor's ``essential patent claims'' are all patent claims
owned or controlled by the contributor, whether already acquired or
hereafter acquired, that would be infringed by some manner, permitted
by this License, of making, using, or selling its contributor version,
but do not include claims that would be infringed only as a
consequence of further modification of the contributor version.  For
purposes of this definition, ``control'' includes the right to grant
patent sublicenses in a manner consistent with the requirements of
this License.

  Each contributor grants you a non-exclusive, worldwide, royalty-free
patent license under the contributor's essential patent claims, to
make, use, sell, offer for sale, import and otherwise run, modify and
propagate the contents of its contributor version.

  In the following three paragraphs, a ``patent license'' is any express
agreement or commitment, however denominated, not to enforce a patent
(such as an express permission to practice a patent or covenant not to
sue for patent infringement).  To ``grant'' such a patent license to a
party means to make such an agreement or commitment not to enforce a
patent against the party.

  If you convey a covered work, knowingly relying on a patent license,
and the Corresponding Source of the work is not available for anyone
to copy, free of charge and under the terms of this License, through a
publicly available network server or other readily accessible means,
then you must either (1) cause the Corresponding Source to be so
available, or (2) arrange to deprive yourself of the benefit of the
patent license for this particular work, or (3) arrange, in a manner
consistent with the requirements of this License, to extend the patent
license to downstream recipients.  ``Knowingly relying'' means you have
actual knowledge that, but for the patent license, your conveying the
covered work in a country, or your recipient's use of the covered work
in a country, would infringe one or more identifiable patents in that
country that you have reason to believe are valid.

  If, pursuant to or in connection with a single transaction or
arrangement, you convey, or propagate by procuring conveyance of, a
covered work, and grant a patent license to some of the parties
receiving the covered work authorizing them to use, propagate, modify
or convey a specific copy of the covered work, then the patent license
you grant is automatically extended to all recipients of the covered
work and works based on it.

  A patent license is ``discriminatory'' if it does not include within
the scope of its coverage, prohibits the exercise of, or is
conditioned on the non-exercise of one or more of the rights that are
specifically granted under this License.  You may not convey a covered
work if you are a party to an arrangement with a third party that is
in the business of distributing software, under which you make payment
to the third party based on the extent of your activity of conveying
the work, and under which the third party grants, to any of the
parties who would receive the covered work from you, a discriminatory
patent license (a) in connection with copies of the covered work
conveyed by you (or copies made from those copies), or (b) primarily
for and in connection with specific products or compilations that
contain the covered work, unless you entered into that arrangement,
or that patent license was granted, prior to 28 March 2007.

  Nothing in this License shall be construed as excluding or limiting
any implied license or other defenses to infringement that may
otherwise be available to you under applicable patent law.

\medskip\para{\normalsize 12. No Surrender of Others' Freedom.}

  If conditions are imposed on you (whether by court order, agreement or
otherwise) that contradict the conditions of this License, they do not
excuse you from the conditions of this License.  If you cannot convey a
covered work so as to satisfy simultaneously your obligations under this
License and any other pertinent obligations, then as a consequence you
may not convey it at all.  For example, if you agree to terms that
obligate you to collect a royalty for further conveying from those to
whom you convey the Program, the only way you could satisfy both those
terms and this License would be to refrain entirely from conveying the
Program.

\medskip\para{\normalsize 13. Use with the GNU Affero General Public
License.}

  Notwithstanding any other provision of this License, you have
permission to link or combine any covered work with a work licensed
under version 3 of the GNU Affero General Public License into a single
combined work, and to convey the resulting work.  The terms of this
License will continue to apply to the part which is the covered work,
but the special requirements of the GNU Affero General Public License,
section 13, concerning interaction through a network will apply to the
combination as such.

\medskip\para{\normalsize 14. Revised Versions of this License.}

  The Free Software Foundation may publish revised and/or new versions
of the GNU General Public License from time to time.  Such new versions
will be similar in spirit to the present version, but may differ in
detail to address new problems or concerns.

  Each version is given a distinguishing version number.  If the
Program specifies that a certain numbered version of the GNU General
Public License ``or any later version'' applies to it, you have the
option of following the terms and conditions either of that numbered
version or of any later version published by the Free Software
Foundation.  If the Program does not specify a version number of the
GNU General Public License, you may choose any version ever published
by the Free Software Foundation.

  If the Program specifies that a proxy can decide which future
versions of the GNU General Public License can be used, that proxy's
public statement of acceptance of a version permanently authorizes you
to choose that version for the Program.

  Later license versions may give you additional or different
permissions.  However, no additional obligations are imposed on any
author or copyright holder as a result of your choosing to follow a
later version.

\medskip\para{\normalsize 15. Disclaimer of Warranty.}

  THERE IS NO WARRANTY FOR THE PROGRAM, TO THE EXTENT PERMITTED BY
APPLICABLE LAW.  EXCEPT WHEN OTHERWISE STATED IN WRITING THE COPYRIGHT
HOLDERS AND/OR OTHER PARTIES PROVIDE THE PROGRAM ``AS IS'' WITHOUT
WARRANTY OF ANY KIND, EITHER EXPRESSED OR IMPLIED, INCLUDING, BUT NOT
LIMITED TO, THE IMPLIED WARRANTIES OF MERCHANTABILITY AND FITNESS FOR A
PARTICULAR PURPOSE.  THE ENTIRE RISK AS TO THE QUALITY AND PERFORMANCE
OF THE PROGRAM IS WITH YOU.  SHOULD THE PROGRAM PROVE DEFECTIVE, YOU
ASSUME THE COST OF ALL NECESSARY SERVICING, REPAIR OR CORRECTION.

\medskip\para{\normalsize 16. Limitation of Liability.}

  IN NO EVENT UNLESS REQUIRED BY APPLICABLE LAW OR AGREED TO IN WRITING
WILL ANY COPYRIGHT HOLDER, OR ANY OTHER PARTY WHO MODIFIES AND/OR
CONVEYS THE PROGRAM AS PERMITTED ABOVE, BE LIABLE TO YOU FOR DAMAGES,
INCLUDING ANY GENERAL, SPECIAL, INCIDENTAL OR CONSEQUENTIAL DAMAGES
ARISING OUT OF THE USE OR INABILITY TO USE THE PROGRAM (INCLUDING BUT
NOT LIMITED TO LOSS OF DATA OR DATA BEING RENDERED INACCURATE OR LOSSES
SUSTAINED BY YOU OR THIRD PARTIES OR A FAILURE OF THE PROGRAM TO OPERATE
WITH ANY OTHER PROGRAMS), EVEN IF SUCH HOLDER OR OTHER PARTY HAS BEEN
ADVISED OF THE POSSIBILITY OF SUCH DAMAGES.

\medskip\para{\normalsize 17. Interpretation of Sections 15 and 16.}

  If the disclaimer of warranty and limitation of liability provided
above cannot be given local legal effect according to their terms,
reviewing courts shall apply local law that most closely approximates
an absolute waiver of all civil liability in connection with the
Program, unless a warranty or assumption of liability accompanies a
copy of the Program in return for a fee.

\medskip\para{\normalsize END OF TERMS AND CONDITIONS}

\newpage

\medskip\para{\normalsize How to Apply These Terms to Your New Programs}

  If you develop a new program, and you want it to be of the greatest
possible use to the public, the best way to achieve this is to make it
free software which everyone can redistribute and change under these
terms.

  To do so, attach the following notices to the program.  It is safest
to attach them to the start of each source file to most effectively
state the exclusion of warranty; and each file should have at least
the ``copyright'' line and a pointer to where the full notice is found.

\begin{verbatim}
   <one line to give the program's name and a brief idea of what it does.>
   Copyright (C) <year>  <name of author>

   This program is free software: you can redistribute it and/or modify
   it under the terms of the GNU General Public License as published by
   the Free Software Foundation, either version 3 of the License, or
   (at your option) any later version.

   This program is distributed in the hope that it will be useful,
   but WITHOUT ANY WARRANTY; without even the implied warranty of
   MERCHANTABILITY or FITNESS FOR A PARTICULAR PURPOSE.  See the
   GNU General Public License for more details.

   You should have received a copy of the GNU General Public License
   along with this program.  If not, see <http://www.gnu.org/licenses/>.
\end{verbatim}

Also add information on how to contact you by electronic and paper mail.

  If the program does terminal interaction, make it output a short
notice like this when it starts in an interactive mode:

\begin{verbatim}
   <program>  Copyright (C) <year>  <name of author>
   This program comes with ABSOLUTELY NO WARRANTY; for details type `show w'.
   This is free software, and you are welcome to redistribute it
   under certain conditions; type `show c' for details.
\end{verbatim}

\noindent
The hypothetical commands `show w' and `show c' should show the
appropriate parts of the General Public License.  Of course, your
program's commands might be different; for a GUI interface, you would
use an ``about box''.

  You should also get your employer (if you work as a programmer) or
school, if any, to sign a ``copyright disclaimer'' for the program, if
necessary. For more information on this, and how to apply and follow the
GNU GPL, see \verb|<http://www.gnu.org/licenses/>|.

  The GNU General Public License does not permit incorporating your
program into proprietary programs.  If your program is a subroutine
library, you may consider it more useful to permit linking proprietary
applications with the library.  If this is what you want to do, use the
GNU Lesser General Public License instead of this License.  But first,
please read \verb|<http://www.gnu.org/philosophy/why-not-lgpl.html>|.

\end{footnotesize}

%* eof *%


\end{document}
