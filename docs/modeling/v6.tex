\documentclass{article}
\usepackage[utf8]{inputenc}
\usepackage[margin=1in]{geometry}
\usepackage{booktabs}
\usepackage{graphicx}
\usepackage[T1]{fontenc}
\usepackage{amsmath}

%% header and footer
\usepackage{fancyhdr}
\pagestyle{fancy}
\lhead{Senior Design CSCI 475}
\rhead{Team lin[t]ers}
\cfoot{\thepage}

\title{
    \vspace{-2.5em} Production Line Scheduling - Integer Program Model
    \vspace{-2em}
}
\author{}
\date{}

\begin{document}
\maketitle
\thispagestyle{fancy}
% \setlength{\parindent}{2ex}
\setlength{\parskip}{1ex}

\section*{Summary}

Describe the problem briefly

\section*{Problem Statement}
The above problem can be formulated mathematically as a linear programming problem using the following model.

\subsection*{\textsc{Inputs}}
\begin{tabular}{@{}cl@{}}
\textbf{Symbol} & \textbf{Description} \\[0.3em]
    $n$ & number of batches \\[0.3em]
    $D_i$ & deadline of batch $i$ \\[0.3em]
    $T_{p,i}$ & time needed to produce batch $i$ \\[0.3em]
    $D_s$ & time to start scheduling \\[0.3em]
    $D_l$ & last deadline, $\texttt{max($D_i$)}$ \\[0.3em]
    $T_r$ & total time available, $D_l - D_s$ \\[0.3em]
    $A_{ij}$ & changeover time between batches $i$, $j$ due to change in allergens \\[0.3em]
    $K_{ij}$ & changeover time between batches $i$, $j$ due to kosher status switch \\[0.3em]
    $C_{ij}$ & additional changeover time between batches $i$, $j$ \\[0.3em]
\end{tabular}

\subsection*{\textsc{Variables}}

\begin{tabular}{@{}cl@{}}
\textbf{Symbol} & \textbf{Description} \\[0.3em]
    $T_{s,i}$ & start time of batch $i$ \\[0.3em]
    $P_{ij}$ & $\begin{cases}
                    1 & T_{s,i} < T_{s,j}\\
                    0 & T_{s,i} > T_{s,j}
                \end{cases}$ \\[1.3em]
    $d_i$ & day on which batch $i$ is scheduled \\[0.3em]
    $t_f$ & \begin{tabular}[t]{@{}l}
                    finish time of all batches\\
                    \texttt{max($T_{s,i} + T_{p,i}$)}
            \end{tabular} \\[0.3em]
\end{tabular}

\subsection*{\textsc{Constraints}}
\subsubsection*{Deadline and Overlapping}
\noindent This ensures that the deadline is met for batch $i$.

$T_{s,i} + T_{p,i} \leq D_i$

$0 \leq T_{s,i} \leq T_r$

\noindent This ensures that any two batches $i$ and $j$ are not separated by more than the total time available for production.

$-T_r \times P_{ij} \leq T_{s,i} - T_{s,j} \leq T_r \times (1 - P_{ij})$

\noindent This ensures that the production times of any two batches $i$ and $j$ do not overlap.

$T_{s,j} - (T_{s,i} + T_{p,i}) \geq T_r \times (P_{ij} - 1)$

$(T_{s,j} + T_{p,j}) - T_{s,i} \leq T_r \times P_{ij}$

\subsubsection*{Changeover Period}
\noindent This ensures that batch $j$ starts only after the various changeover periods of batch $i$.

$(T_{s,j} - (T_{s,i} + T_{p,i})) + T_r \times (1 - P_{ij}) \geq C_{ij} + A_{ij} + K_{ij}$

\subsubsection*{Shifts}
\noindent This ensures that no start or finish time lies outside the workday or during the weekends. Here, $d_i$ can only take on values which correspond to valid workdays. For example, if you begin scheduling today (say, a Tuesday) and end on Monday the following week, $d_i \in \{0,\, 1,\, 2,\, 3,\, 6\}$.

$24 \cdot d_i + 8 \leq T_{s,i} \leq 24 \cdot d_i + 16$

$24 \cdot d_i + 8 \leq T_{s,i} + T_{p,i} \leq 24 \cdot d_i + 16$

\subsection*{\textsc{Objective}}
minimize $t_f$

\end{document}
